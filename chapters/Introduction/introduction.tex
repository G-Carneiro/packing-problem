% --------------------------------------------
% Aqui você deve organizar as seções.
% --------------------------------------------
\chapter*[Introdução]{Introdução}\label{ch:introducao}
\addcontentsline{toc}{chapter}{Introdução}

Este trabalho visa estudar o Problema de Empacotamento de peças retangulares em uma caixa também retangular no espaço de duas dimensões, sendo sua solução considerada NP-difícil \cite{2DPackLib}.
Antes de abordar o problema (\autoref{ch:problema-de-empacotamento}) e buscar soluções alguns conceitos básicos são mostrados na \autoref{ch:conceitos-basicos}.

O \autoref{ch:conceitos-basicos} foca em definições sobre otimização (\autoref{sec:definicoes}) e em modelos de otimização (\autoref{sec:modelos-de-otimizacao} e \autoref{sec:tipos-de-modelo}).
No \autoref{ch:problema-de-empacotamento} é dada a definição do problema (\autoref{sec:definicao}), para então mostrar algumas classificações (\autoref{sec:classificacao}) e variantes (\autoref{sec:variantes}), por fim é explicada a heurística \textit{bottom-left} (\autoref{sec:bottom-left}), a qual será utilizada na resolução das instâncias de teste.

O problema tem várias a aplicações nas indústrias de móveis, têxtil e metal-mecânica \cite{queiroz2022estudo, cavali2004problemas, belluzzo2005otimizacao}, além ser extramente útil em carregamento de paletes e contêineres \cite{morabito1992abordagem}.
É possível dividir o problema de acordo com sua dimensão.

Problemas unidimensionais podem ser associados ao corte de barras ou canos, para atender uma demanda por peças de diferentes tamanhos.
As indústrias de tecido ou couro usam o caso 2D para minimizar o desperdício ao se cortar suas peças.
O caso 3D é fácilmente associável ao carregamento de contêineres, onde objetos são geralmente caixas a serem alocadas em algum veículo.
A \autoref{fig:packing-example} mostra um exemplo para cada dimensão do problema.

\begin{figure}[!htb]
    \centering
    \caption{Representação para o problema de empacotamento 1D, 2D e 3D.}
    \includegraphics[scale=1]{utils/images/packing-example}
    \label{fig:packing-example}
    \fonte{\citeauthoryear{castellucci2019consolidation}.}
\end{figure}


Basicamente, pode-se aplicá-lo em qualquer área que precise de organização ou logística, bem como situações que envolvam o corte de algum material.
Ao utilizar soluções para resolver problemas de empacotamento, é possível reduzir o desperdício de materiais e impacto ambiental, diminuir tempo de entregas e otimizar espaços de estoque.

% --------------------------------------------
% Aqui você deve organizar as subsubseções.
% --------------------------------------------
\section*{Objetivos}\label{sec:Objetivos}

O principal objetivo deste trabalho é estudar e compreender o problema de empacotamento bem como suas aplicações no mundo real.
Outros objetivos mais específicos são: revisar a bibliografia, implementar heurísticas baseadas em \textit{bottom-left}, definir instâncias de teste e analisar os dados obtidos e compará-los com os de outros autores.

