\section{Comparativo entre criação de regiões}\label{sec:comparativo-entre-criacao-de-regioes}

O comparativo entre métodos de criação de regiões gerou resultados interessantes,
na \Cref{tab:regioes} ocorre algo semelhante ao da \Cref{tab:ordenacoes}.
Regiões criadas de modo a ser necessário verificar sobreposições obtiveram excelentes resultados
qualitativos (última linha), pois são pouco afetadas pela ordenação crescente
(\cref{sec:ordenacao-crescente-decrescente}), assim como o critério de ordenação por \textit{id},
já que as regiões sempre possuem área máxima (\cref{sec:criacao-de-regioes}).

\begin{table}[!htb]
    \centering
    \caption{Resultado da comparação entre criação de regiões.}
    \label{tab:regioes}
    \IBGEtab{}{
        \ttfamily\begin{tabular}{lrrrr}
\hline
Divisão     & Vitórias & Empates & Qualidade \% & Tempo (s)  \\
\hline
Vertical    & 155      & 102     & 65.0025      & 2.4311e-03 \\
Horizontal  & 122      & 101     & 63.0163      & 7.2323e-03 \\
Maior área  & 189      & 158     & 69.5409      & 1.3313e-02 \\
Sobrepostas & 334      & 195     & 75.0333      & 8.4208e+00 \\
\hline
\end{tabular}
    }{}
    \fonte{feito pelo autor.}
\end{table}

Ao considerar somente a ordenação decrescente (\Cref{tab:ordenacoes-true}), a diferença em relação
aos demais métodos de criação diminui, ainda que o último modo permaneça sendo o melhor qualitativa
e quantitativamente.
Os modos criados traçando uma linha vertical ou horizontal apresentaram qualidades semelhantes e
os menores tempos de execução, mas o método o qual traça uma linha vertical obteve mais vitórias.
Regiões criadas para maximizar uma das mesmas conseguiram o segundo melhor resultado qualitativo e
quantitativo, ao custo de um pequeno acréscimo no tempo de execução em relação aos dois primeiros.
O último modo de fato conseguiu os melhores resultados, porém a um custo altíssimo, levando cerca
de 1000 vezes mais tempo que métodos mais rápidos.

\begin{table}[!htb]
    \centering
    \caption{Resultado da comparação entre criação de regiões.}
    \label{tab:regioes-true}
    \IBGEtab{}{
        \ttfamily\begin{tabular}{lrrrrr}
\hline
Região & Wons & Draws & Quality \% & Items \% & Time (s)   \\
\hline
V      & 98   & 79    & 76.4030    & 45.0191  & 2.7157e-03 \\
H      & 70   & 60    & 75.9970    & 45.5439  & 6.2101e-03 \\
M      & 104  & 89    & 79.7175    & 47.6795  & 1.3743e-02 \\
N      & 176  & 119   & 83.6420    & 47.2335  & 7.2176e+00 \\
\hline
\end{tabular}
    }{}
    \fonte{autor}
\end{table}

A \Cref{tab:superposition} traz um comparativo entre os dois tipos de criação de regiões,
os que é preciso checar sobreposição e os que não (\cref{sec:criacao-de-regioes}).
Na primeira linha da tabela a coluna “Qualidade \%” representa a média do melhor resultado obtido
em cada instância e a coluna “Tempo Total (s)” mostra a soma dos tempos que cada método de solução
levou para cada instância.
As duas colunas consideram somente métodos de solução que usam regiões onde não são necessárias
verificações de sobreposição, ou seja, são considerados 30 modos de solução.
A segunda linha considera somente método de solução o qual utiliza ordenação decrescente pela área
e criação de regiões onde é necessário verificar sobreposições, esse método foi escolhido para o
comparativo por apresentar os melhores resultados qualitativos (\cref{sec:comparativo-entre-combinacoes}).

\begin{table}[!htb]
    \centering
    \caption{Resultado da comparação entre tipos de regiões.}
    \label{tab:superposition}
    \IBGEtab{}{
        \ttfamily\begin{tabular}{lrrr}
    \hline
    Superposition & Quality \% & Time (s)   \\
    \hline
    No            & 90.8278    & 1.6299e+01 \\
    Yes           & 87.2957    & 2.8313e+02
    \hline
\end{tabular}
    }{}
    \fonte{autor}
\end{table}

Com a \Cref{tab:superposition} fica claro que, por mais que usar regiões onde é preciso checar
sobreposições apresente o melhor resultado, é melhor executar todos demais métodos os quais não
precisem e escolher somente o de melhor solução, pois assim é possível obter, na média, soluções
de maior qualidade e ainda levando 10 vezes menos tempo.