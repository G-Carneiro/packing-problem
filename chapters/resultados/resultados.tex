\chapter{Resultados}\label{ch:resultados}

Para testar os métodos de solução criados foram usadas 45 instâncias de teste da literatura,
separadas em cinco conjuntos de instância de características diferentes:
BKW, GCUT, NGCUT, OF e OKP\@.
Todas as elas foram obtidas através da biblioteca pública
\href{https://site.unibo.it/operations-research/en/research/2dpacklib}{2DPackLib}.

O foco do trabalho é no 2D-KP e com o critério de maximização sendo a área ocupada do espaço.
Mas nem todos conjuntos foram feitos para ser resolvidos dessa forma, nesses casos foram
feitas leves adaptações para usá-los.
O motivo de usar instâncias feitas com outro objetivo é para não viciar o modelo em instâncias
específicas.

As instâncias BKW foram propostas para 2D-SPP~\cite{burke2004new}, esse conjunto é interessante, pois
existe uma solução ótima onde todos os itens podem ser alocados.
\citeauthor*{cote2014combinatorial} apresentam alguns resultados para esse conjunto, já
\citeauthor*{delorme2017logic} trazem resultados usando rotações ortogonais.

Instâncias GCUT foram propostas para 2D-KP com corte guilhotinados~\cite{beasley1985algorithms},
como o trabalho não usa variantes será usada sem os cortes.
Esse conjunto já foi usado na literatura no 2D-SPP~\cite{cote2014combinatorial},
2D-SPP com corte guilhotinado~\cite{mrad2015arc}, 2D-SPP com rotações ortogonais~\cite{delorme2017logic}
e 2D-CSP com corte guilhotinado~\cite{cintra2008algorithms}.

NGCUT é um conjunto proposto para 2D-KP~\cite{beasley1985exact}.
Ele possui resultados recentes para 2D-KP, 2D-SPP~\cite{cote2014combinatorial} e 2D-SSP com
rotações ortogonais~\cite{delorme2017logic}.

As instâncias OF foram inicialmente feitas para 2D-KP com cortes
guilhotinados~\cite{oliveira1990improved} e foram resolvidas recentemente com o mesmo propósito
~\cite{velasco2019improved, martin2020models}.

Por fim, instâncias OKP foram criadas para 2D-KP~\cite{fekete1997new} e já foram resolvidas para
versões sem e com corte guilhotinado~\cite{furini2016modeling}.