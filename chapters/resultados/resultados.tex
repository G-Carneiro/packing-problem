\chapter{Resultados}\label{ch:resultados}

Para testar os métodos de solução criados foram usadas 45 instâncias de teste da literatura,
separadas em cinco conjuntos de instância de características diferentes:
BKW, GCUT, NGCUT, OF e OKP\@.
Todas as elas foram obtidas através da biblioteca pública
\href{https://site.unibo.it/operations-research/en/research/2dpacklib}{2DPackLib}\footnote{
    Disponível em: https://site.unibo.it/operations-research/en/research/2dpacklib.
    Acessado em: \today.}~\cite{2DPackLib}.

O foco do trabalho é no 2D-KP e com o critério de maximização sendo a área ocupada do espaço.
Mas nem todos conjuntos foram feitos para ser resolvidos dessa forma, nesses casos foram
feitas leves adaptações para usá-los.
O motivo de usar instâncias feitas com outro objetivo é para não viciar o modelo em instâncias
específicas.

As instâncias BKW foram propostas para 2D-SPP~\cite{burke2004new}, esse conjunto é interessante, pois
existe uma solução ótima onde todos os itens podem ser alocados.
\citeauthoryear{cote2014combinatorial} apresentam alguns resultados para esse conjunto, já
\citeauthoryear{delorme2017logic} trazem resultados usando rotações ortogonais.

Instâncias GCUT foram propostas para 2D-KP com corte guilhotinados~\cite{beasley1985algorithms}.
Esse conjunto já foi usado na literatura no 2D-SPP~\cite{cote2014combinatorial},
2D-SPP com corte guilhotinado~\cite{mrad2015arc}, 2D-SPP com rotações ortogonais~\cite{delorme2017logic}
e 2D-CSP com corte guilhotinado~\cite{cintra2008algorithms}.

NGCUT é um conjunto proposto para 2D-KP~\cite{beasley1985exact}.
Ele possui resultados recentes para 2D-KP, 2D-SPP~\cite{cote2014combinatorial} e 2D-SPP com
rotações ortogonais~\cite{delorme2017logic}.

As instâncias OF foram inicialmente elaboradas para 2D-KP com cortes
guilhotinados~\cite{oliveira1990improved} e foram resolvidas recentemente com o mesmo propósito
~\cite{velasco2019improved, martin2020models}.

Por fim, instâncias OKP foram criadas para 2D-KP~\cite{fekete1997new} e já foram resolvidas para
versões sem e com corte guilhotinado~\cite{furini2016modeling}.

Como são cinco critérios de ordenação~(\cref{sec:criterios-de-ordenacao}), com cada critério
podendo ser crescente ou decrescente, quatro formas de criar regiões~(\cref{sec:criacao-de-regioes})
e 45 instâncias, tem-se o total de 1800 casos de teste.
Além disso, para conseguir resultados mais fiéis, a média, mediana e desvio padrão foram calculados.
Por isso, cada caso foi executado cinco vezes, totalizando 9000 execuções.
Outros dados como a qualidade de solução (objetivo do trabalho), porcentagem de itens alocados
e tempo, também foram computados.
Todas as execuções foram feitas em um mesmo computador, com configurações conforme a
\Cref{tab:config}.

\begin{table}
    \centering
    \caption{Configuração do computador de testes.}
    \label{tab:config}
    \begin{tabular}{ll}
        \hline
        CPU    & AMD Ryzen™ 5 3600X       \\
        RAM    & 16 GiB                   \\
        Python & 3.11.0                   \\
        SO     & Linux Mint 21.1 Cinnamon \\
        Kernel & 5.15                     \\
        \hline
    \end{tabular}
\end{table}


Ao analisar a média, a mediana e o desvio padrão do tempo de execução, observou-se que a média
e mediana possuem valores quase idênticos, enquanto o desvio padrão é pequeno ao ponto de poder
ser ignorado, indicando que cinco execuções por caso de teste são suficientes.
Portanto, a mediana e desvio padrão serão omitidos no restante do trabalho, podendo ser encontrados
na versão completa dos dados gerados no \href{https://github.com/G-Carneiro/packing-problem/}{
    Github}\footnote{Disponível em: https://github.com/G-Carneiro/packing-problem/. Acessado em: \today.}.
No \Cref{ch:resultados-das-instancias} é possível ver uma versão resumida de todos os dados gerados
(sem mediana e desvio padrão) para cada instância.

Nas tabelas apresentadas nas seções seguintes, as colunas “Vitórias” e “Empates” trazem os
resultados de forma quantitativa, enquanto a coluna “Qualidade \%” de modo qualitativo.
Nos testes, o valor $p_i$ dos itens sempre é sua área, desconsiderando os valores dados pelas
instâncias, caso hajam.

% TODO: falar sobre tempo de execução
A primeira coisa a qual fica evidente com os resultados é a discrepância na qualidade de solução
entre a ordenação crescente e a decrescente, algo já esperado.
Na \Cref{tab:combinations} é possível notar que ordenando de forma decrescente é possível ocupar
cerca de $20\%$ a mais do espaço (coluna “Qualidade \%”), quando comparado a ordenação crescente.

\begin{table}[!htb]
    \centering
    \caption{Resultado da comparação entre Descending.}
    \label{tab:descending}
    \IBGEtab{}{
        \begin{tabular}{lrrrrr}
            \hline
            Descending & Wons & Draws & Quality \% & Items \% & Time (s) \\
            \hline
            F          & 167  & 8     & 57.306     & 47.6518  & 2.37153  \\
            T          & 736  & 8     & 78.9136    & 46.3642  & 1.77985  \\
            \hline
        \end{tabular}
    }{}
    \fonte{autor}
\end{table}

A coluna “Vitórias” indica quantas vezes tal método de solução obteve o melhor resultado em
comparação com os demais métodos em outras linhas.
Enquanto a coluna “Empates” mostra a quantidade de vezes que o método conseguiu a melhor qualidade,
mas outros também conseguiram.
Essas colunas foram feitas da seguinte forma: entre cada combinação de critério de ordenação,
modo de criar regiões e instâncias, é feita a comparação se a qualidade de solução foi melhor
para ordenação crescente ou decrescente.
No caso de ambas conseguirem o melhor resultado, é acrescido 1 tanto na coluna “Vitórias”, quanto
na “Empates” de ambas.
Por fim, a coluna “Tempo (s)” mostra o tempo médio de execução do método.

Com isso, fica claro que ordenar a fila de entrada da \textit{bottom-left} de modo decrescente é
vantajoso em termos de qualidade, quantidade e tempo de execução.
\section{Comparativo entre critérios de ordenação}\label{sec:comparativo-entre-criterios-de-ordenacao}

Ao comparar os diferentes critérios de ordenação (\Cref{tab:ordenacoes}), inicialmente tem-se
um resultado curioso, a ordenação por \textit{id} (nenhuma ordenação) está com os melhores
resultados qualitativos.
Além disso, todos os critérios conseguiram praticamente a mesma qualidade média.

\begin{table}[!htb]
    \centering
    \caption{Resultado da comparação entre critérios de ordenação.}
    \label{tab:ordenacoes}
    \IBGEtab{}{
        \ttfamily\begin{tabular}{lrrrr}
 \hline
 Ordenação & Vitórias & Empates & Qualidade \% & Tempo (s)  \\
 \hline
 Área      & 81       & 56      & 67.7915      & 2.0726e+00 \\
 Perímetro & 93       & 52      & 68.9870      & 2.0366e+00 \\
 Altura    & 73       & 24      & 65.9848      & 1.8254e+00 \\
 Largura   & 94       & 32      & 69.1286      & 2.3444e+00 \\
 Id        & 127      & 12      & 68.6572      & 2.0995e+00 \\
 \hline
\end{tabular}
    }{}
    \fonte{feito pelo autor.}
\end{table}

Como mencionado anteriormente (\cref{sec:ordenacao-crescente-decrescente}), ordenar a fila de
forma crescente gera resultados ruins.
Mas isso não ocorre quando o critério de ordenação é o \textit{id}, porque nele tem-se peças
sem seguirem alguma ordem.
Com esse critério o impacto da escolha entre ordenação crescente ou decrescente é praticamente
nulo.
Por isso ele também se saí muito melhor que os demais na ordenação crescente, sendo o vitorioso
na esmagadora maioria.
Já as qualidades próximas se justificam nos critérios diferentes do \textit{id} pois a
discrepância entre ordenação crescente e decrescente é muito grande, mas ao tirar média isso acaba
não ficando evidente.

A \Cref{tab:ordenacoes-true} mostra o mesmo comparativo que a \Cref{tab:ordenacoes}, porém agora
somente considerando a ordenação decrescente, já que não existem motivos para usar a crescente.
Representando os dados dessa forma fica fácil identificar que utilizar algum critério de ordenação
para a fila de entrada é vantajoso, pois ao usar o \textit{id} os resultados foram os piores.
Além disso, percebe-se que as ordenações por área e perímetro obtiveram os melhores resultados,
ainda que os demais também sejam competitivos.

\begin{table}[!htb]
    \centering
    \caption{Resultado da comparação entre critérios de ordenação decrescente.}
    \label{tab:ordenacoes-true}
    \IBGEtab{}{
        \ttfamily\begin{tabular}{lrrrr}
    \hline
    Ordenação & Vitórias & Empates & Qualidade \% & Tempo (s)  \\
    \hline
    Área      & 63       & 39      & 82.7353      & 1.5874e+00 \\
    Perímetro & 71       & 38      & 84.6986      & 1.5769e+00 \\
    Altura    & 40       & 16      & 77.4182      & 1.5655e+00 \\
    Largura   & 66       & 24      & 81.1899      & 2.0805e+00 \\
    Id        & 16       & 5       & 68.5261      & 2.0889e+00 \\
    \hline
\end{tabular}
    }{}
    \fonte{feito pelo autor.}
\end{table}

% TODO: perguntar ao Pedro oq acha do parágrafo
A alta competitividade entre os critérios de ordenação é interessante, pois a maioria dos trabalhos
na literatura, como o de \citeauthor*{chen2019efficient}~\citeyear{chen2019efficient}, usam somente
ordenação pela área, e isso pode ser um forte indicativo que os demais critérios devem ser mais
explorados em certas circunstâncias.
Ao considerar somente a ordenação decrescente deixa claro que a alta quantidade vitórias do critério
\textit{id} na \Cref{tab:ordenacoes} realmente se deve a ordenação crescente.

\section{Comparativo entre criação de regiões}\label{sec:comparativo-entre-criacao-de-regioes}

O comparativo entre métodos de criação de regiões gerou resultados interessantes,
na \Cref{tab:regioes} ocorre algo semelhante ao da \Cref{tab:ordenacoes}.
Regiões criadas de modo a ser necessário verificar sobreposições obtiveram excelentes resultados
qualitativos (última linha), pois são pouco afetadas pela ordenação crescente
(\cref{sec:ordenacao-crescente-decrescente}), assim como o critério de ordenação por \textit{id},
já que as regiões sempre possuem área máxima (\cref{sec:criacao-de-regioes}).

\begin{table}[!htb]
    \centering
    \caption{Resultado da comparação entre criação de regiões.}
    \label{tab:regioes}
    \IBGEtab{}{
        \ttfamily\begin{tabular}{lrrrr}
\hline
Divisão     & Vitórias & Empates & Qualidade \% & Tempo (s)  \\
\hline
Vertical    & 155      & 102     & 65.0025      & 2.4311e-03 \\
Horizontal  & 122      & 101     & 63.0163      & 7.2323e-03 \\
Maior área  & 189      & 158     & 69.5409      & 1.3313e-02 \\
Sobrepostas & 334      & 195     & 75.0333      & 8.4208e+00 \\
\hline
\end{tabular}
    }{}
    \fonte{feito pelo autor.}
\end{table}

Ao considerar somente a ordenação decrescente (\Cref{tab:ordenacoes-true}), a diferença em relação
aos demais métodos de criação diminui, ainda que o último modo permaneça sendo o melhor qualitativa
e quantitativamente.
Os modos criados traçando uma linha vertical ou horizontal apresentaram qualidades semelhantes e
os menores tempos de execução, mas o método o qual traça uma linha vertical obteve mais vitórias.
Regiões criadas para maximizar uma das mesmas conseguiram o segundo melhor resultado qualitativo e
quantitativo, ao custo de um pequeno acréscimo no tempo de execução em relação aos dois primeiros.
O último modo de fato conseguiu os melhores resultados, porém a um custo altíssimo, levando cerca
de 1000 vezes mais tempo que métodos mais rápidos.

\begin{table}[!htb]
    \centering
    \caption{Resultado da comparação entre criação de regiões.}
    \label{tab:regioes-true}
    \IBGEtab{}{
        \ttfamily\begin{tabular}{lrrrrr}
\hline
Região & Wons & Draws & Quality \% & Items \% & Time (s)   \\
\hline
V      & 98   & 79    & 76.4030    & 45.0191  & 2.7157e-03 \\
H      & 70   & 60    & 75.9970    & 45.5439  & 6.2101e-03 \\
M      & 104  & 89    & 79.7175    & 47.6795  & 1.3743e-02 \\
N      & 176  & 119   & 83.6420    & 47.2335  & 7.2176e+00 \\
\hline
\end{tabular}
    }{}
    \fonte{autor}
\end{table}

A \Cref{tab:superposition} traz um comparativo entre os dois tipos de criação de regiões,
os que é preciso checar sobreposição e os que não (\cref{sec:criacao-de-regioes}).
Na primeira linha da tabela a coluna “Qualidade \%” representa a média do melhor resultado obtido
em cada instância e a coluna “Tempo Total (s)” mostra a soma dos tempos que cada método de solução
levou para cada instância.
As duas colunas consideram somente métodos de solução que usam regiões onde não são necessárias
verificações de sobreposição, ou seja, são considerados 30 modos de solução.
A segunda linha considera somente método de solução o qual utiliza ordenação decrescente pela área
e criação de regiões onde é necessário verificar sobreposições, esse método foi escolhido para o
comparativo por apresentar os melhores resultados qualitativos (\cref{sec:comparativo-entre-combinacoes}).

\begin{table}[!htb]
    \centering
    \caption{Resultado da comparação entre tipos de regiões.}
    \label{tab:superposition}
    \IBGEtab{}{
        \ttfamily\begin{tabular}{lrrr}
    \hline
    Superposition & Quality \% & Time (s)   \\
    \hline
    No            & 90.8278    & 1.6299e+01 \\
    Yes           & 87.2957    & 2.8313e+02
    \hline
\end{tabular}
    }{}
    \fonte{autor}
\end{table}

Com a \Cref{tab:superposition} fica claro que, por mais que usar regiões onde é preciso checar
sobreposições apresente o melhor resultado, é melhor executar todos demais métodos os quais não
precisem e escolher somente o de melhor solução, pois assim é possível obter, na média, soluções
de maior qualidade e ainda levando 10 vezes menos tempo.
\section{Comparativo entre combinações}\label{sec:comparativo-entre-combinacoes}

A \Cref{tab:combinations} contém os resultados para cada um dos quarenta métodos de solução feitos.
Com ela é possível identificar qual dos métodos apresenta melhores resultados qualitativos e
quantitativos.

\begin{table}[!htb]
    \centering
    \caption{Resultado da comparação entre ['Split', 'Order', 'Descending', 'Wons', 'Draws', 'Quality \%', 'Items \%', 'Time (s)'].}
    \label{tab:combinations}
    \IBGEtab{}{
        \ttfamily\begin{tabular}{lllrrrrr}
    \hline
    Split & Order & Descending & Wons & Draws & Quality \% & Items \% & Time (s)   \\
    \hline
    V     & A     & F          & 0    & 0     & 50.9443    & 47.8026  & 0.00248052 \\
    V     & A     & T          & 6    & 6     & 78.9961    & 43.2429  & 0.00308339 \\
    V     & P     & F          & 0    & 0     & 51.2033    & 46.2057  & 0.00214879 \\
    V     & P     & T          & 7    & 6     & 82.621     & 45.6727  & 0.00232849 \\
    V     & H     & F          & 0    & 0     & 55.2624    & 48.4624  & 0.00201776 \\
    V     & H     & T          & 4    & 4     & 70.7811    & 42.5165  & 0.00253342 \\
    V     & W     & F          & 0    & 0     & 47.9606    & 42.6878  & 0.00166203 \\
    V     & W     & T          & 9    & 8     & 84.5497    & 47.058   & 0.002482   \\
    V     & I     & F          & 1    & 1     & 62.6394    & 46.0333  & 0.00242357 \\
    V     & I     & T          & 1    & 1     & 65.067     & 46.6058  & 0.00315097 \\
    H     & A     & F          & 1    & 1     & 44.9575    & 42.0545  & 0.0088805  \\
    H     & A     & T          & 5    & 5     & 81.5022    & 43.6548  & 0.00599634 \\
    H     & P     & F          & 0    & 0     & 45.0368    & 41.4449  & 0.00852498 \\
    H     & P     & T          & 4    & 3     & 82.639     & 43.4046  & 0.00475726 \\
    H     & H     & F          & 2    & 2     & 43.4125    & 35.4793  & 0.00770928 \\
    H     & H     & T          & 4    & 4     & 79.2274    & 47.835   & 0.00614419 \\
    H     & W     & F          & 0    & 0     & 51.7897    & 47.2147  & 0.0100625  \\
    H     & W     & T          & 4    & 4     & 74.9317    & 45.5948  & 0.00761566 \\
    H     & I     & F          & 2    & 2     & 64.9816    & 47.1122  & 0.00609561 \\
    H     & I     & T          & 1    & 1     & 61.6848    & 47.23    & 0.006537   \\
    M     & A     & F          & 2    & 2     & 53.1636    & 50.7883  & 0.0172841  \\
    M     & A     & T          & 7    & 7     & 83.2483    & 45.6017  & 0.0132333  \\
    M     & P     & F          & 1    & 1     & 53.7023    & 49.7762  & 0.0176751  \\
    M     & P     & T          & 7    & 6     & 85.8682    & 46.3078  & 0.012944   \\
    M     & H     & F          & 2    & 2     & 54.8203    & 45.9835  & 0.0108422  \\
    M     & H     & T          & 4    & 4     & 78.5353    & 47.1767  & 0.0132691  \\
    M     & W     & F          & 0    & 0     & 62.4641    & 51.3552  & 0.00401001 \\
    M     & W     & T          & 5    & 5     & 79.457     & 48.8029  & 0.0148469  \\
    M     & I     & F          & 2    & 2     & 72.6713    & 50.9134  & 0.0146069  \\
    M     & I     & T          & 2    & 2     & 71.4787    & 50.5082  & 0.0144215  \\
    N     & A     & F          & 0    & 0     & 61.151     & 51.4896  & 10.1459    \\
    N     & A     & T          & 13   & 11    & 85.3558    & 42.9123  & 6.29188    \\
    N     & P     & F          & 0    & 0     & 61.975     & 51.4671  & 9.90181    \\
    N     & P     & T          & 9    & 6     & 85.784     & 42.8241  & 6.25238    \\
    N     & H     & F          & 2    & 2     & 63.498     & 49.9569  & 8.27386    \\
    N     & H     & T          & 6    & 5     & 79.4085    & 46.6447  & 6.20546    \\
    N     & W     & F          & 0    & 0     & 64.7865    & 51.0338  & 10.3591    \\
    N     & W     & T          & 16   & 10    & 84.0171    & 48.1854  & 8.25089    \\
    N     & I     & F          & 4    & 4     & 73.3323    & 50.4803  & 8.37008    \\
    N     & I     & T          & 5    & 3     & 74.351     & 50.3528  & 8.28526    \\
    \hline
\end{tabular}
    }{}
    \fonte{autor}
\end{table}

Dentre os métodos que usam regiões onde é preciso checar sobreposições, os mais interessantes são
o de ordenação decrescente pela área (linha 30), pelo perímetro (linha 32) e pela largura (linha 36).
Utilizando a área conseguiu-se o segundo melhor resultado quantitativo e também qualitativo.
Com o perímetro foi possível atingir o terceiro maior número de vitórias e a melhor qualidade de
solução.
Por fim, a largura obteve o melhor resultado quantitativo e ficou em terceiro na qualidade de solução.

Nos métodos que usam regiões mais simples, os resultados foram bem variados nas combinações,
dentre eles se destacam: regiões criadas usando linha vertical e ordenação pela largura (linha 6) e
maximizar uma região e ordenar pela área (linha 20) ou pelo perímetro (linha 22).
O método da linha 6 ficou em terceiro lugar no critério quantitativo (empate com a linha 32) e
conseguiu a quinta posição na qualidade de solução.
As linhas 20 e 22 tiveram a quarta maior quantidade de vitórias (empate também com a linha 2), já
qualitativamente a linha 20 obteve a sexta melhor média nas soluções, enquanto a linha 22 conseguiu
a quarta.

Ainda que os resultados dos métodos que usam regiões as quais calculam sobreposições sejam melhores,
não compensa utilizá-los, como visto na \cref{sec:comparativo-entre-criacao-de-regioes}.
\section{Conjuntos de instâncias}\label{sec:conjuntos-de-instancias}

\section{Complexidade}\label{sec:complexidade}

As \cref{sec:comparativo-entre-criacao-de-regioes,sec:comparativo-entre-combinacoes} deixaram claro
que regiões onde a sobreposição pode ocorrer são mais custosas no tempo de execução.
Isso acontece devido à complexidade do algoritmo de solução do modelo.

Em regiões simples, onde não podem acontecer sobreposições, é necessário verificar se o item a ser
alocado cabe em uma região.
No pior caso todas as peças só conseguiriam ser alocadas na última região checada.
É possível descobrir o número máximo de regiões disponíveis no momento de alocação do item $i$ da
fila.

Para o primeiro item, existe somente uma região, o próprio espaço.
Após alocar o primeiro item, tem-se duas regiões disponíveis para posicionar o segundo.
Ao posicionar uma peça o número de regiões é acrescido em um.
Com isso, é possível afirmar que para alocar o item $i$ será preciso verificar no máximo $i$ regiões.
Somando o máximo de regiões para cada item chega-se no número máximo de regiões a serem verificadas
no pior caso.
Considerando $n$ itens, tem-se a \Cref{eq:sum-1-to-n}.

\begin{equation}
    \label{eq:sum-1-to-n}
    \sum_{i=1}^{n} i
\end{equation}


Como a soma de 1 até $n$ pode ser reescrita utilizando a \Cref{eq:formula-sum-1-to-n}
~\cite{merca2015alternative}, tem-se que o número máximo de regiões a serem verificadas
para solucionar o modelo é $\frac{n^2 + n}{2}$.

\begin{equation}
    \label{eq:formula-sum-1-to-n}
    \sum_{k=1}^{n} k = \dfrac{n(n+1)}{2}
\end{equation}


Para regiões complexas, além desse número de regiões, ainda é necessário verificar sobreposições.
No pior caso, ao alocar o item $i$, para cada região será preciso verificar sobreposições com
todos os itens $i - 1$ (\Cref{eq:sum-i-pow-2}).

\begin{equation}
    \label{eq:sum-i-pow-2}
    \sum_{i=1}^{n} i (i - 1)
\end{equation}


Usando a Fórmula de Faulhaber~\cite{merca2015alternative} é possível reescrever a soma dos quadrados
dos $n$ primeiros números inteiros como mostra a \Cref{eq:sum-k-pow-2}.

\begin{equation}
    \label{eq:sum-k-pow-2}
    \sum_{k=1}^{n} k^2 = \dfrac{n(n+1)(2n+1)}{6}
\end{equation}


Assim, ao reescrever a \Cref{eq:sum-i-pow-2} utilizando as \Cref{eq:formula-sum-1-to-n,eq:sum-k-pow-2},
tem-se que, no pior caso, será necessário verificar se existe sobreposição entre peças
$\frac{n^3-n}{3}$ vezes.
Para $n = 3152$, seriam quase cinco milhões de regiões e mais 10 bilhões de sobreposições a serem
checadas, explicando o tempo elevado ao executar um método de solução utilizando regiões complexas.
