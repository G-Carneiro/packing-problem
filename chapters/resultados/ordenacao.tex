A primeira coisa a qual fica evidente com os resultados é a discrepância na qualidade de solução
entre a ordenação crescente e a decrescente, algo já esperado.
Na \Cref{tab:combinations} é possível notar que ordenando de forma decrescente é possível ocupar
cerca de $20\%$ a mais do espaço (coluna “Qualidade \%”), quando comparado a ordenação crescente.

\begin{table}[!htb]
    \centering
    \caption{Resultado da comparação entre Descending.}
    \label{tab:descending}
    \IBGEtab{}{
        \begin{tabular}{lrrrrr}
            \hline
            Descending & Wons & Draws & Quality \% & Items \% & Time (s) \\
            \hline
            F          & 167  & 8     & 57.306     & 47.6518  & 2.37153  \\
            T          & 736  & 8     & 78.9136    & 46.3642  & 1.77985  \\
            \hline
        \end{tabular}
    }{}
    \fonte{autor}
\end{table}

A coluna “Vitórias” indica quantas vezes tal método de solução obteve o melhor resultado em
comparação com os demais métodos em outras linhas.
Enquanto a coluna “Empates” mostra a quantidade de vezes que o método conseguiu a melhor qualidade,
mas outros também conseguiram.
Essas colunas foram feitas da seguinte forma: entre cada combinação de critério de ordenação,
modo de criar regiões e instâncias, é feita a comparação se a qualidade de solução foi melhor
para ordenação crescente ou decrescente.
No caso de ambas conseguirem o melhor resultado, é acrescido 1 tanto na coluna “Vitórias”, quanto
na “Empates” de ambas.
Por fim, a coluna “Tempo (s)” mostra o tempo médio de execução do método.

Com isso, fica claro que ordenar a fila de entrada da \textit{bottom-left} de modo decrescente é
vantajoso em termos de qualidade, quantidade e tempo de execução.