\section{Ordenação crescente × decrescente}\label{sec:ordenacao-crescente-decrescente}

A primeira coisa a qual fica evidente com os resultados é a discrepância na qualidade de solução
entre a ordenação crescente e a decrescente, algo já esperado.
Na \Cref{tab:ordenacao} é possível notar que ordenando de forma decrescente é possível ocupar
cerca de $20\%$ a mais do espaço (coluna “Qualidade \%”), quando comparado a ordenação crescente.

\begin{table}[!htb]
    \centering
    \caption{Resultado da comparação entre Descending.}
    \label{tab:descending}
    \IBGEtab{}{
        \begin{tabular}{lrrrrr}
            \hline
            Descending & Wons & Draws & Quality \% & Items \% & Time (s) \\
            \hline
            F          & 167  & 8     & 57.306     & 47.6518  & 2.37153  \\
            T          & 736  & 8     & 78.9136    & 46.3642  & 1.77985  \\
            \hline
        \end{tabular}
    }{}
    \fonte{autor}
\end{table}

A coluna “Vitórias” indica quantas vezes tal método de solução obteve o melhor resultado em
comparação com os demais métodos em outras linhas.
Enquanto a coluna “Empates” mostra a quantidade de vezes que o método conseguiu a melhor qualidade,
mas outros também conseguiram.
Essas colunas foram feitas da seguinte forma: entre cada combinação de critério de ordenação,
modo de criar regiões e instâncias, é feita a comparação se a qualidade de solução foi melhor
para ordenação crescente ou decrescente.
No caso de ambas conseguirem o melhor resultado, é acrescido 1 tanto na coluna “Vitórias”, quanto
na “Empates” de ambas.
Por fim, a coluna “Tempo (s)” mostra o tempo médio de execução do método em segundos.

Com isso, fica claro que ordenar a fila de entrada da \textit{bottom-left} de modo decrescente é
vantajoso em termos de qualidade, quantidade e tempo de execução.
Isso se deve a como as regiões são criadas,
as \Crefrange{fig:estado-inicial}{fig:estado-final} serão utilizadas para exemplificar,
elas representam estados do modelo usando ordenação crescente pela altura e linhas horizontais
para criação de regiões para a instância BKW01.
A \Cref{fig:estado-inicial} representa o estado inicial do algoritmo de solução, onde ainda
não existem peças alocadas.
% TODO: talvez separar as justificativas para qualidade e quantidade e falar sobre o tempo separado.

\begin{figure}[H]
    \centering
    \includegraphics[scale=0.5]{output/figures/bkw/bkw01/horizontally/height/false/00}
    \caption{Regiões criadas na ordenação crescente - estado inicial.}
    \label{fig:estado-inicial}
\end{figure}


Ao posicionar uma peça, uma das regiões ficará com a mesma altura do item recém-posicionado
(\Cref{fig:estado-1}), como a ordenação é crescente a próxima peça terá no mínimo a mesma altura,
mas o provável é que seja mais alta, impossibilitando seu posicionamento nessa região.

\begin{figure}[H]
    \centering
    \caption{Regiões criadas na ordenação crescente - estado 1.}
    \includegraphics[scale=0.5]{output/figures/bkw/bkw01/horizontally/height/false/01}
    \label{fig:estado-1}
    \fonte{feito pelo autor.}
\end{figure}


O mesmo irá ocorrer para todos os itens seguintes (\Cref{fig:estado-2}), fazendo com que muitas
regiões fiquem sem poder receber peças.

\begin{figure}
    \centering
    \includegraphics[scale=0.5]{output/figures/bkw/bkw01/horizontally/height/false/02}
    \caption{Regiões criadas na ordenação crescente - estado 2.}
    \label{fig:estado-2}
\end{figure}


A \Cref{fig:estado-final} mostra o estado final do algoritmo de solução e grande parte do espaço
ainda está livre.
Na \Cref{tab:bkw01} do \Cref{ch:resultados-das-instancias} é possível ver que a qualidade de
solução do modelo com essas combinações foi de $12.75\%$, enquanto ao utilizar ordenação decrescente
foi possível encontrar uma solução de $100\%$.
Algo semelhante ocorre com outros critérios de ordenação e criação de regiões.

\begin{figure}
    \centering
    \includegraphics[scale=0.5]{output/figures/bkw/bkw01/horizontally/height/false/06}
    \caption{Regiões criadas na ordenação crescente - estado final.}
    \label{fig:estado-final}
\end{figure}

% TODO: adicionar tabela bkw01 aqui
