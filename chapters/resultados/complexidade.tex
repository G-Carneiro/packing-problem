\section{Complexidade}\label{sec:complexidade}

As \cref{sec:comparativo-entre-criacao-de-regioes,sec:comparativo-entre-combinacoes} deixaram claro
que regiões onde a sobreposição pode ocorrer são mais custosas no tempo de execução.
Isso acontece devido à complexidade do algoritmo de solução do modelo.

Em regiões simples, onde não podem acontecer sobreposições, é necessário verificar se o item a ser
alocado cabe em uma região.
No pior caso todas as peças só conseguiriam ser alocadas na última região checada.
É possível descobrir o número máximo de regiões disponíveis no momento de alocação do item $i$ da
fila.

Para o primeiro item, existe somente uma região, o próprio espaço.
Após alocar o primeiro item, tem-se duas regiões disponíveis para posicionar o segundo.
Ao posicionar uma peça o número de regiões é acrescido em um.
Com isso, é possível afirmar que para alocar o item $i$ será preciso verificar no máximo $i$ regiões.
Somando o máximo de regiões para cada item chega-se no número máximo de regiões a serem verificadas
no pior caso.
Considerando $n$ itens, tem-se a \Cref{eq:sum-1-to-n}.

\begin{equation}
    \label{eq:sum-1-to-n}
    \sum_{i=1}^{n} i
\end{equation}


Como a soma de 1 até $n$ pode ser reescrita utilizando a \Cref{eq:formula-sum-1-to-n}
~\cite{merca2015alternative}, tem-se que o número máximo de regiões a serem verificadas
para solucionar o modelo é $\frac{n^2 + n}{2}$.

\begin{equation}
    \label{eq:formula-sum-1-to-n}
    \sum_{k=1}^{n} k = \dfrac{n(n+1)}{2}
\end{equation}


Para regiões complexas, além desse número de regiões, ainda é necessário verificar sobreposições.
No pior caso, ao alocar o item $i$, para cada região será preciso verificar sobreposições com
todos os itens $i - 1$ (\Cref{eq:sum-i-pow-2}).

\begin{equation}
    \label{eq:sum-i-pow-2}
    \sum_{i=1}^{n} i (i - 1)
\end{equation}


Usando a Fórmula de Faulhaber~\cite{merca2015alternative} é possível reescrever a soma dos quadrados
dos $n$ primeiros números inteiros como mostra a \Cref{eq:sum-k-pow-2}.

\begin{equation}
    \label{eq:sum-k-pow-2}
    \sum_{k=1}^{n} k^2 = \dfrac{n(n+1)(2n+1)}{6}
\end{equation}


Assim, ao reescrever a \Cref{eq:sum-i-pow-2} utilizando as \Cref{eq:formula-sum-1-to-n,eq:sum-k-pow-2},
tem-se que, no pior caso, será necessário verificar se existe sobreposição entre peças
$\frac{n^3-n}{3}$ vezes.
Para $n = 3152$, seriam quase cinco milhões de regiões e mais 10 bilhões de sobreposições a serem
checadas, explicando o tempo elevado ao executar um método de solução utilizando regiões complexas.