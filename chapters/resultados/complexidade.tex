\section{Complexidade}\label{sec:complexidade}

As \cref{sec:comparativo-entre-criacao-de-regioes,sec:comparativo-entre-combinacoes} deixaram claro
que regiões onde a sobreposição pode ocorrer são mais custosas no tempo de execução.
Isso acontece devido à complexidade do algoritmo de solução do modelo.

Em regiões simples, onde não podem acontecer sobreposições, é necessário verificar se o item a ser
alocado cabe em uma região.
O \Cref{alg:simple-region} é um pseudocódigo de solução para esse caso (código fonte feito em
Python pode ser visto no \Cref{ch:codigo-fonte}).
Onde $\mathcal{I}$ é o conjunto de itens a serem alocados e $R$ são as regiões disponíveis no
momento (ordenadas para respeitar a \textit{bottom-left}).
A função \texttt{CriarNovasRegiões()} elimina $r$, cria duas novas regiões com base no
posicionamento de $i$ e atualiza $R$ com as novas regiões.
A linha 3 verifica se um item $i$ cabe em uma região $r$, se for possível, o item é alocado e a
função \texttt{CriarNovasRegiões()} é chamada.
Quando um item é alocado, o conjunto $\mathcal{I'}$ (conjunto de itens alocados) é atualizado.
Após, um \texttt{\textbf{break}} é acionado para avançar ao próximo item.
No pior caso, todas as peças só conseguiriam ser alocadas na última região checada.
É possível descobrir o número máximo de regiões disponíveis no momento de alocação do item $i$ da
fila.

\begin{algorithm}
    \caption{Solução usando regiões simples.}
    \label{alg:simple-region}
    \SetKw{AND}{\textbf{and}}
    \SetKw{Break}{\textbf{break}}
    \SetKwFunction{Regioes}{CriarNovasRegioes}
    \ForAll{$i \in \mathcal{I}$}{
        \ForAll{$r  \in R$}{
            \If{$i.largura \leq r.largura$ \AND $i.altura \leq r.altura$}{
                aloca item $i$ na região $r$\;
                \tcc{elimina $r$ e cria duas novas regiões com base em $i$}
                \Regioes{i, r}\;
                \Break\;
            }
        }
    }
\end{algorithm}


Para o primeiro item, existe somente uma região, o próprio espaço.
Após alocar o primeiro item, tem-se duas regiões disponíveis para posicionar o segundo (ver
\cref{sec:ordenacao-crescente-decrescente}).
Ao posicionar uma peça o número de regiões sempre é acrescido em um (uma região é eliminada e duas
são criadas).
Com isso, é possível afirmar que para alocar o item $i$ será preciso verificar no máximo $i$ regiões.
Somando o máximo de regiões para cada item chega-se no número máximo de regiões a serem verificadas
no pior caso.
Considerando $n$ itens, tem-se a \Cref{eq:sum-1-to-n}.

\begin{equation}
    \label{eq:sum-1-to-n}
    \sum_{i=1}^{n} i
\end{equation}


Como a soma de 1 até $n$ pode ser reescrita utilizando a \Cref{eq:formula-sum-1-to-n}
~\cite{merca2015alternative}, tem-se que o número máximo de regiões a serem verificadas
para solucionar o modelo é $\frac{n^2 + n}{2}$.

\begin{equation}
    \label{eq:formula-sum-1-to-n}
    \sum_{k=1}^{n} k = \dfrac{n(n+1)}{2}
\end{equation}


Para regiões complexas, além desse número de regiões, ainda é necessário verificar sobreposições,
conforme o \Cref{alg:complex-region}.
O conjunto $\mathcal{I'}$ se refere aos itens já alocados no recipiente e a função
\texttt{Sobreposicao} verifica se existe sobreposição entre duas peças.
Agora, para cada região $r \in R$ onde é possível alocar o item $i \in \mathcal{I}$, também é
preciso verificar todos itens $i' \in \mathcal{I'}$.
A linha 6 garante que $i$ e $i'$ são diferentes e, caso haja sobreposição, o item $i$ é removido
da região $r$ (foi alocado na linha 4) e um \texttt{\textbf{break}} é acionado, pois não existe
necessidade de verificar outros itens $i' \in \mathcal{I'}$.
Na linha 11 é checado se as coordenadas do item $i$ existem, ou seja, se o item ainda está
posicionado (não houve sobreposição), então a função \texttt{CriarNovasRegiões} é chamada.

\begin{algorithm}
    \caption{Solução usando regiões complexas.}
    \label{alg:complex-region}
    \SetKw{AND}{\textbf{and}}
    \SetKw{Break}{\textbf{break}}
    \SetKwFunction{Sobreposicao}{Sobreposicao}
    \SetKwFunction{Regioes}{CriarNovasRegioes}
    \ForAll{$i \in \mathcal{I}$}{
        \ForAll{$r \in R$}{
            \If{$i.largura \leq r.largura$ \AND $i.altura \leq r.altura$}{
                aloca item $i$ na região $r$\;
                \ForAll{$i' \in \mathcal{I'}$}{
                    \If{$i \neq i'$ \AND \Sobreposicao{i, i'}}{
                        remove item $i$ na região $r$\;
                        \Break\;
                    }
                }
                \If{$i.coordenadas \neq null$}{
                    \tcc{elimina $r$ e cria duas novas regiões com base em $i$}
                    \Regioes{i, r}\;
                    \Break\;
                }
            }
        }
    }
\end{algorithm}


No pior caso, ao alocar o item $i$, para cada região será preciso verificar sobreposições com
todos os itens $i - 1$ (\Cref{eq:sum-i-pow-2}).

\begin{equation}
    \label{eq:sum-i-pow-2}
    \sum_{i=1}^{n} i (i - 1)
\end{equation}


Usando a Fórmula de Faulhaber~\cite{merca2015alternative} é possível reescrever a soma dos quadrados
dos $n$ primeiros números inteiros como mostra a \Cref{eq:sum-k-pow-2}.

\begin{equation}
    \label{eq:sum-k-pow-2}
    \sum_{k=1}^{n} k^2 = \dfrac{n(n+1)(2n+1)}{6}
\end{equation}


Assim, ao reescrever a \Cref{eq:sum-i-pow-2} utilizando as \Cref{eq:formula-sum-1-to-n,eq:sum-k-pow-2},
tem-se que, no pior caso, será necessário verificar se existe sobreposição entre peças
$\frac{n^3-n}{3}$ vezes.
Para $n = 3152$, seriam quase cinco milhões de regiões e mais 10 bilhões de sobreposições a serem
checadas, explicando o tempo elevado ao executar um método de solução utilizando regiões complexas.