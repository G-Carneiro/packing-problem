\section{Comparativo entre combinações}\label{sec:comparativo-entre-combinacoes}

A \Cref{tab:combinations} contém os resultados para cada um dos quarenta métodos de solução feitos.
Com ela é possível identificar qual dos métodos apresenta melhores resultados qualitativos e
quantitativos.

\begin{table}[!htb]
    \centering
    \caption{Resultado da comparação entre ['Split', 'Order', 'Descending', 'Wons', 'Draws', 'Quality \%', 'Items \%', 'Time (s)'].}
    \label{tab:combinations}
    \IBGEtab{}{
        \ttfamily\begin{tabular}{lllrrrrr}
    \hline
    Split & Order & Descending & Wons & Draws & Quality \% & Items \% & Time (s)   \\
    \hline
    V     & A     & F          & 0    & 0     & 50.9443    & 47.8026  & 0.00248052 \\
    V     & A     & T          & 6    & 6     & 78.9961    & 43.2429  & 0.00308339 \\
    V     & P     & F          & 0    & 0     & 51.2033    & 46.2057  & 0.00214879 \\
    V     & P     & T          & 7    & 6     & 82.621     & 45.6727  & 0.00232849 \\
    V     & H     & F          & 0    & 0     & 55.2624    & 48.4624  & 0.00201776 \\
    V     & H     & T          & 4    & 4     & 70.7811    & 42.5165  & 0.00253342 \\
    V     & W     & F          & 0    & 0     & 47.9606    & 42.6878  & 0.00166203 \\
    V     & W     & T          & 9    & 8     & 84.5497    & 47.058   & 0.002482   \\
    V     & I     & F          & 1    & 1     & 62.6394    & 46.0333  & 0.00242357 \\
    V     & I     & T          & 1    & 1     & 65.067     & 46.6058  & 0.00315097 \\
    H     & A     & F          & 1    & 1     & 44.9575    & 42.0545  & 0.0088805  \\
    H     & A     & T          & 5    & 5     & 81.5022    & 43.6548  & 0.00599634 \\
    H     & P     & F          & 0    & 0     & 45.0368    & 41.4449  & 0.00852498 \\
    H     & P     & T          & 4    & 3     & 82.639     & 43.4046  & 0.00475726 \\
    H     & H     & F          & 2    & 2     & 43.4125    & 35.4793  & 0.00770928 \\
    H     & H     & T          & 4    & 4     & 79.2274    & 47.835   & 0.00614419 \\
    H     & W     & F          & 0    & 0     & 51.7897    & 47.2147  & 0.0100625  \\
    H     & W     & T          & 4    & 4     & 74.9317    & 45.5948  & 0.00761566 \\
    H     & I     & F          & 2    & 2     & 64.9816    & 47.1122  & 0.00609561 \\
    H     & I     & T          & 1    & 1     & 61.6848    & 47.23    & 0.006537   \\
    M     & A     & F          & 2    & 2     & 53.1636    & 50.7883  & 0.0172841  \\
    M     & A     & T          & 7    & 7     & 83.2483    & 45.6017  & 0.0132333  \\
    M     & P     & F          & 1    & 1     & 53.7023    & 49.7762  & 0.0176751  \\
    M     & P     & T          & 7    & 6     & 85.8682    & 46.3078  & 0.012944   \\
    M     & H     & F          & 2    & 2     & 54.8203    & 45.9835  & 0.0108422  \\
    M     & H     & T          & 4    & 4     & 78.5353    & 47.1767  & 0.0132691  \\
    M     & W     & F          & 0    & 0     & 62.4641    & 51.3552  & 0.00401001 \\
    M     & W     & T          & 5    & 5     & 79.457     & 48.8029  & 0.0148469  \\
    M     & I     & F          & 2    & 2     & 72.6713    & 50.9134  & 0.0146069  \\
    M     & I     & T          & 2    & 2     & 71.4787    & 50.5082  & 0.0144215  \\
    N     & A     & F          & 0    & 0     & 61.151     & 51.4896  & 10.1459    \\
    N     & A     & T          & 13   & 11    & 85.3558    & 42.9123  & 6.29188    \\
    N     & P     & F          & 0    & 0     & 61.975     & 51.4671  & 9.90181    \\
    N     & P     & T          & 9    & 6     & 85.784     & 42.8241  & 6.25238    \\
    N     & H     & F          & 2    & 2     & 63.498     & 49.9569  & 8.27386    \\
    N     & H     & T          & 6    & 5     & 79.4085    & 46.6447  & 6.20546    \\
    N     & W     & F          & 0    & 0     & 64.7865    & 51.0338  & 10.3591    \\
    N     & W     & T          & 16   & 10    & 84.0171    & 48.1854  & 8.25089    \\
    N     & I     & F          & 4    & 4     & 73.3323    & 50.4803  & 8.37008    \\
    N     & I     & T          & 5    & 3     & 74.351     & 50.3528  & 8.28526    \\
    \hline
\end{tabular}
    }{}
    \fonte{autor}
\end{table}

Dentre os métodos que usam regiões onde é preciso checar sobreposições, os mais interessantes são
o de ordenação decrescente pela área (linha 30), pelo perímetro (linha 32) e pela largura (linha 36).
Utilizando a área conseguiu-se o segundo melhor resultado quantitativo e também qualitativo.
Com o perímetro foi possível atingir o terceiro maior número de vitórias e a melhor qualidade de
solução.
Por fim, a largura obteve o melhor resultado quantitativo e ficou em terceiro na qualidade de solução.

Nos métodos que usam regiões mais simples, os resultados foram bem variados nas combinações,
dentre eles se destacam: regiões criadas usando linha vertical e ordenação pela largura (linha 6) e
maximizar uma região e ordenar pela área (linha 20) ou pelo perímetro (linha 22).
O método da linha 6 ficou em terceiro lugar no critério quantitativo (empate com a linha 32) e
conseguiu a quinta posição na qualidade de solução.
As linhas 20 e 22 tiveram a quarta maior quantidade de vitórias (empate também com a linha 2), já
qualitativamente a linha 20 obteve a sexta melhor média nas soluções, enquanto a linha 22 conseguiu
a quarta.

Ainda que os resultados dos métodos que usam regiões as quais calculam sobreposições sejam melhores,
não compensa utilizá-los, como visto na \cref{sec:comparativo-entre-criacao-de-regioes}.