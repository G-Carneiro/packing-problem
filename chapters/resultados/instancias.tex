\section{Conjuntos de instâncias}\label{sec:conjuntos-de-instancias}

A \Cref{tab:instancias} considera os melhores resultados para cada conjunto de instância.
As colunas “Qualidade \%” e “Itens \%” representam, respectivamente, a média percentual da qualidade
de solução e itens alocados no melhor resultado de cada instância, considerando apenas os
métodos que utilizam regiões simples.
Já a coluna “Tempo Total (s)” traz o tempo total, em segundos, de execução para todos os métodos
em todas as instâncias do conjunto.

\begin{table}[!htb]
    \centering
    \caption{Resultado da comparação entre InstanceSet.}
    \label{tab:combinations}
    \IBGEtab{}{
        \ttfamily\begin{tabular}{lrrr}
    \hline
    InstanceSet & Quality \% & Items \% & Time (s)    \\
    \hline
    BKW         & 62.9619    & 84.1578  & 7.28255     \\
    GCUT        & 64.3739    & 19.152   & 0.000270303 \\
    NGCUT       & 71.7568    & 47.3996  & 0.00038499  \\
    OF          & 72.1888    & 30.7337  & 0.000756912 \\
    OKP         & 77.1125    & 27.6778  & 0.00360035  \\
    \hline
\end{tabular}
    }{}
    \fonte{autor}
\end{table}

É notável que ótimos resultados foram obtidos em um baixo período.
No conjunto BKW, o qual é possível alocar todos os itens, levou mais tempo que os demais, pois
possuí a maior quantidade de instâncias e algumas com várias peças.
Os conjuntos OF e OKP possuem poucas instâncias e, por isso, tiveram pouco tempo de execução.
De forma geral, todos os conjuntos conseguiram uma excelente qualidade média, sendo o GCUT e NGCUT
os únicos a ficarem abaixo de $90\%$.

% TODO: adicionar tabela comparando com outros autores.