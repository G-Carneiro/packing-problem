\section{Comparativo entre critérios de ordenação}\label{sec:comparativo-entre-criterios-de-ordenacao}

Ao comparar os diferentes critérios de ordenação (\Cref{tab:ordenacoes}), inicialmente tem-se
um resultado curioso, a ordenação por \textit{id} (nenhuma ordenação) está com os melhores
resultados qualitativos.
Além disso, todos os critérios conseguiram praticamente a mesma qualidade média.

\begin{table}[!htb]
    \centering
    \caption{Resultado da comparação entre critérios de ordenação.}
    \label{tab:ordenacoes}
    \IBGEtab{}{
        \ttfamily\begin{tabular}{lrrrr}
 \hline
 Ordenação & Vitórias & Empates & Qualidade \% & Tempo (s)  \\
 \hline
 Área      & 81       & 56      & 67.7915      & 2.0726e+00 \\
 Perímetro & 93       & 52      & 68.9870      & 2.0366e+00 \\
 Altura    & 73       & 24      & 65.9848      & 1.8254e+00 \\
 Largura   & 94       & 32      & 69.1286      & 2.3444e+00 \\
 Id        & 127      & 12      & 68.6572      & 2.0995e+00 \\
 \hline
\end{tabular}
    }{}
    \fonte{feito pelo autor.}
\end{table}

Como mencionado anteriormente (\cref{sec:ordenacao-crescente-decrescente}), ordenar a fila de
forma crescente gera resultados ruins.
Mas isso não ocorre quando o critério de ordenação é o \textit{id}, porque nele tem-se peças
sem seguirem alguma ordem.
Com esse critério o impacto da escolha entre ordenação crescente ou decrescente é praticamente
nulo.
Por isso ele também se saí muito melhor que os demais na ordenação crescente, sendo o vitorioso
na esmagadora maioria.
Já as qualidades próximas se justificam nos critérios diferentes do \textit{id} pois a
discrepância entre ordenação crescente e decrescente é muito grande, mas ao tirar média isso acaba
não ficando evidente.

A \Cref{tab:ordenacoes-true} mostra o mesmo comparativo que a \Cref{tab:ordenacoes}, porém agora
somente considerando a ordenação decrescente, já que não existem motivos para usar a crescente.
Representando os dados dessa forma fica fácil identificar que utilizar algum critério de ordenação
para a fila de entrada é vantajoso, pois ao usar o \textit{id} os resultados foram os piores.
Além disso, percebe-se que as ordenações por área e perímetro obtiveram os melhores resultados,
ainda que os demais também sejam competitivos.

\begin{table}[!htb]
    \centering
    \caption{Resultado da comparação entre critérios de ordenação decrescente.}
    \label{tab:ordenacoes-true}
    \IBGEtab{}{
        \ttfamily\begin{tabular}{lrrrr}
    \hline
    Ordenação & Vitórias & Empates & Qualidade \% & Tempo (s)  \\
    \hline
    Área      & 63       & 39      & 82.7353      & 1.5874e+00 \\
    Perímetro & 71       & 38      & 84.6986      & 1.5769e+00 \\
    Altura    & 40       & 16      & 77.4182      & 1.5655e+00 \\
    Largura   & 66       & 24      & 81.1899      & 2.0805e+00 \\
    Id        & 16       & 5       & 68.5261      & 2.0889e+00 \\
    \hline
\end{tabular}
    }{}
    \fonte{feito pelo autor.}
\end{table}

% TODO: perguntar ao Pedro oq acha do parágrafo
A alta competitividade entre os critérios de ordenação é interessante, pois a maioria dos trabalhos
na literatura, como o de \citeauthor*{chen2019efficient}~\citeyear{chen2019efficient}, usam somente
ordenação pela área, e isso pode ser um forte indicativo que os demais critérios devem ser mais
explorados em certas circunstâncias.
Ao considerar somente a ordenação decrescente deixa claro que a alta quantidade vitórias do critério
\textit{id} na \Cref{tab:ordenacoes} realmente se deve a ordenação crescente.
