\chapter*[Conclusão e trabalhos futuros]{Conclusão e trabalhos futuros}\label{ch:conclusao}
\addcontentsline{toc}{chapter}{\texorpdfstring{CONCLUSÃO E TRABALHOS FUTUROS}{Conclusão e trabalhos futuros}}

No trabalho foram avaliadas experimentalmente 40 métodos de solução baseados na heurística
\textit{bottom-left} para o problema de empacotamento de retângulos, considerando a versão da
mochila e com o valor dos itens sua própria área.
O resultado mais óbvio obtido foi a supremacia da ordenação decrescente sobre a crescente
(\cref{sec:ordenacao-crescente-decrescente}), não compensando utilizar a segunda para solucionar o
problema.

O alto uso da ordenação decrescente pela área na literatura~\cite{chen2019efficient} foi justificado
pelos resultados, já que essa composição conseguiu alguns dos melhores números
(\cref{sec:comparativo-entre-criterios-de-ordenacao}).
Os resultados também indicam que qualquer critério de ordenação obtêm melhores resultados do que
deixar a fila desordenada (ordenação por \textit{id}).
Mas a alta competitividade de outros critérios de ordenação, esses não tão comuns na literatura,
conseguindo melhores resultados em algumas instâncias, mostra que ainda há espaço para outras
formas de ordenação.

Em relação às diferentes regiões criadas, por mais que regiões complexas tenham obtido melhores
resultados (\cref{sec:comparativo-entre-criacao-de-regioes}), não compensa utilizá-las.
Ao executar todos os métodos de regiões simples é possível conseguir melhores resultados e em menor
tempo, quando comparado ao método de ordenação decrescente pela área e usando regiões complexas
(ver a \Cref{tab:superposition} na \cref{sec:comparativo-entre-criacao-de-regioes}).
Isso devido à complexidade do algoritmo que precisa verificar a existência de sobreposições de peças
em regiões complexas (\cref{sec:complexidade}).

Como trabalho futuro, cabe a investigação, a fundo, de abordagens meta-heurísticas para a
\textit{bottom-left}, considerando os outros critérios promissores que não seja a área.
Buscando identificar se há alguma característica na instância que proporcione melhores soluções com
outros critérios, a suspeita inicial é que esteja relacionado a quão quadrados são os itens a serem
alocados no recipiente.

Outros trabalhos possíveis são: expandir o algoritmo de solução para o caso 3D (ou $N$-dimensões) e
possibilitar a resolução de outras classes do problema (2D-SPP, 2D-CSP, 2D-OPP) e suas variantes.
Também é possível implementar métodos de solução da literatura diferentes da \textit{bottom-left},
sejam eles exatos ou heurísticos.
Com isso, seria possível criar uma ampla biblioteca em Python para problemas de corte e empacotamento.

Por fim, seria interessante buscar soluções quânticas para o problema.
No momento, existem poucos trabalhos disponíveis na literatura que abordem o tema.
Um dos mais recentes, feito por \citeauthoryear{de2022hybrid}, traz uma heurística híbrida
clássica-quântica para o \textit{bin packing problem} (BPP) unidimensional, conseguindo com essa
nova heurística bons resultados para o problema.
Outro resultado interessante foi o de \citeauthoryear{fujimura2021quantum}, com um algoritmo quântico
foi possível chegar a uma complexidade $3(\log_2 k)n$ para o BPP, onde $k$ representa o número de
caixas e $n$ o número de peças.
Trabalhos como esses ainda são raros, principalmente para dimensões maiores, mas os poucos que
existem indicam bons resultados para soluções quânticas, ou híbridas, do problema de empacotamento.
