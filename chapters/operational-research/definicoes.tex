\section{Definições}\label{sec:definicoes}

A seguir serão dadas as definições de quatro expressões que aparecem com frequência no estudo de problemas de otimização.

Uma solução $x'$ é \textbf{factível} somente se satisfaz todas as restrições dadas ao problema, ou seja, $x' \in \mathcal{X}$.
Existem casos onde o problema não tem solução, possivelmente por muitas restrições terem sido aplicadas.
Isso é chamado \textbf{problema infactível} e $\mathcal{X} = \emptyset$.
Se para toda solução for possível encontrar outra melhor o problema é dito \textbf{ilimitado}.

Uma solução $x'$ é \textbf{ótima} somente se for \textbf{factível} e possuir resultado melhor que as demais soluções, isto é, $f(x') \le f(x), \forall x \in \mathcal{X}$ (caso seja um problema de maximização é necessário substituir “$\le$” por “$\ge$”).
Importante observar que existe somente solução ótima se o problema não for infactível nem ilimitado.
