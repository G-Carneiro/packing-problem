\chapter{Conceitos Básicos}\label{ch:conceitos-basicos}

Antes de estudar o problema, são necessários alguns conceitos básicos e definição formal de termos importantes para a área de Pesquisa Operacional e otimização.
Pesquisa Operacional pode ser entendida como o estudo e a aplicação de métodos científicos para tomada de decisões em problemas complexos \cite[p.IX]{arenales}.
Ela permite modelar, analisar e solucionar tais problemas de modo, geralmente, satisfatório.

Neste capítulo será mostrado o que são modelos de otimização (\autoref{sec:modelos-de-otimizacao}) e seus tipos (\autoref{sec:tipos-de-modelo}), além de algumas definições sobre otimização (\autoref{sec:definicoes}).


\section{Modelos de Otimização}\label{sec:modelos-de-otimizacao}

Modelos são aproximações da realidade, representam o problema de maneira simples e objetiva, usando restrições.
Eles são o que baseiam a Pesquisa Operacional.
De forma geral, um modelo de otimização quer minimizar ou maximizar uma função $f(x)$ com $x$ obedecendo algumas restrições.
Pode-se então representar o modelo do seguinte modo:

\[
    \min\!/\!\max f(x), x \in \mathcal{X}.
\]

Onde

\begin{itemize}
    \item $x$: variável de decisão, $x = x_1, x_2, \dots, x_n$.
    \item $\mathcal{X}$: conjunto factível ou domínio, possui todas as soluções possíveis para o problema.
    \item $f(x)$: função objetivo, a qual determinará o critério de escolha da solução.
\end{itemize}


\section{Definições}\label{sec:definicoes}

A seguir serão dadas as definições de quatro expressões que aparecem com frequência no estudo de problemas de otimização.

Uma solução $x'$ é \textbf{factível} somente se satisfaz todas as restrições dados ao problema, ou seja, $x' \in \mathcal{X}$.
Existem casos onde o problema não tem solução, possivelmente por muitas restrições terem sido aplicadas.
Isso é chamado \textbf{problema infactível} e $\mathcal{X} = \emptyset$.
Se para toda solução for possível encontrar outra melhor o problema é dito \textbf{ilimitado}.

Uma solução $x'$ é \textbf{ótima} somente se for \textbf{factível} e possuir resultado melhor que as demais soluções, isto é, $f(x') \le f(x), \forall x \in \mathcal{X}$ (caso seja um problema de maximização é necessário substituir “$\le$” por “$\ge$”).
Importante observar que existe somente solução ótima se o problema não for infactível nem ilimitado.


\section{Tipos de Modelo}\label{sec:tipos-de-modelo}
% TODO: adicionar figuras

É importante saber diferenciar os modelos devido ao método de resolução que varia para cada um deles.

\subsection{Modelo Linear x Não-linear}\label{subsec:modelo-linear}


Modelos lineares possuem como função objetivo uma função linear e todas as restrições também são lineares.
Exemplos:

\begin{itemize}
    \item $f(x) = ax + b$.
    \item $f(x_1, x_2) = x_1 + x_2 - 5$.
\end{itemize}

Já os não-lineares não obedecem essa regra, podendo ter suas variáveis se multiplicando ou funções trigonométricas e logarítmicas.
Exemplos:

\begin{itemize}
    \item $f(x_1, x_2) = x_1^2 + x_2^2$.
    \item $f(x_1, x_2) = \tan(x_1 + x_2)$.
\end{itemize}

\subsection{Modelo Contínuo x Discreto}\label{subsec:modelo-continuo-x-discreto}

Um modelo é contínuo quando sua região factível é contínua, ou seja, dado um ponto dessa região todos os seus vizinhos também serão uma solução.
Modelos discretos não possuem seu domínio contínuo.

\subsection{Modelo Determinístico x Estocástico}\label{subsec:modelo-deterministico-x-estocastico}

Em modelos determinísticos seus dados são conhecidos, enquanto os estocásticos possuem uma incerteza quanto aos dados.

\subsection{Tipos de Programação}\label{subsec:tipos-de-programacao}

Com base nas categorias de modelo é possível também dividir métodos de programação (planejamento) para sua solução.

\begin{itemize}
    \item Linear: modelo linear contínuo determinístico.
    \item Inteira: modelo linear discreto determinístico.
    \item Estocástica: modelo linear contínuo estocástico.
    \item Não-linear: modelo não-linear contínuo determinístico.
\end{itemize}


\section{Métodos Exatos x Heurísticos}\label{sec:metodos-exatos-heuristicos}
% TODO: falar mais sobre heurísticas e como escapar de ótimos locais.

Métodos exatos sempre vão garantir a solução ótima para o problema, porém encontrar tal solução pode requerer grande tempo e/ou muitos recursos computacionais.
Já heurísticas buscam por soluções factíveis e são geralmente usadas em problemas de grande porte.

Como o problema de interesse é NP-difícil e o principal interesse é em instâncias de médio e grande porte, utilizar um método exato seria bastante desafiador e, provavelmente, não seria possível obter um resultado em tempo hábil devido aos recursos computacionais atuais.
Portanto, métodos heurísticos serão usados, já que eles tendem a diminuir a demanda computacional, porém não garantem otimalidade da solução resultante.

Heurísticas geralmente convergem para ótimos locais, por isso geralmente mecanismos de fuga são usados para se escapar dessa região e tentar atingir um resultado melhor.
Alguns exemplos desses mecanismos são o \textit{multi-start} e o \textit{simulated annealing}.
