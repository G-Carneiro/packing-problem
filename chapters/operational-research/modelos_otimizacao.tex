\section{Modelos de otimização e definições}\label{sec:modelos-de-otimizacao}
% FIXME: adicionar referência

Modelos de otimização são aproximações da realidade, representam o problema de maneira simples
e objetiva, usando restrições.
De forma geral, um modelo de otimização busca minimizar ou maximizar uma função $f(x)$ com $x$
obedecendo algumas restrições.
Pode-se então representar o modelo do seguinte modo:

\[
    \min\!/\!\max f(x), x \in \mathcal{X}.
\]


Onde

\begin{itemize}
    \item $x$: variáveis de decisão, $x = (x_1, x_2, \dots, x_n)$.
    \item $\mathcal{X}$: conjunto factível ou domínio, possui todas as soluções viáveis para o problema.
    \item $f(x)$: função objetivo, a qual determinará o critério de escolha da solução.
\end{itemize}

A seguir serão dadas as definições de quatro expressões que aparecem com frequência no estudo de
problemas de otimização.

Uma solução $x'$ é \textbf{factível} somente se satisfaz todas as restrições dadas ao problema,
ou seja, $x' \in \mathcal{X}$.
Existem casos onde o problema não tem solução, possivelmente por muitas restrições terem sido
aplicadas.
Isso é chamado \textbf{problema infactível} e $\mathcal{X} = \emptyset$.
Se para toda solução for possível encontrar outra melhor o problema é dito \textbf{ilimitado}.

Uma solução $x'$ é \textbf{ótima} somente se for \textbf{factível} e possuir resultado melhor,
ou igual, que as demais soluções, isto é, $f(x') \le f(x), \forall x \in \mathcal{X}$ (caso seja um
problema de maximização é necessário substituir “$\le$” por “$\ge$”).
Importante observar que existe somente solução ótima se o problema não for infactível nem ilimitado.

Usando o problema da mochila como exemplo, onde se busca maximizar o valor dos itens alocados
sem que seus pesos ultrapassem a capacidade da mochila~\cite{exact-solution-techniques},
um modelo possível é:

\[
    \max { \sum_{i \in I} v_{i} x_{i}: \sum_{i \in I} w_{i} x_{i} \leq C}
\]


Em que $I$ é um conjunto de itens, $v_{i}$ e $w_{i}$ são, respectivamente o valor e o peso do item
$i \in I$, $C$ é a capacidade da mochila e $x_{i}$ são variáveis binárias indicando se o
item foi escolhido para mochila ou não.

O problema não é ilimitado, já que a melhor solução (solução ótima) é ocupar toda a capacidade $C$
da mochila, caso seja possível com os itens $I$.
Qualquer solução que não ultrapasse o limite da mochila é factível.
O problema só será considerado infactível caso não seja possível alocar pelo menos um item, ou seja,
todos os itens possuírem peso $w_i > C$.
