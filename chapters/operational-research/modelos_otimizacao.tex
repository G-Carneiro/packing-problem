\section{Modelos de Otimização}\label{sec:modelos-de-otimizacao}
% FIXME: adicionar referência

Modelos são aproximações da realidade, representam o problema de maneira simples e objetiva,
usando restrições.
Eles são o que baseiam a Pesquisa Operacional.
De forma geral, um modelo de otimização quer minimizar ou maximizar uma função $f(x)$ com $x$
obedecendo algumas restrições.
Pode-se então representar o modelo do seguinte modo:

\[
    \min\!/\!\max f(x), x \in \mathcal{X}.
\]


Onde

\begin{itemize}
    \item $x$: variável de decisão, $x = x_1, x_2, \dots, x_n$.
    \item $\mathcal{X}$: conjunto factível ou domínio, possui todas as soluções possíveis para o problema.
    \item $f(x)$: função objetivo, a qual determinará o critério de escolha da solução.
\end{itemize}

Usando o problema da mochila como exemplo, onde se busca maximizar o valor dos itens alocados
sem que seus pesos ultrapassem a capacidade da mochila,
um modelo possível é:

% TODO: confirmar modelo
\[
    \max { \sum_{i \in I} v_{i} x_{i}: \sum_{i \in I} w_{i} x_{i} \leq C}
\]


Em que $I$ é um conjunto de itens, $v_{i}$ e $w_{i}$ são, respectivamente o valor e o peso do item
$i \in I$, $C$ é a capacidade da mochila e $x_{i}$ são variáveis binárias indicando se o
item foi escolhido para mochila ou não.
