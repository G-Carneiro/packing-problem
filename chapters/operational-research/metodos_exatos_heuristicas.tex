\section{Métodos Exatos × Heurísticos}\label{sec:metodos-exatos-heuristicos}
% TODO: falar mais sobre heurísticas e como escapar de ótimos locais.
% FIXME: referências

Métodos exatos sempre vão garantir a solução ótima para o problema, porém encontrar tal solução pode requerer grande tempo e/ou muitos recursos computacionais.
Já heurísticas buscam por soluções factíveis e são geralmente usadas em problemas de grande porte.

Como o problema de interesse é NP-difícil e o principal interesse é em instâncias de médio e grande porte, utilizar um método exato seria bastante desafiador e, provavelmente, não seria possível obter um resultado em tempo hábil devido aos recursos computacionais atuais.
Portanto, métodos heurísticos serão usados, já que eles tendem a diminuir a demanda computacional, porém não garantem otimalidade da solução resultante.

Heurísticas geralmente convergem para ótimos locais, por isso geralmente mecanismos de fuga são usados para se escapar dessa região e tentar atingir um resultado melhor.
Alguns exemplos desses mecanismos são o \textit{multi-start} e o \textit{simulated annealing}.
