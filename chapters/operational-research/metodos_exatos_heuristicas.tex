\section{Métodos Exatos × Heurísticos}\label{sec:metodos-exatos-heuristicos}
% TODO: falar mais sobre heurísticas e como escapar de ótimos locais.
% FIXME: referências

Métodos exatos sempre vão garantir a solução ótima para o problema, porém encontrar
tal solução pode requerer grande tempo e/ou muitos recursos computacionais.
Já heurísticas buscam por soluções factíveis e são geralmente usadas em problemas de grande porte.

Um dos métodos exatos mais conhecidos é o algoritmo \textit{branch-and-bound}, ele realiza a
enumeração implícita das soluções viáveis de um problema de programação linear inteira mista,
mantendo valores para os limitantes inferior e superior de um problema de otimização.
O algoritmo termina sua execução quando ambos os limitantes se igualam, garantindo a otimalidade
da solução.
Detalhes do algoritmo \textit{branch-and-bound} e outros para problemas de programação inteira,
como \textit{branch-and-cut} e planos de corte podem ser vistos em \citeauthoryear{wolsey2020integer}.

Como o problema de empacotamento de retângulos é NP-difícil e o principal interesse é em instâncias
de médio e grande porte, utilizar um método exato seria bastante desafiador e, provavelmente,
não seria possível obter um resultado em tempo hábil devido aos recursos computacionais disponíveis.
Portanto, métodos heurísticos serão usados, já que eles tendem a diminuir a demanda computacional,
porém não garantem otimalidade da solução resultante.

Soluções heurísticas tipicamente alternam entre explorar o espaço de busca de forma mais ampla
e se concentrar em uma vizinhança de uma solução viável já encontrada.
Por isso, em geral, uma heurística garante apenas a otimalidade local da solução.
Para escapar de ótimos locais e buscar atingir um resultado melhor, mecanismos de fuga são usados.
Alguns exemplos desses mecanismos são o \textit{multi-start} e o \textit{simulated annealing}~\cite{
    firat2020effective,rakotonirainy2020improved,hopper2001empirical}.

Continuando o exemplo do problema da mochila~(\cref{sec:modelos-de-otimizacao}), uma heurística
possível para sua solução é ordenar os itens de maneira decrescente de acordo com $\dfrac{v_i}{w_i}$
e colocá-los na mochila enquanto couber.

As heurísticas podem ser divididas entre heurísticas construtivas e heurísticas de melhoria.
Heurísticas construtivas, como diz o nome, constroem uma solução para o problema,
enquanto heurísticas de melhoria, partem de uma solução viável e realizam tentativas de
melhorar tal solução~\cite{michalewicz2013solve}.
A heurística \textit{bottom-left} (detalhada no \Cref{ch:bottom-left}), a qual será usada para
resolver o problema de empacotamento de retângulos~(\Cref{ch:problema-de-empacotamento}),
é uma heurística construtiva.