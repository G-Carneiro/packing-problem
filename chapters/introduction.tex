% --------------------------------------------
% Aqui você deve organizar as seções.
% --------------------------------------------
\chapter*[Introdução]{Introdução}\label{ch:introducao}
\addcontentsline{toc}{chapter}{\texorpdfstring{INTRODUÇÃO}{Introdução}}

Serviços de loja \textit{online} com entrega como Amazon e Mercado Livre estão tornando-se cada vez
mais presentes no dia a dia.
Para tornar as entregas mais rápidas é necessário fazer uma série de estudos de logística e
planejamento sobre como organizar os produtos nos estoques e nos veículos de entrega~\cite{
    silva2022integer,morabito1992abordagem}, muitas vezes sendo necessário considerar a ordem
em que eles precisarão ser retirados.
Além de tornar o processo mais rápido, a organização também pode permitir o melhor uso de espaços,
aumentando a quantidade máxima de itens ou evitando o desperdício dos espaços.

Ainda sobre evitar desperdícios, esse quesito é muito importante para as indústrias de papel,
móveis, têxtil e metal-mecânica~\cite{queiroz2022estudo,cavali2004problemas,belluzzo2005otimizacao}.
Todas essas áreas querem gerar o máximo de produtos com o mínimo de recursos materiais utilizados,
para evitar o descarte desnecessário do material e prejuízos financeiros.

Os problemas citados são considerados problemas de corte e empacotamento.
Problemas de corte envolvem cortar um objeto, como blocos de gesso, chapas de aço e barras de ferro,
em itens menores.
Enquanto problemas de empacotamento tratam sobre alocar um conjunto de itens $\mathcal{I}$ em um
recipiente $\mathcal{B}$.
Ambos são equivalentes entre si e é possível separá-los de acordo com sua dimensão.

Problemas unidimensionais podem ser associados ao corte de barras ou canos, para atender uma
demanda por peças de diferentes tamanhos, ou ao empilhamento de caixas em busca de uma altura mínima.
Outra aplicação viável do caso 1D é o famoso problema da mochila~\cite{exact-solution-techniques},
onde se deseja alocar itens de valor $v_i$ e peso $w_i$ em uma mochila com capacidade $C$,
de forma que se maximize a soma dos valores dos itens escolhidos para entrar na mochila.

As indústrias de móveis, tecido, couro e papel usam o caso 2D para minimizar o desperdício ao se
cortar suas peças~\cite{queiroz2022estudo, cavali2004problemas, belluzzo2005otimizacao},
enquanto os setores de entregas usam para organizar paletes~\cite{morabito1992abordagem}.
O caso 3D é facilmente associável ao carregamento de \textit{containers}~\cite{morabito1992abordagem},
onde objetos são geralmente caixas a serem alocadas em algum veículo, ou ao corte de blocos de
gesso, já citado.

Basicamente, o problema é aplicável em qualquer área que precise de organização ou logística,
bem como em situações que envolvam o corte de algum material.
Ao utilizar soluções para resolver problemas de corte e empacotamento, é possível reduzir o
desperdício de materiais e impacto ambiental, diminuir tempo de entregas e otimizar espaços
de estoque.

O caso 2D, dimensão de estudo deste trabalho, possui uma vasta literatura de métodos de solução.
As abordagens de solução se dividem entre exatas, que buscam a solução ótima do problema, e
heurísticas, as quais podem não encontrar uma solução ótima, mas conseguem uma solução aceitável em
tempo hábil.
Dentre os métodos de solução exatos, um que se destaca é o procedimento de busca em árvore~\cite{
    beasley1985exact}, mas existem muitos outros na literatura~\cite{exact-solution-techniques,
    fekete1997new,delorme2016bin,kenmochi2009exact}.
Na parte de heurísticas, tem-se a \textit{bottom-left}~\cite{baker1980orthogonal,chehrazad2022fast}
e \textit{skyline}~\cite{wei2011skyline}, as heurísticas também possuem grande presença na
literatura~\cite{burke2004new,rakotonirainy2020improved,hopper2001empirical,chen2019efficient,
    huang2007efficient,hopper2001review}.

Este trabalho visa estudar o problema de empacotamento no espaço de duas dimensões, onde as peças
são retangulares e com um recipiente também retangular, mais especificamente na versão do
empacotamento 2D da mochila, considerado NP-difícil~\cite{2DPackLib}.
Nessa versão do problema, dado um conjunto de itens $\mathcal{I}$, com cada item $i$ possuindo um
valor $p_i$, e uma caixa $\mathcal{B}$, o objetivo é maximizar a soma dos valores dos itens alocados
dentro do recipiente.
A abordagem escolhida para resolver o problema foi a utilizar a heurística \textit{bottom-left},
devido a sua simplicidade e aos limites computacionais e de tempo ao escolher algum método exato.
Mesmo com a heurística sendo proposta em~\citeyear*{baker1980orthogonal}, ela ainda está presente
na literatura recente~\cite{chehrazad2022fast,hopper2001empirical,wei2011skyline}.

Empacotamento de retângulos tem importância prática~\cite{firat2020effective} e o fato de somente
eles serem considerados não é um grande demérito, já que é possível transformar qualquer polígono
em um retângulo com algumas manipulações, ainda que haja um desperdício de área no recipiente ao
resolver o problema desse modo.
A escolha de focar somente no empacotamento de retângulos foi feita justamente por isso, é possível
utilizá-lo para empacotar qualquer tipo de item, com as devidas adaptações.

Antes de abordar o problema (\autoref{ch:problema-de-empacotamento}) e buscar soluções
(\Cref{ch:bottom-left}), alguns conceitos básicos são mostrados.
O \autoref{ch:conceitos-basicos} foca em modelos de otimização e definições
(\cref{sec:modelos-de-otimizacao,sec:tipos-de-modelo}) e a diferença entre métodos exatos e
heurísticos (\cref{sec:metodos-exatos-heuristicos}).
No \autoref{ch:problema-de-empacotamento} é dada a definição do problema (\autoref{sec:definicao}),
para então mostrar algumas classificações (\autoref{sec:classificacao}) e variantes
(\autoref{sec:variantes}).
Depois, é explicada a heurística \textit{bottom-left} (\autoref{ch:bottom-left}) e os métodos feitos
com base nela, os quais serão utilizados na resolução das instâncias de teste.
Por fim, no \Cref{ch:resultados}, os principais resultados obtidos são mostrados.

O principal objetivo deste trabalho é criar métodos de solução para o problema de
empacotamento da mochila de peças retangulares, todos baseados na heurística \textit{bottom-left}.
Outros objetivos mais específicos são: implementar a \textit{bottom-left} e os métodos derivados em
Python, executá-los com instâncias de teste da literatura, comparar seus resultados e identificar
vantagens e desvantagens de cada um.
