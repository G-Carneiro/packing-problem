\section{Classificação}\label{sec:classificacao}
% TODO: adicionar imagens e revisão bibliografica para cada classe
% FIXME: adicionar formalizações. (Todos tem a mesma formalização apresentada antes?)

Por existirem diferentes objetivos na solução de um problema de empacotamento foram criadas algumas
classificações.
Algumas delas (as principais) são mostradas em \citeauthoryear{exact-solution-techniques} e
\citeauthoryear{2DPackLib}, as quais serão exploradas em seguida, com alguns exemplos já vistos
na~\nameref{ch:introducao}.

O objetivo do \textbf{Empacotamento 2D em Faixa}, em inglês \textit{Two-Dimensional Strip Packing
Problem} (2D-SPP), é encontrar um empacotamento de altura mínima $H$ para um dado conjunto de itens
$\mathcal{I}$ em uma caixa $\mathcal{B} = (W, H)$ com comprimento fixo $W$.
Muito aplicado na área têxtil para minimizar o comprimento de tecido cortado para fazer peças
de roupas.

No \textbf{Empacotamento 2D da Mochila}, em inglês \textit{Two-Dimensional Knapsack Problem} (2D-KP),
dado um conjunto de itens $\mathcal{I}$, onde cada item $i \in \mathcal{I}$ é associado a um valor
$p_i$, e uma caixa $\mathcal{B}$, deve-se encontrar um subconjunto $\mathcal{I}' \subseteq
\mathcal{I}$ que maximize $\sum_{i \in \mathcal{I'}}^{} p_i$.
Geralmente o valor $p_i$ é dado pela área do item, dessa forma, outra interpretação do problema
seria minimizar a área desperdiçada (vazia) da caixa $\mathcal{B}$.
Pode ser utilizado para maximizar o número de peças cortadas de um pedaço de couro, por exemplo.


Já o \textbf{Empacotamento 2D em Caixas}, em inglês \textit{Two-Dimensional Cutting Stock Problem}
(2D-CSP), envolve encontrar uma solução que minimize o número de caixas idênticas necessárias para
empacotar todos os itens $i \in \mathcal{I}$, onde cada item possui uma demanda
$d_i \in \mathbb{Z}_+$ (número mínimo de cópias do item que precisam ser empacotadas).
Existe uma versão menos genérica do 2D-CSP, o \textit{Two-Dimensional Bin Packing Problem} (2D-BPP),
onde a demanda $d_i$ de cada item é 1.
As caixas podem possuir diferentes tamanhos ao utilizar uma variante (\cref{sec:variantes}),
mas a maioria dos problemas lida com as mesmas dimensões.
Facilmente aplicável na área logística e de transporte, seja minimizando o número de paletes
ou veículos de entrega.

Por fim, no \textbf{Empacotamento 2D Ortogonal}, em inglês \textit{Two-Dimensional Orthogonal
Packing Problem} (2D-OPP), busca-se uma solução, caso exista, para empacotar \textbf{todos} os
itens $i \in \mathcal{I}$ na caixa $\mathcal{B}$.
Usado em situações onde se precisa alocar todos os itens dentro de um caminhão.

Todos os problemas descritos são NP-difícil, com exceção do Ortogonal, sendo NP-completo~\cite{2DPackLib}.
Existem resultados recentes para todas as classes do problema~\cite{
    cote2014combinatorial,delorme2017logic,velasco2019improved,martin2020models,mrad2015arc,
    cintra2008algorithms,furini2016modeling}.
A \Cref{fig:exemplo-classificacoes} traz exemplos de solução ótima para 2D-SPP, 2D-BPP e 2D-KP,
para um dado conjunto de itens.

\begin{figure}[!htb]
    \centering
    \caption{Exemplo de solução ótima para algumas classes.}
    \includegraphics[scale=0.8]{utils/images/exemplo_classificacoes}
    \label{fig:exemplo-classificacoes}
    \legend{(a) conjunto de itens; (b) solução ótima para 2D-SPP; (c) solução ótima para 2D-BPP;
        (d) solução ótima para 2D-KP (se os valores dos itens corresponderem a sua área).}
    \fonte{\!\cite{exact-solution-techniques}.}
\end{figure}


Este trabalho resolverá as instâncias de teste somente para o 2D-KP, onde o valor $p_i$ de cada
item $i$ será sua própria área, independente do valor original da instância (caso tenha).
