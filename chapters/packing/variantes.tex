\section{Variantes}\label{sec:variantes}


Variantes são pequenas alterações no escopo do problema, também podem ser vistas como restrições
ou relaxamento.
Existem quatro comuns~\cite{2DPackLib,exact-solution-techniques}, as quais são descritas a seguir.

\textbf{Corte guilhotinado} consiste em cortar a caixa de forma paralela a um de seus lados de ponta
a ponta recursivamente, é útil na resolução de problemas de corte (problemas de empacotamento
são equivalentes aos de corte).
Máquinas de corte, usualmente, só conseguem produzir peças por meio de sequências de corte
guilhotinados, ou seja, de ponta a ponta e paralelos aos lados do recipiente.
Isso pode ser um problema, já que, com essa restrição, nem todos os conjuntos de itens são
compatíveis com tal método (\Cref{fig:corte-guilhotinado}).
Frequentemente ainda existe um limite $k$ de cortes por recipiente, conhecidos como problemas de
$k$-estágios.
Geralmente, $k$ é igual a 2 ou 3, com um corte extra chamado \textit{trimming}.

\begin{figure}[H]
    \centering
    \caption{Exemplos de corte guilhotinado.}
    \includegraphics{utils/images/corte_guilhotinado}
    \label{fig:corte-guilhotinado}
    \legend{(a) não permite corte guilhotinado; (b) corte de 2-estágios com \textit{trimming};
        (c) corte de 3-estágios com \textit{trimming}.}
    \fonte{\citeauthoryear{exact-solution-techniques}.}
\end{figure}


\textbf{Rotações ortogonais} são um modo de relaxar o problema, permitindo rotações de 90 graus
para os itens a serem alocados.
Porém, isso também leva a desafios mais complexos, pois aumenta o número de decisões a serem tomadas,
fazendo o modelo possuir mais varáveis e restrições.
Essa variante, geralmente, é tratada adicionando um parâmetro de decisão binária $r_i$ para cada
item $i \in \mathcal{I}$, indicando se a rotação do item é permitida ou não.

\textbf{Restrições de carga e descarga} implicam que algumas peças \textbf{devem} ser posicionadas
em dada posição para não ser necessário mover outros itens quando tal peça for carregada/descarregada.
Usando como exemplo um caminhão de entregas, visa evitar situações onde um produto
precisa ser descarregado para se ter acesso a um item mais ao fundo e então carregar novamente o
primeiro item.
O tratamento da variante pode ser feito ao fixar as coordenadas $(x_i, y_i)$, dos itens
$i \in \mathcal{I}$ que possuem essa restrição, no modelo.

Existem variantes aplicáveis somente a algumas categorias do problema, é o caso de \textbf{caixas
de tamanho variável}, aplicável ao 2D-CSP e ao 2D-BPP e define que caixas
não precisam possuir as mesmas dimensões, custos e disponibilidade.
Com essa variante, o objetivo passa a ser empacotar todos os itens com um custo mínimo de recipientes.
