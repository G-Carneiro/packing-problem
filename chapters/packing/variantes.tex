\section{Variantes}\label{sec:variantes}


Variantes são pequenas alterações no escopo do problema, também podem ser vistas como restrições ou relaxamento.
Existem quatro mais comuns~\cite{2DPackLib}, as quais são descritas a seguir.

\textbf{Corte guilhotinado} consiste em cortar a caixa de forma paralela a um de seus lados recursivamente, é útil na resolução de problemas de corte (problemas de empacotamento podem ser reduzidos para essa categoria e vice-versa).
\textbf{Rotações ortogonais} são um modo de relaxar o problema, permitindo rotações de 90 graus para os itens a serem alocados.

\textbf{Restrições de carga e descarga} implicam que algumas peças \textbf{devem} ser posicionadas em dada posição, usando como exemplo um caminhão de entregas, visa evitar situações onde um produto precisa ser descarregado para se ter acesso a um item mais ao fundo e então carregar novamente o primeiro item.
Existem variantes aplicáveis somente a algumas categorias do problema, é o caso de \textbf{caixas de tamanho variável}  que pode ser unida ao \textbf{Empacotamento 2D em Caixas} e define que caixas não têm de ter as mesmas dimensões.
