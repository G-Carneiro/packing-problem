\section{Definição}\label{sec:definicao}

De acordo com~\cite{2DPackLib}, dado uma caixa retangular $\mathcal{B} = (W, H)$ de comprimento $W \in \mathbb{Z}_+$ e altura $H \in \mathbb{Z}_+$ e um conjunto $\mathcal{I}$ de itens também retangulares, onde cada item $i \in \mathcal{I}$ com comprimento $w_i \in \mathbb{Z}_+, w_i \le W$ e altura $h_i \in \mathbb{Z}_+, h_i \le H$.
Um empacotamento $\mathcal{I}' \subseteq \mathcal{I}$ em $\mathcal{B}$ pode ser descrito como uma função $\mathcal{F}: \mathcal{I}' \to \mathbb{Z}_+^2$ que mapeie cada item $i \in \mathcal{I}'$ para um par de coordenadas $\mathcal{F}(i) = (x_i, y_i)$, de forma

\begin{restrictions}
    x_i \in \{0, \dots, W - w_i\}, y_i \in \{0, \dots, H - h_i\} \left(i \in \mathcal{I}'\right) \label[restriction]{eq:1} \\
    [x_i, x_i + w_i) \cap [x_j, x_j + w_j) = \emptyset \text{ ou } [y_i, y_i + h_i) \cap [y_j, y_j + h_j) = \emptyset \left(i, j \in \mathcal{I}', i \neq j\right) \label[restriction]{eq:2}
\end{restrictions}


Nessa forma de representação a caixa está posicionada no plano cartesiano, com seu canto inferior esquerdo na origem.
Já as coordenadas $\mathcal{F}(i) = (x_i, y_i)$ representam a posição em que o canto inferior esquerdo da peça será alocado.
A \Cref{eq:1} garante que cada item deve estar inteiramente na caixa, enquanto a \Cref{eq:2} impede sobreposição entre peças.
Ambas restrições indicam uma orientação fixa, ou seja, peças não podem ser rotacionadas.
