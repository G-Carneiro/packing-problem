\chapter{Problema de Empacotamento}\label{ch:problema-de-empacotamento}

Nas últimas três décadas, as publicações na área de corte e empacotamento tiveram um aumento
considerável~\cite{exact-solution-techniques,wascher2007improved}.
Devido a isso, a categorização e organização desses problemas é cada vez mais importante.
O trabalho de \citeauthor*{wascher2007improved} fornece a classificação de problemas de corte e
empacotamento baseado na sua dimensão, tipos de itens, tipo de recipiente e função objetivo,
fornecendo uma atualização da tipologia definida em~\cite{dyckhoff1990typology}.

O problema de empacotamento, é um problema de otimização de difícil resolução.
Seu objetivo é simples, colocar peças em um espaço $N$-dimensional, na \autoref{fig:packing-example}
é possível ver representações para os casos 1D, 2D e 3D\@.
Tanto as peças quanto o espaço, podem ser de formato regular (convexo) ou não (côncavo).
Pensando no caso 2D, triângulos, retângulos, círculos e outros polígonos convexos são
considerados regulares, enquanto estrelas e outros polígonos côncavos são irregulares.

\begin{figure}[!htb]
    \centering
    \caption{Representação para o problema de empacotamento 1D, 2D e 3D.}
    \includegraphics[scale=1]{utils/images/packing-example}
    \label{fig:packing-example}
    \fonte{\citeauthoryear{castellucci2019consolidation}.}
\end{figure}


Existem diversas formas de identificar se um polígono é convexo ou não.
A primeira delas é verificando se existe alguma diagonal que não pertença à região interna do
polígono, caso exista, o polígono é côncavo, caso contrário, convexo.
Também é possível identificar através dos ângulos internos, polígonos côncavos possuem pelo menos
um ângulo interno com mais de 180 graus.
Outra forma de definir se uma peça é regular ou não, é identificar o número de parâmetros
necessários para representá-la.
Se for preciso três ou mais é irregular, caso contrário, regular~\cite{aprendizado-reforco}.
A \autoref{fig:pieces-example} mostra alguns exemplos de peças regulares (à direita) e seus
contornos convexos (à esquerda).

\begin{figure}[H]
    \centering
    \caption{Exemplos de peças regulares e irregulares.}
    \label{fig:pieces-example}
    \includegraphics[scale=0.7]{utils/images/pieces-example}
    \fonte{\citeauthoryear{aprendizado-reforco}.}
%    \legend{}
\end{figure}


O foco deste trabalho será em problemas de empacotamento 2D de peças e objetos retangulares
ortogonais, sem qualquer variante (\autoref{sec:variantes}).
Por mais simples que seja, é uma categoria muito importante do problema, visto que, no mundo real,
a maioria do que temos interesse em resolver se encaixa nessas características.
Inclusive, existem vários trabalhos como~\cite{wei2011skyline} e outros mais recentes~\cite{
    martin2020models,firat2020effective,chen2019efficient} com o mesmo propósito.
Esse escopo também é utilizado para resolver outros problemas, como o planejamento de integração em
larga escala~\cite{huang2007efficient} e para o roteamento de veículos levando paletes~\cite{
    silva2022integer}.
Existem até mesmo instâncias padronizadas para realizar comparativos entre algoritmos~\cite{
    2DPackLib}, as instâncias usadas neste trabalho serão explicadas no \Cref{ch:resultados}.

Tratar somente de objetos retangulares não é um grande limitador para resolver com outros tipos
de itens, já que é possível usar os contornos convexos de um polígono côncavo para transformá-lo
em um polígono convexo.
Após isso, basta transformar o polígono regular em um retângulo.
Assim é possível empacotar qualquer polígono usando o empacotamento de retângulos, ainda que
haja uma área desperdiçada devido às transformações.

\section{Definição}\label{sec:definicao}

De acordo com~\cite{2DPackLib}, dado uma caixa retangular $\mathcal{B} = (W, H)$ de comprimento $W \in \mathbb{Z}_+$ e altura $H \in \mathbb{Z}_+$ e um conjunto $\mathcal{I}$ de itens também retangulares, onde cada item $i \in \mathcal{I}$ com comprimento $w_i \in \mathbb{Z}_+, w_i \le W$ e altura $h_i \in \mathbb{Z}_+, h_i \le H$.
Um empacotamento $\mathcal{I}' \subseteq \mathcal{I}$ em $\mathcal{B}$ pode ser descrito como uma função $\mathcal{F}: \mathcal{I}' \to \mathbb{Z}_+^2$ que mapeie cada item $i \in \mathcal{I}'$ para um par de coordenadas $\mathcal{F}(i) = (x_i, y_i)$, de forma

\begin{restrictions}
    x_i \in \{0, \dots, W - w_i\}, y_i \in \{0, \dots, H - h_i\} \left(i \in \mathcal{I}'\right) \label[restriction]{eq:1} \\
    [x_i, x_i + w_i) \cap [x_j, x_j + w_j) = \emptyset \text{ ou } [y_i, y_i + h_i) \cap [y_j, y_j + h_j) = \emptyset \left(i, j \in \mathcal{I}', i \neq j\right) \label[restriction]{eq:2}
\end{restrictions}


Nessa forma de representação a caixa está posicionada no plano cartesiano, com seu canto inferior esquerdo na origem.
Já as coordenadas $\mathcal{F}(i) = (x_i, y_i)$ representam a posição em que o canto inferior esquerdo da peça será alocado.
A \Cref{eq:1} garante que cada item deve estar inteiramente na caixa, enquanto a \Cref{eq:2} impede sobreposição entre peças.
Ambas restrições indicam uma orientação fixa, ou seja, peças não podem ser rotacionadas.

\section{Classificação}\label{sec:classificacao}
% TODO: adicionar imagens e revisão bibliografica para cada classe
% FIXME: adicionar formalizações. (Todos tem a mesma formalização apresentada antes?)

Por existirem diferentes objetivos na solução de um problema de empacotamento foram criadas algumas classificações.
Algumas delas (as principais) são mostradas em~\cite{exact-solution-techniques}, as quais serão exploradas em seguida, com alguns exemplos já vistos na \href{ch:introducao}{Introdução}.

O objetivo do \textbf{Empacotamento 2D em Faixa} é encontrar um empacotamento de altura mínima para um dado conjunto de itens em uma caixa com comprimento fixo.
Muito aplicado na área têxtil para minimizar o comprimento de tecido cortado para fazer peças de roupas.

No \textbf{Empacotamento 2D da Mochila} deve-se encontrar $\mathcal{I}' \subseteq \mathcal{I}$ que maximize o valor de $\mathcal{B}$.
Geralmente o valor é dado pela área de caixa ocupada pelos itens, dessa forma, outra interpretação do problema seria minimizar a área desperdiçada (vazia).
Pode ser utilizado para maximizar o número de peças cortadas de um pedaço de couro, por exemplo.

Já o \textbf{Empacotamento 2D em Caixas} envolve encontrar uma solução que minimize o número de caixas necessárias para empacotar todos os itens.
As caixas podem possuir diferentes tamanhos, mas a maioria dos problemas lida com as mesmas dimensões.
Facilmente aplicável na área logística e de transporte, seja minimizando o número de paletes ou veículos de entrega.

Por fim, no \textbf{Empacotamento 2D Ortogonal} busca-se uma solução, caso exista, para empacotar \textbf{todos} os itens na caixa.
Usado em situações onde se precisa alocar todos os itens dentro de um caminhão.

Todos os problemas descritos são NP-difícil, com exceção do Ortogonal, sendo NP-completo~\cite{2DPackLib}.

\section{Variantes}\label{sec:variantes}


Variantes são pequenas alterações no escopo do problema, também podem ser vistas como restrições ou relaxamento.
Existem quatro mais comuns~\cite{2DPackLib}, as quais são descritas a seguir.

\textbf{Corte guilhotinado} consiste em cortar a caixa de forma paralela a um de seus lados recursivamente, é útil na resolução de problemas de corte (problemas de empacotamento podem ser reduzidos para essa categoria e vice-versa).
\textbf{Rotações ortogonais} são um modo de relaxar o problema, permitindo rotações de 90 graus para os itens a serem alocados.

\textbf{Restrições de carga e descarga} implicam que algumas peças \textbf{devem} ser posicionadas em dada posição, usando como exemplo um caminhão de entregas, visa evitar situações onde um produto precisa ser descarregado para se ter acesso a um item mais ao fundo e então carregar novamente o primeiro item.
Existem variantes aplicáveis somente a algumas categorias do problema, é o caso de \textbf{caixas de tamanho variável}  que pode ser unida ao \textbf{Empacotamento 2D em Caixas} e define que caixas não têm de ter as mesmas dimensões.
