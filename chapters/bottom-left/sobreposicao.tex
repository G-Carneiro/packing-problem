\section{Sobreposição e domínio infinito}\label{sec:sobreposicao-e-dominio-infinito}
% TODO: talvez trocar para domínio contínuo

Por mais simples que a heurística \textit{bottom-left} seja, ainda existem duas complicações,
respeitar a \Cref{eq:2} sobre sobreposição de peças e o domínio infinito de coordenadas
ao tentar alocar um item.
As \Crefrange{fig:sobreposicao-dominio}{fig:sobreposicao-dominio5} serão usadas para
demonstrar as complicações citadas.

A \Cref{fig:sobreposicao-dominio} mostra o estado intermediário do algoritmo, onde o item 0
foi o primeiro alocado e o item 1 foi alocado a sua direita na posição (2, 0) para respeitar
a \Cref{eq:1}, porque não cabia logo acima na posição (0, 2).


\begin{figure}[H]
    \centering
    \caption{Itens 0 e 1 posicionados.}
    \includegraphics[scale=0.5]{utils/images/continuous_example}
    \label{fig:sobreposicao-dominio}
    \fonte{feito pelo autor.}
\end{figure}


Supondo que a próxima peça da fila tenha largura 3 e altura 1, ao posicioná-la na posição (0, 2)
percebe-se que a \Cref{eq:1} é satisfeita, mas a \Cref{eq:2} não (\Cref{fig:sobreposicao-dominio2}).

\begin{figure}[H]
    \centering
    \includegraphics[scale=0.5]{utils/images/continuous_example2}
    \caption{Resolvendo sobreposição e domínio infinito.}
    \label{fig:sobreposicao-dominio2}
\end{figure}


Nesse caso, com poucas peças alocadas e auxílio visual, é fácil identificar que a posição correta,
seguindo a lógica \textit{bottom-left}, seria a de coordenadas (0, 4)
(\Cref{fig:sobreposicao-dominio3}).
Porém, encontrar tal posição pode ser extremamente complexo.
Entre as coordenadas (0, 2) e (0, 4) o espaço é contínuo, existindo infinitas outras coordenadas
entre elas, sendo impossível checar todas.

\begin{figure}[!htb]
    \centering
    \includegraphics[scale=0.5]{utils/images/continuous_example3}
    \caption{Resolvendo sobreposição e domínio infinito.}
    \label{fig:sobreposicao-dominio3}
\end{figure}


Como todas as instâncias usadas para testes tratam somente de peças e recipientes com valores
inteiros, uma abordagem possível seria discretizar o domínio, conforme a
\Cref{fig:sobreposicao-dominio4}.

\begin{figure}[H]
    \centering
    \includegraphics[scale=0.5]{utils/images/discrete_example}
    \caption{Itens 0 e 1 posicionados, com domínio discreto.}
    \label{fig:sobreposicao-dominio4}
\end{figure}


Dessa forma somente coordenadas de valores inteiros precisariam ser verificadas, resolvendo
parcialmente o problema com domínio, já que ainda existiriam muitos pontos, principalmente em
instâncias com espaços grandes.
Mas isso ainda não resolve a parte de sobreposição.
Para cada ponto ainda é necessário verificar se existe sobreposição com cada uma das peças já
alocadas, algo extremamente custoso.

Na \Cref{fig:sobreposicao-dominio5} é possível notar que três coordenadas precisaram ser checadas
até encontrar uma que cumprisse as \Cref{eq:1,eq:2}.
Por se tratar de um estado inicial de um modelo pequeno esse processo não é tão custoso, porém em
modelos maiores ou estados mais avançados isso se tornará cada vez mais custoso, visto que o domínio
será maior e, principalmente, mais peças estarão alocadas para checar possíveis sobreposições.

\begin{figure}[H]
    \centering
    \caption{Itens 0, 1 e 2 posicionados, com domínio discreto.}
    \includegraphics[scale=0.5]{utils/images/discrete_example2}
    \label{fig:sobreposicao-dominio5}
    \fonte{feito pelo autor.}
\end{figure}


Outro fator o qual deve ser observado é de que a discretização do domínio pode não funcionar bem em
casos onde os valores trabalhados não sejam inteiros.
Nesses casos a discretização poderia ocorrer com algum nível de precisão nas casas decimais,
mas ainda assim as chances de desperdício de área são grandes.
Ou seja, muitos problemas reais seriam complexos de serem resolvidos com a discretização.
