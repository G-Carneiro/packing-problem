\subsection{Critérios de ordenação}\label{subsec:criterios-de-ordenacao}

Para determinar o impacto da ordenação da fila, cinco critérios de ordenação
foram escolhidos, sendo eles: área, perímetro, largura, altura e \textit{id}.
A ordenação por \textit{id} considera a ordem em que os itens foram colocados na lista, ou seja,
seria a forma padrão de resolver e ele será a base para definir se os demais critérios possuem
algum benefício.
Além disso, cada critério pode ser usado para ordenar a fila em ordem crescente ou decrescente,
algo que também será analisado.
Na literatura o mais comum é utilizar a ordenação decrescente pela área~\cite{chen2019efficient}.
