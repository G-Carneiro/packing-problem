\section{Conclusão}\label{sec:conclusao}
\addcontentsline{toc}{chapter}{\texorpdfstring{CONCLUSÃO E TRABALHOS FUTUROS}{Conclusão e trabalhos futuros}}

No trabalho foram avaliados experimentalmente 40 métodos de solução baseados na heurística
\textit{bottom-left} para o problema de empacotamento de retângulos, considerando a versão da
mochila e com o valor dos itens sua própria área.
O resultado mais óbvio obtido foi a supremacia da ordenação decrescente sobre a crescente
(\cref{subsec:ordenacao-crescente-decrescente}), não compensando utilizar a segunda para solucionar o
problema.

O alto uso da ordenação decrescente pela área na literatura~\cite{chen2019efficient} foi justificado
pelos resultados, já que essa composição conseguiu alguns dos melhores números
(\cref{subsec:comparativo-entre-criterios-de-ordenacao}).
Os resultados também indicam que qualquer critério de ordenação obtêm melhores resultados do que
deixar a fila desordenada (ordenação por \textit{id}).
Mas a alta competitividade de outros critérios de ordenação, esses não tão comuns na literatura,
conseguindo melhores resultados em algumas instâncias, mostra que ainda há espaço para outras
formas de ordenação.

Em relação às diferentes regiões criadas, por mais que regiões complexas tenham obtido melhores
resultados (\cref{subsec:comparativo-entre-criacao-de-regioes}), não compensa utilizá-las.
Ao executar todos os métodos de regiões simples é possível conseguir melhores resultados e em menor
tempo, quando comparado ao método de ordenação decrescente pela área e usando regiões complexas
(ver a \Cref{tab:superposition} na \cref{subsec:comparativo-entre-criacao-de-regioes}).
