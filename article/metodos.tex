\section{Métodos de solução}\label{sec:metodos-de-solucao}

Como descrito na \autoref{sec:introducao}, a maioria das classes do problema são NP-difíceis.
Isso torna métodos de soluções exatos, os quais buscam pela solução ótima, extremamente custosos
em tempo e recursos computacionais em instâncias de porte moderado, muitas vezes sendo inviáveis
por falta de algum desses dois motivos.
Consequentemente a literatura é dominada por abordagens que usam heurísticas e meta-heurísticas,
sendo a \textit{bottom-left} uma das principais estratégias de solução e será usada no estudo
deste trabalho.

A \textit{bottom-left} é uma heurística construtiva proposta por\citeauthoryear{baker1980orthogonal}.
Embora tenha sido proposta a décadas, ainda é bastante usada na literatura atual, além de poder
ser usada como componente de algoritmos mais sofisticados e para diferentes classes e variantes
do problema.
Ela foi utilizada nos trabalhos de \citeauthoryear{hopper2001empirical} na comparação de vários
métodos de solução, \citeauthoryear{wei2011skyline} trazendo uma revisão do método e seus derivados
e, mais recentemente, \citeauthoryear{chehrazad2022fast} através de uma adaptação para o
empacotamento de itens irregulares de forma gulosa.

Sua premissa é simples, dado uma fila de itens como entrada, enquanto ela não estiver vazia,
basta retirar o primeiro item dela e alocar no canto mais a baixo e à esquerda quanto for
possível~\cite{aprendizado-reforco}, sem sobreposições entre peças.
Caso não exista uma posição válida, a peça é desconsiderada e passa-se para próxima da fila.
A \autoref{fig:bottom-left} mostra um exemplo de alocação para um dado conjunto de peças regulares.

\begin{figure}[H]
    \centering
    \includegraphics{utils/images/bottom-left}
    \caption{Representação de alocação usando \textit{bottom-left}.}
    \label{fig:bottom-left}
    \fonte{\cite{aprendizado-reforco}}
\end{figure}


Vale destacar que a própria ordem da fila pode gerar resultados diferentes, alterando a qualidade
da solução.
Um dos resultados esperados deste trabalho é identificar se há alguma forma de ordenação que
se destaque na qualidade de solução, através da comparação entre os diferentes modos.
Para isso, serão usados conjuntos de instâncias frequentemente utilizados na literatura.


\subsection{Critérios de ordenação}\label{subsec:criterios-de-ordenacao}

Para determinar o impacto da ordenação da fila, cinco critérios de ordenação
foram escolhidos, sendo eles: área, perímetro, largura, altura e \textit{id}.
A ordenação por \textit{id} considera a ordem em que os itens foram colocados na lista, ou seja,
seria a forma padrão de resolver e ele será a base para definir se os demais critérios possuem
algum benefício.
Além disso, cada critério pode ser usado para ordenar a fila em ordem crescente ou decrescente,
algo que também será analisado.
Na literatura o mais comum é utilizar a ordenação decrescente pela área~\cite{chen2019efficient}.

\section{Criação de regiões}\label{sec:criacao-de-regioes}
% TODO: talvez dar nomes para soluções com e sem região

Os dois problemas expostos na \cref{sec:sobreposicao-e-dominio-infinito} podem ser facilmente
resolvidos utilizando a estratégia de criação de regiões.
Com essa técnica é possível ignorar a \Cref{eq:2}.
Nela, ao posicionar uma peça, duas regiões são criadas (\Cref{fig:regiao-vertical}) e o item
seguinte será somente posicionado se couber em uma das regiões disponíveis.

\begin{figure}[H]
    \centering
    \caption{Regiões criadas traçando uma linha vertical.}
    \includegraphics[scale=0.5]{output/figures/bkw/bkw01/vertically/height/false/01}
    \label{fig:regiao-vertical}
    \fonte{feito pelo autor.}
\end{figure}


Agora o domínio passa a ser somente o canto inferior esquerdo de cada uma das regiões e
sobreposições deixam de ser possíveis.
Além disso, a regra para definir se uma peça cabe em dada região é igual a \Cref{eq:1}, tornando
o algoritmo de solução bem simples.
A fim de identificar o impacto das regiões na solução do modelo, quatro formas de criação
delas foram usadas.

A primeira delas é \textbf{traçando uma linha vertical} a partir do canto superior direito de cada
peça alocada (\Cref{fig:regiao-vertical}).
A segunda é igual a primeira, porém \textbf{traçando uma linha horizontal} (\Cref{fig:regiao-horizontal}).

\begin{figure}[!htb]
    \centering
    \includegraphics[scale=0.5]{output/figures/bkw/bkw01/horizontally/height/false/01}
    \caption{Regiões criadas traçando uma linha horizontal.}
    \label{fig:regiao-horizontal}
\end{figure}


Já na terceira, a linha traçada (vertical ou horizontal) depende da área das regiões criadas
com cada linha.
Nesse modo o objetivo é maximizar a área de uma das regiões geradas, ele identifica qual linha
irá gerar a região de maior área e a traça.
Por exemplo, a \Cref{fig:regiao-vertical} gerou uma região com 252 de área e outra com 1320,
enquanto a \Cref{fig:regiao-horizontal} obteve regiões com 1440 e 132, então, nesse caso, a linha
traçada será a horizontal (\Cref{fig:regiao-max}).

\begin{figure}[H]
    \centering
    \includegraphics[scale=0.5]{output/figures/bkw/bkw01/max_area/height/false/01}
    \caption{Regiões criadas maximizando uma das regiões.}
    \label{fig:regiao-max}
\end{figure}


Maximizar uma região pode ser interessante, pois aumenta as chances do próximo item conseguir ser
alocado, visto que uma das regiões será mais espaçosa.
Em contrapartida, esse método também pode acabar gerando muitas regiões pequenas que não sejam
utilizadas, diminuindo a qualidade da solução.

No quarto e último modo de criar regiões nenhuma linha é traçada, todas as regiões vão até o final
do recipiente (\Cref{fig:regiao-none}).
Nesse caso, sobreposições de peças podem ocorrer, então verificações são necessária para cumprir
a \Cref{eq:2}.
Ao fazer isso, possibilita que mais peças sejam alocadas, visto que todas as regiões possuem área
máxima.
Esse modo foi criado para identificar se é de fato melhor que os demais e qual seu custo.

\begin{figure}[!htb]
    \centering
    \includegraphics[scale=0.5]{output/figures/bkw/bkw01/none/height/false/01}
    \caption{Regiões criadas possibilitando sobreposições.}
    \label{fig:regiao-none}
\end{figure}


Com os critérios para criação de regiões explicados, é possível diferenciá-los em dois tipos.
O primeiro é dos que permitem sobreposição entre peças e, por isso, precisam de verificações para
respeitar a \Cref{eq:2}, nesse tipo se encaixa somente o quarto modo.
O segundo tipo contém os três primeiros critérios, onde somente a \Cref{eq:1} precisa ser checada.
