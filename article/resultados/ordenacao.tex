\subsection{Ordenação crescente × decrescente}\label{subsec:ordenacao-crescente-decrescente}

Uma primeira observação é a discrepância na qualidade de solução entre a ordenação crescente
e a decrescente, algo já esperado.
Na \Cref{tab:ordenacao} é possível notar que ordenando de forma decrescente é possível ocupar
cerca de $20\%$ a mais do espaço (coluna “Qualidade \%”), quando comparado a ordenação crescente,
em média.

\begin{table}[!htb]
    \centering
    \caption{Resultado da comparação entre Descending.}
    \label{tab:descending}
    \IBGEtab{}{
        \begin{tabular}{lrrrrr}
            \hline
            Descending & Wons & Draws & Quality \% & Items \% & Time (s) \\
            \hline
            F          & 167  & 8     & 57.306     & 47.6518  & 2.37153  \\
            T          & 736  & 8     & 78.9136    & 46.3642  & 1.77985  \\
            \hline
        \end{tabular}
    }{}
    \fonte{autor}
\end{table}

Com isso, fica claro que ordenar a fila de entrada da \textit{bottom-left} de modo decrescente é
vantajoso em termos de qualidade, quantidade e tempo de execução.
