\subsection{Comparativo entre critérios de ordenação}\label{subsec:comparativo-entre-criterios-de-ordenacao}

A \Cref{tab:ordenacoes-true} mostra o comparativo entre os critérios de criação de regiões,
somente considerando a ordenação decrescente, já que não existem motivos para usar a crescente.
Representando os dados dessa forma fica fácil identificar que utilizar algum critério de ordenação
para a fila de entrada é vantajoso, pois ao usar o \textit{id} os resultados foram os piores.
Além disso, percebe-se que as ordenações por área e perímetro obtiveram os melhores resultados,
ainda que os demais também sejam competitivos.

\begin{table}[!htb]
    \centering
    \caption{Resultado da comparação entre critérios de ordenação decrescente.}
    \label{tab:ordenacoes-true}
    \IBGEtab{}{
        \ttfamily\begin{tabular}{lrrrr}
    \hline
    Ordenação & Vitórias & Empates & Qualidade \% & Tempo (s)  \\
    \hline
    Área      & 63       & 39      & 82.7353      & 1.5874e+00 \\
    Perímetro & 71       & 38      & 84.6986      & 1.5769e+00 \\
    Altura    & 40       & 16      & 77.4182      & 1.5655e+00 \\
    Largura   & 66       & 24      & 81.1899      & 2.0805e+00 \\
    Id        & 16       & 5       & 68.5261      & 2.0889e+00 \\
    \hline
\end{tabular}
    }{}
    \fonte{feito pelo autor.}
\end{table}

A alta competitividade entre os critérios de ordenação é interessante, pois a maioria dos trabalhos
na literatura, como o de \citeauthoryear{chen2019efficient}, usam somente ordenação pela área, e
isso pode ser um forte indicativo que os demais critérios devem ser mais explorados em certas
circunstâncias.
