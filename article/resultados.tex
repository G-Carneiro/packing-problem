\section{Resultados}\label{sec:resultados}

Para testar os métodos de solução criados foram usadas 45 instâncias de teste da literatura,
separadas em cinco conjuntos de instância de características diferentes:
BKW, GCUT, NGCUT, OF e OKP\@.
Todas as elas foram obtidas através da biblioteca pública
\href{https://site.unibo.it/operations-research/en/research/2dpacklib}{2DPackLib}\footnote{
    Disponível em: https://site.unibo.it/operations-research/en/research/2dpacklib.
    Acessado em: \today.}~\cite{2DPackLib}.

O foco do trabalho é no empacotamento 2D da mochila, com o critério de maximização sendo a
área ocupada do espaço.
Mas nem todos conjuntos foram feitos para ser resolvidos dessa forma, nesses casos foram
feitas leves adaptações para usá-los.
O motivo de usar instâncias feitas com outro objetivo é para não viciar o modelo em instâncias
específicas.

Como são cinco critérios de ordenação~(\cref{subsec:criterios-de-ordenacao}), com cada critério
podendo ser crescente ou decrescente, quatro formas de criar regiões~(\cref{subsec:criacao-de-regioes})
e 45 instâncias, tem-se o total de 1800 casos de teste.
Além disso, para conseguir resultados mais fiéis, a média, mediana e desvio padrão do tempo de
execução foram calculados.
Por isso, cada caso foi executado cinco vezes, totalizando 9000 execuções.
Outros dados como a qualidade de solução (objetivo do trabalho), porcentagem de itens alocados
e tempo, também foram computados.
Todas as execuções foram feitas em um mesmo computador, com configurações conforme a
\Cref{tab:config}.

\begin{table}
    \centering
    \caption{Configuração do computador de testes.}
    \label{tab:config}
    \begin{tabular}{ll}
        \hline
        CPU    & AMD Ryzen™ 5 3600X       \\
        RAM    & 16 GiB                   \\
        Python & 3.11.0                   \\
        SO     & Linux Mint 21.1 Cinnamon \\
        Kernel & 5.15                     \\
        \hline
    \end{tabular}
\end{table}


Ao analisar a média, a mediana e o desvio padrão do tempo de execução, observou-se que a média
e mediana possuem valores quase idênticos, enquanto o desvio padrão é pequeno ao ponto de poder
ser ignorado, indicando que cinco execuções por caso de teste são suficientes.
Portanto, a mediana e desvio padrão serão omitidos no restante do trabalho, podendo ser encontrados
na versão completa dos dados gerados no \href{https://github.com/G-Carneiro/packing-problem/}{
    Github}\footnote{Disponível em: https://github.com/G-Carneiro/packing-problem/. Acessado em: \today.}.

Nas tabelas apresentadas nas seções seguintes, as colunas “Vitórias” e “Empates” trazem os
resultados de forma quantitativa, enquanto a coluna “Qualidade \%” de modo qualitativo.
Nos testes, o valor $p_i$ dos itens sempre é sua área, desconsiderando os valores dados pelas
instâncias, caso hajam.

A qualidade de solução é determinada pela área ocupada do recipiente, no caso de todos os itens
serem alocados, a qualidade é $100\%$ independente da área do recipiente.
A coluna “Vitórias” indica quantas vezes tal método de solução obteve o melhor resultado em
comparação com os demais métodos em outras linhas.
Enquanto a coluna “Empates” mostra a quantidade de vezes que o método conseguiu a melhor qualidade,
mas outros também conseguiram.
Essas colunas foram feitas da seguinte forma: entre cada combinação de critério de ordenação,
modo de criar regiões e instâncias, é feita a comparação se a qualidade de solução foi melhor
para ordenação crescente ou decrescente.
No caso de ambas conseguirem o melhor resultado, é acrescido 1 tanto na coluna “Vitórias”, quanto
na “Empates” de ambas.
Por fim, a coluna “Tempo (s)” mostra o tempo médio de execução do método em segundos.

\subsection{Ordenação crescente × decrescente}\label{subsec:ordenacao-crescente-decrescente}

Uma primeira observação é a discrepância na qualidade de solução entre a ordenação crescente
e a decrescente, algo já esperado.
Na \Cref{tab:ordenacao} é possível notar que ordenando de forma decrescente é possível ocupar
cerca de $20\%$ a mais do espaço (coluna “Qualidade \%”), quando comparado a ordenação crescente,
em média.

\begin{table}[!htb]
    \centering
    \caption{Resultado da comparação entre Descending.}
    \label{tab:descending}
    \IBGEtab{}{
        \begin{tabular}{lrrrrr}
            \hline
            Descending & Wons & Draws & Quality \% & Items \% & Time (s) \\
            \hline
            F          & 167  & 8     & 57.306     & 47.6518  & 2.37153  \\
            T          & 736  & 8     & 78.9136    & 46.3642  & 1.77985  \\
            \hline
        \end{tabular}
    }{}
    \fonte{autor}
\end{table}

Com isso, fica claro que ordenar a fila de entrada da \textit{bottom-left} de modo decrescente é
vantajoso em termos de qualidade, quantidade e tempo de execução.

\subsection{Comparativo entre critérios de ordenação}\label{subsec:comparativo-entre-criterios-de-ordenacao}

A \Cref{tab:ordenacoes-true} mostra o comparativo entre os critérios de criação de regiões,
somente considerando a ordenação decrescente, já que não existem motivos para usar a crescente.
Representando os dados dessa forma fica fácil identificar que utilizar algum critério de ordenação
para a fila de entrada é vantajoso, pois ao usar o \textit{id} os resultados foram os piores.
Além disso, percebe-se que as ordenações por área e perímetro obtiveram os melhores resultados,
ainda que os demais também sejam competitivos.

\begin{table}[!htb]
    \centering
    \caption{Resultado da comparação entre critérios de ordenação decrescente.}
    \label{tab:ordenacoes-true}
    \IBGEtab{}{
        \ttfamily\begin{tabular}{lrrrr}
    \hline
    Ordenação & Vitórias & Empates & Qualidade \% & Tempo (s)  \\
    \hline
    Área      & 63       & 39      & 82.7353      & 1.5874e+00 \\
    Perímetro & 71       & 38      & 84.6986      & 1.5769e+00 \\
    Altura    & 40       & 16      & 77.4182      & 1.5655e+00 \\
    Largura   & 66       & 24      & 81.1899      & 2.0805e+00 \\
    Id        & 16       & 5       & 68.5261      & 2.0889e+00 \\
    \hline
\end{tabular}
    }{}
    \fonte{feito pelo autor.}
\end{table}

A alta competitividade entre os critérios de ordenação é interessante, pois a maioria dos trabalhos
na literatura, como o de \citeauthoryear{chen2019efficient}, usam somente ordenação pela área, e
isso pode ser um forte indicativo que os demais critérios devem ser mais explorados em certas
circunstâncias.

\subsection{Comparativo entre criação de regiões}\label{subsec:comparativo-entre-criacao-de-regioes}

Conforme mostra a \autoref{tab:regioes-true}, os modos criados traçando uma linha vertical ou
horizontal apresentaram qualidades semelhantes e
os menores tempos de execução, mas o método o qual traça uma linha vertical obteve mais vitórias.
Regiões criadas para maximizar uma das mesmas conseguiram o segundo melhor resultado qualitativo e
quantitativo, ao custo de um pequeno acréscimo no tempo de execução em relação aos dois primeiros.
O último modo de fato conseguiu os melhores resultados, porém a um custo altíssimo, levando cerca
de 1000 vezes mais tempo que métodos mais rápidos.

\begin{table}[!htb]
    \centering
    \caption{Resultado da comparação entre criação de regiões.}
    \label{tab:regioes-true}
    \IBGEtab{}{
        \ttfamily\begin{tabular}{lrrrrr}
\hline
Região & Wons & Draws & Quality \% & Items \% & Time (s)   \\
\hline
V      & 98   & 79    & 76.4030    & 45.0191  & 2.7157e-03 \\
H      & 70   & 60    & 75.9970    & 45.5439  & 6.2101e-03 \\
M      & 104  & 89    & 79.7175    & 47.6795  & 1.3743e-02 \\
N      & 176  & 119   & 83.6420    & 47.2335  & 7.2176e+00 \\
\hline
\end{tabular}
    }{}
    \fonte{autor}
\end{table}

A \Cref{tab:superposition} traz um comparativo entre os dois tipos de criação de regiões,
os que é preciso checar sobreposição e os que não (\cref{subsec:criacao-de-regioes}).
Na primeira linha da tabela a coluna “Qualidade \%” representa a média do melhor resultado obtido
em cada instância e a coluna “Tempo Total (s)” mostra a soma dos tempos que cada método de solução
levou para cada instância.
As duas colunas consideram somente métodos de solução que usam regiões onde não são necessárias
verificações de sobreposição, ou seja, são considerados 30 modos de solução.
A segunda linha considera somente método de solução o qual utiliza ordenação decrescente pela área
e criação de regiões onde é necessário verificar sobreposições, esse método foi escolhido para o
comparativo por apresentar os melhores resultados quantitativos e ser o segundo melhor
qualitativamente.

\begin{table}[!htb]
    \centering
    \caption{Resultado da comparação entre tipos de regiões.}
    \label{tab:superposition}
    \IBGEtab{}{
        \ttfamily\begin{tabular}{lrrr}
    \hline
    Superposition & Quality \% & Time (s)   \\
    \hline
    No            & 90.8278    & 1.6299e+01 \\
    Yes           & 87.2957    & 2.8313e+02
    \hline
\end{tabular}
    }{}
    \fonte{autor}
\end{table}

Com a \Cref{tab:superposition} fica claro que, por mais que usar regiões onde é preciso checar
sobreposições apresente o melhor resultado, é melhor executar todos demais métodos os quais não
precisem e escolher somente o de melhor solução, pois assim é possível obter, na média, soluções
de maior qualidade e ainda levando 10 vezes menos tempo.
