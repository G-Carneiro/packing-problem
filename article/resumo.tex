\begin{resumo}
    Problemas de empacotamento consistem em alocar um conjunto de itens $\mathcal{I}$ em uma
    caixa $\mathcal{B}$.
    No problema de empacotamento da mochila, foco deste trabalho, cada item é associado a um valor
    e busca-se uma solução que maximize a soma dos valores dos itens alocados.
    Este trabalho compara 40 métodos de solução criados com base na heurística
    \textit{bottom-left} para o problema de empacotamento de retângulos.
    Os métodos criados são uma combinação de diferentes formas de ordenação dos itens e criação
    de regiões, as quais evitam as sobreposições e o domínio contínuo presentes no problema.
    O principal resultado foi a alta competitividade de diferentes modos de ordenação,
    não sendo a área a única relevante, com o perímetro obtendo os melhores resultados.
\end{resumo}

% --------------------------------------------
% Caso seja necessário um resumo em inglês,
% descomentar linhas abaixo.
% --------------------------------------------
\begin{abstract}
    Packing problems consist of allocating a set of items $\mathcal{I}$ into a box $\mathcal{B}$.
    In the knapsack packing problem, the focus of this work, each item is associated with a value
    and a solution is sought that maximizes the sum of the values of the allocated items.
    This work compare 40 created solution methods based on \textit{bottom-left}
    heuristic for the rectangle packing problem.
    The methods created are a combination of different ways of ordering items and creating regions,
    which avoid superposition and continuous domain present in the problem.
    The main result was the high competitiveness of different ordering modes, the area not being
    the only relevant one, with the perimeter obtaining the best results.
\end{abstract}
