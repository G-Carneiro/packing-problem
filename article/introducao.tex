\section{Introdução}\label{sec:introducao}

Serviços de loja \textit{online} com entrega como Amazon e Mercado Livre estão tornando-se cada vez
mais presentes no dia a dia.
Para tornar as entregas mais rápidas é necessário fazer uma série de estudos de logística e
planejamento sobre como organizar os produtos nos estoques e nos veículos de entrega~\cite{
    silva2022integer,morabito1992abordagem}, muitas vezes sendo necessário considerar a ordem
em que eles precisarão ser retirados.
Além de tornar o processo mais rápido, a organização também pode permitir o melhor uso de espaços,
aumentando a quantidade máxima de itens ou evitando o desperdício dos espaços.

Ainda sobre evitar desperdícios, esse quesito é muito importante para as indústrias de papel,
móveis, têxtil e metal-mecânica~\cite{queiroz2022estudo,cavali2004problemas,belluzzo2005otimizacao}.
Todas essas áreas querem gerar o máximo de produtos com o mínimo de recursos materiais utilizados,
para evitar o descarte desnecessário do material e prejuízos financeiros.

Os problemas citados são considerados problemas de corte e empacotamento.
Problemas de corte envolvem cortar um objeto, como blocos de gesso, chapas de aço e barras de ferro,
em itens menores.
Enquanto problemas de empacotamento tratam sobre alocar um conjunto de itens $\mathcal{I}$ em um
recipiente $\mathcal{B}$.
Ambos são equivalentes entre si e é possível separá-los de acordo com sua dimensão.

O caso 2D, dimensão de estudo deste trabalho, possui uma vasta literatura de métodos de solução.
As abordagens de solução se dividem entre exatas, que buscam a solução ótima do problema, e
heurísticas, as quais podem não encontrar uma solução ótima, mas conseguem uma solução aceitável em
tempo hábil.
Dentre os métodos de solução exatos, um que se destaca é o procedimento de busca em árvore~\cite{
    beasley1985exact}, mas existem muitos outros na literatura~\cite{exact-solution-techniques,
    fekete1997new,delorme2016bin,kenmochi2009exact}.
Na parte de heurísticas, tem-se a \textit{bottom-left}~\cite{baker1980orthogonal,chehrazad2022fast}
e \textit{skyline}~\cite{wei2011skyline}, as heurísticas também possuem grande presença na
literatura~\cite{burke2004new,rakotonirainy2020improved,hopper2001empirical,chen2019efficient,
    huang2007efficient,hopper2001review}.

Este trabalho visa criar métodos de solução para o problema de empacotamento no espaço de duas
dimensões, onde as peças
são retangulares e com um recipiente também retangular, mais especificamente na versão do
empacotamento 2D da mochila, considerado NP-difícil~\cite{2DPackLib}.
Nessa versão do problema, dado um conjunto de itens $\mathcal{I}$, com cada item $i$ possuindo um
valor $p_i$, e uma caixa $\mathcal{B}$, o objetivo é maximizar a soma dos valores dos itens alocados
dentro do recipiente.
A abordagem escolhida para resolver o problema foi utilizar a heurística \textit{bottom-left},
devido a sua simplicidade e aos limites computacionais e de tempo ao escolher algum método exato.
Mesmo com a heurística sendo proposta em 1980, ela ainda está presente
na literatura recente~\cite{chehrazad2022fast,hopper2001empirical,wei2011skyline}.

O principal objetivo deste trabalho é criar métodos de solução para o problema de
empacotamento da mochila de peças retangulares, todos baseados na heurística \textit{bottom-left}.
Outros objetivos mais específicos são: implementar a \textit{bottom-left} e os métodos derivados em
Python, executá-los com instâncias de teste da literatura, comparar seus resultados e identificar
vantagens e desvantagens de cada um.
