\section{Conclusão}\label{sec:conclusao}
\begin{frame}
    \frametitle{Conclusão}
    \begin{itemize}
        \item Múltiplos métodos de solução.
        \item Resultados interessantes.
        \item Com sobreposição $\times$ sem sobreposição.
        \begin{itemize}
            \item Escalabilidade.
        \end{itemize}
    \end{itemize}
    \note{\scriptsize Bom, indo para as conclusões. Foram testados vários métodos de solução,
        todos baseados em \textit{bottom-left}, e ficou evidente que ordenar a lista de entrada de
        forma decrescente é vantajoso.
        Tivemos alguns resultados interessantes, como a competitividade entre todos os critérios de
        ordenação, sendo necessária uma investigação sobre características das instâncias.
        E também a pouca, ou nenhuma, vantagem em termos de qualidade quando usamos regiões que
        permitem sobreposições.
        De modo geral, pode-se resolver um problema com todas as combinações que usem regiões
        simples e buscar a de melhor solução, já que seu tempo de execução é pequeno.
        Resolver usando regiões com sobreposição só é recomendado em casos onde o modelo será usado
        várias vezes e não sofrerá alterações.
        Por fim, caso se queira aumentar a escala, seja na dimensão ou na quantidade
        de itens, compensa somente trabalhar com regiões sem sobreposição. No caso de aumentar
        a dimensão seu custo é baixo, sendo necessário verificar somente um parâmetro extra para
        cada item.}
\end{frame}
