\section*{Extras}\label{sec:extras}

\subsection{Tipos de peças}\label{subsec:tipos-de-pecas}
\begin{frame}[noframenumbering]
    \frametitle{Problema}
    \framesubtitle{Tipos de peças}
    \begin{figure}[!htb]
        \centering
        \caption{Exemplos de peças convexas (esquerda) e côncavas (direita).}
        \includegraphics[scale=0.6]{utils/images/pieces-example}
        \caption*{Fonte: \citeauthoryear{aprendizado-reforco}.}
        \label{fig:pieces}
    \end{figure}
    \note{
        \begin{itemize}
            \item Convexas: todos ângulos internos são menores que 180°.
            \item Côncavas: pelo menos um ângulo interno maior que 180°.
            \item Regulares: peças convexas com todos lados e ângulos iguais.
            \item Irregulares: pelo menos um lado ou ângulo diferente.
            \item Outra forma de se definir entre convexa ou não é checar se existe
            alguma reta que atravesse o objeto em dois pontos diferentes, se sim, é côncava.
            \item O trabalho foca em peças retangulares.
        \end{itemize}
    }
\end{frame}

\subsection{Classificação}\label{subsec:classificacao}
\begin{frame}[noframenumbering] % Necessário para usar environment dentro de \note
    \frametitle{Problema}
    \framesubtitle{Classificação}
    \begin{itemize}
        \item Empacotamento em faixa.
        \item Empacotamento da mochila.
        \item Empacotamento em caixas.
        \item Empacotamento ortogonal.
    \end{itemize}
    \note{
        \begin{itemize}
            \item Empacotamento em faixa: Dado um conjunto de itens e uma caixa
            com comprimento fixo, queremos encontrar uma solução de altura mínima.
            \item Empacotamento da mochila: Nesse caso, queremos maximizar o valor da caixa
            (geralmente é a área da caixa).
            \item Empacotamento em caixas: Minimizar o número de caixas necessárias para
            empacotar todos os itens.
            \item Empacotamento ortogonal: Alocar todos os itens numa caixa.
            \item Todas classificações do problema são NP-difícil, com exceção da ortogonal
            (NP-completo)\cite{2DPackLib}.
        \end{itemize}
    }
\end{frame}

\subsection{Variantes}\label{subsec:variantes}
\begin{frame}[noframenumbering] % Necessário para usar environment dentro de \note
    \frametitle{Problema}
    \framesubtitle{Variantes}
    \begin{itemize}
        \item Corte guilhotinado.
        \item Rotações ortogonais.
        \item Restrições de carga e descarga.
        \item Caixas de tamanho variável.
    \end{itemize}
    \note{
        Aqui vou citar algumas variantes do problema, mas nenhuma foi usada no trabalho.
        \begin{itemize}
            \item Corte guilhotinado: Consiste em cortar a caixa de forma paralela
            a um dos lados de forma recursiva.
            \item Rotações ortogonais: É um modo de relaxar o problema, permitindo rotações
            de 90° nos itens.
            \item Restrições de carga e descarga: Algumas peças precisam ser posicionadas
            em certa posição ou próximas a outras.
            \item Caixas de tamanho variável: Define que caixas não precisam ter o mesmo tamanho
            (aplicável somente para Empacotamento em Caixas).
        \end{itemize}
    }
\end{frame}

\begin{frame}[noframenumbering]
    \frametitle{Resultados}
    \framesubtitle{Melhores combinações de solução}
    \begin{table}
        \centering
        \caption{Resultados da comparação entre todas combinações.}
        \label{tab:combinacoes}
        \scriptsize\ttfamily
        \begin{tabular}{lllrrrrr}
            \hline
            Regiões     & Ordenação & Desc.\  & V  & E  & Qualidade \% & Tempo (s)  \\
            \hline
            Vertical    & Largura   & Sim    & 9  & 8  & 84.5497      & 2.4820e-03 \\
            Maior área  & Perímetro & Sim    & 7  & 6  & 85.8682      & 1.2944e-02 \\
            Sobrepostas & Área      & Sim    & 13 & 11 & 87.2957      & 6.4349e+00 \\
            Sobrepostas & Largura   & Sim    & 16 & 10 & 85.9266      & 8.4384e+00 \\
            \hline
        \end{tabular}
    \end{table}
    \note{Aqui eu mostro uma tabela com os resultados das combinações que se saíram melhores.
    Em termos de quantidade, quem se saiu melhor foi a criação de regiões com sobreposição, com
    a largura como critério de ordenação. Já em qualidade o melhor resultado foi obtido com
    sopreposição e ordenação por área. A maximização de regiões e ordenação por perímetro ficou
    bem próxima, ainda mais consirando o custo-benefício.
    A primeira linha da tabela mostra a combinação entre a criação de regiões na vertical e
    ordenação pela largura, esse resultado é bem interessante pois têm um dos menores tempos e ainda
    consegue ser competitivo tanto em qualidade quanto em quantidade.}
\end{frame}

\begin{frame}[noframenumbering]
    \frametitle{Resultados}
    \framesubtitle{Conjuntos de instâncias}
    \begin{table}
        \centering
        \caption{Resultados para os conjuntos de instância.}
        \label{tab:instances}
        \small\ttfamily\begin{tabular}{lrrr}
    \hline
    InstanceSet & Quality \% & Items \% & Time (s)    \\
    \hline
    BKW         & 62.9619    & 84.1578  & 7.28255     \\
    GCUT        & 64.3739    & 19.152   & 0.000270303 \\
    NGCUT       & 71.7568    & 47.3996  & 0.00038499  \\
    OF          & 72.1888    & 30.7337  & 0.000756912 \\
    OKP         & 77.1125    & 27.6778  & 0.00360035  \\
    \hline
\end{tabular}
    \end{table}
    \note{Nessa tabela eu trouxe os resultados separados de acordo com o conjunto de
    instância, a BKW demorou mais por ter os maiores números de itens a serem alocados dentre todas
    as instâncias. Em geral, temos bons resultados para cada conjunto, mas vale investigar se algum
    deles possui alguma característica que torne melhor determinado método de solução.}
\end{frame}
