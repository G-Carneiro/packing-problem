\section{Problema}\label{sec:problema}
\begin{frame}
    \frametitle{Problema}
    Alocar peças em um espaço.
    \begin{itemize}
        \item Difícil resolução.
        \item $N$-dimensional.
        \item Tipos de peças.
        \item Classificação.
        \item Variantes.
    \end{itemize}
    \note{
        A premissa do problema é simples, alocar peças em um espaço.
        Pode parecer algo bobo de resolver, mas é de difícil resolução já que pode possuir
        $N$-dimensões e diversos tipos de peças, de modo é preciso separar o problema em
        diferentes classes e ainda existem variantes dentro das classificações.
    }
\end{frame}

\subsection{N-dimensões}\label{subsec:n-dimensoes}
\begin{frame}
    \frametitle{$N$-dimensões}
    \begin{figure}[!htb]
        \centering
        \includegraphics[scale=0.6]{utils/images/packing-example}
        \caption{Represeção 1D, 2D e 3D.}
        \caption*{Fonte:\cite{castellucci2019consolidation}}
        \label{fig:packing}
    \end{figure}
    \note{
        \begin{itemize}
            \item O caso 1D pode ser usado para empilhar caixas de mesma profundidade e largura.
            \item Já no 2D poderia ser aplicado em casos onde somente a profundidade é fixa.
            \item E o 3D seria alocar caixas em um depósito ou container.
            \item O trabalho se concentra somente no caso 2D.
        \end{itemize}
    }
\end{frame}

\subsection{Restrições}\label{subsec:restricoes}
\begin{frame}
    \frametitle{Problema}
    \framesubtitle{Restrições}
    \scriptsize{
        \begin{align}
            x_i \in \{0, \dots, W - w_i\}, y_i \in \{0, \dots, H - h_i\} \left(i \in \mathcal{I}'\right) \label{eq:eq1} \\
            [x_i, x_i + w_i) \cap [x_j, x_j + w_j) = \emptyset \text{ ou } [y_i, y_i + h_i) \cap [y_j, y_j + h_j) = \emptyset \left(i, j \in \mathcal{I}', i \neq j\right) \label{eq:eq2}
        \end{align}
    }
    \note{
        Como já definimos a dimensão do problema, podemos ver as restrições do modelo.

        A primeira restrição garante que um item só é alocado no recipiente se couber nele.

        Já a segunda impede sobreposição entre as peças.
    }
\end{frame}

\subsection{Tipos de peças}\label{subsec:tipos-de-pecas}
\begin{frame}
    \frametitle{Tipos de peças}
    \begin{figure}[!htb]
        \centering
        \includegraphics[scale=0.6]{utils/images/pieces-example}
        \caption{Exemplos de peças regulares (esquerda) e irregulares (direita).}
        \caption*{Fonte:\cite{aprendizado-reforco}}
        \label{fig:pieces}
    \end{figure}
    \note{
        \begin{itemize}
            \item Regulares: Possuem formato convexo.
            \item Irregulares: Possuem formato côncavo.
            \item Outra forma de se definir é checar se existe
            alguma reta que atravesse o objeto em dois pontos diferentes.
            \item O trabalho foca em peças regulares retangulares.
        \end{itemize}
    }
\end{frame}

\subsection{Classificação}\label{subsec:classificacao}
\begin{frame}<-4> % Necessário para usar environment dentro de \note
    \frametitle{Classificação}
    \begin{itemize}
        \item<1-> Empacotamento em faixa.
        \item<2-> Empacotamento da mochila.
        \item<3-> Empacotamento em caixas.
        \item<4-> Empacotamento ortogonal.
    \end{itemize}
    \note{
        \begin{itemize}
            \item<1-> Dado um conjunto de itens e uma caixa com comprimento fixo,
            queremos encontrar uma solução de altura mínima.
            \item<2-> Nesse caso, queremos maximizar o valor da caixa
            (geralmente é a área da caixa).
            \item<3-> Minimizar o número de caixas necessárias para empacotar todos os itens.
            \item<4-> Alocar todos os itens numa caixa.
            \item<4-> Todos os problemas são NP-difícil, com exceção do ortogonal (NP-completo).
        \end{itemize}
    }
\end{frame}

\subsection{Variantes}\label{subsec:variantes}
\begin{frame}<-4> % Necessário para usar environment dentro de \note
    \frametitle{Variantes}
    \begin{itemize}
        \item<1-> Corte guilhotinado.
        \item<2-> Rotações ortogonais.
        \item<3-> Restrições de carga e descarga.
        \item<4-> Caixas de tamanho variável.
    \end{itemize}
    \note{
        \begin{itemize}
            \item<1-> Consiste em cortar a caixa de forma paralela
            a um dos lados de forma recursiva.
            \item<2-> É um modo de relaxar o problema, permitindo rotações de 90° nos itens.
            \item<3-> Algumas peças precisam ser posicionadas em certa posição
            ou próximas a outras.
            \item<4-> Define que caixas não precisam ter o mesmo tamanho
            (aplicável somente para Empacotamento em Caixas).
        \end{itemize}
    }
\end{frame}