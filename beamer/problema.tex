\section{Problema}\label{sec:problema}
\begin{frame}
    \frametitle{Problema}
    Alocar peças em um espaço.
    \begin{itemize}
        \item Difícil resolução.
        \item $N$-dimensional.
        \item Tipos de peças.
        \item Classificação.
        \item Variantes.
    \end{itemize}
    \note{
        A premissa do problema é simples, alocar peças em um espaço.
        Pode parecer algo bobo de resolver, mas é de difícil resolução já que pode possuir
        $N$-dimensões e diversos tipos de peças, de modo que é preciso separar o problema em
        diferentes classes e ainda existem variantes dentro das classificações.
    }
\end{frame}

\subsection{N-dimensões}\label{subsec:n-dimensoes}
\begin{frame}
    \frametitle{Problema}
    \framesubtitle{$N$-dimensões}
    \begin{figure}[!htb]
        \centering
        \caption{Represeção 1D, 2D e 3D.}
        \includegraphics[scale=0.6]{utils/images/packing-example}
        \caption*{Fonte: \citeauthoryear{castellucci2019consolidation}.}
        \label{fig:packing}
    \end{figure}
    \note{
        Como eu disse, o problema pode ter $N$-dimensões, aqui vou citar alguns exemplos.
        \begin{itemize}
            \item O caso 1D pode ser usado para empilhar caixas de mesma profundidade e largura.
            \item Já no 2D poderia ser aplicado em casos onde somente a profundidade é fixa.
            \item E o 3D seria alocar caixas em um depósito ou container.
            \item O trabalho se concentra somente no caso 2D.
        \end{itemize}
    }
\end{frame}

\subsection{Restrições}\label{subsec:restricoes}
\begin{frame}
    \frametitle{Problema}
    \framesubtitle{Restrições}
    \scriptsize{
        \begin{align}
            x_i \in \{0, \dots, W - w_i\}, y_i \in \{0, \dots, H - h_i\} \left(i \in \mathcal{I}'\right) \label{eq:1} \\
            [x_i, x_i + w_i) \cap [x_j, x_j + w_j) = \emptyset \text{ ou } [y_i, y_i + h_i) \cap [y_j, y_j + h_j) = \emptyset \left(i, j \in \mathcal{I}', i \neq j\right) \label{eq:2}
        \end{align}
    }
    \note{
        Como já definimos a dimensão do problema, podemos ver as restrições do modelo.

        A primeira restrição garante que um item só é alocado no recipiente se couber nele.

        Já a segunda impede sobreposição entre as peças.
    }
\end{frame}
