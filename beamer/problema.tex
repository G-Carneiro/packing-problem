\section{Problema}\label{sec:problema}

\begin{frame}
    \frametitle{Problema}
    \begin{figure}[!htb]
        \centering
        \caption{Represeção 1D, 2D e 3D.}
        \includegraphics[scale=0.6]{utils/images/packing-example}
        \caption*{Fonte: \citeauthoryear{castellucci2019consolidation}.}
        \label{fig:packing}
    \end{figure}
    \note{
        A premissa do problema é simples, alocar peças em um espaço.
        É algo simples, mas é de difícil resolução, podendo possuir
        $N$-dimensões e diversos tipos de peças, de modo que é preciso separar o problema em
        diferentes classes e ainda existem variantes dentro das classificações.
        Como exemplos:

        O caso 1D pode ser usado para cortar uma barra de ferro em certo tamanhos,
        de forma que seu desperdício seja mínimo.

        Já no 2D poderia ser aplicado em casos para alocação de paletes com altura fixa
        em um caminhão ou depósito.

        O caso 3D pode ser usado para alocar caixas em um depósito ou container,
        acredito que esse seja o mais simples de pensar em aplicações.
    }
\end{frame}

\subsection{Restrições}\label{subsec:restricoes}
\begin{frame}
    \frametitle{Problema}
    \framesubtitle{Restrições}
    \begin{itemize}
        \item Recipiente $\mathcal{B} = (W, H)$.
        \item Conjunto de itens $\mathcal{I}$.
        \item Conjunto solução $\mathcal{I'} \subseteq \mathcal{I}$.
    \end{itemize}
    \scriptsize{
        \begin{align}
            x_i \in \{0, \dots, W - w_i\}, y_i \in \{0, \dots, H - h_i\} \left(i \in \mathcal{I}'\right) \label{eq:1} \\
            [x_i, x_i + w_i) \cap [x_j, x_j + w_j) = \emptyset \text{ ou } [y_i, y_i + h_i) \cap [y_j, y_j + h_j) = \emptyset \left(i, j \in \mathcal{I}', i \neq j\right) \label{eq:2}
        \end{align}
    }
    \note{
        O trabalho lida somente com peças retangulares, então a dimensão de interesse é 2D.
        Para resolver o problema ele será representado como um modelo de otimização, portanto,
        são necessárias algumas restrições para o modelo.

        Então, dado um recipiente $\mathcal{B}$ de largura $W$ e altura $H$, um conjunto de itens
        $\mathcal{I}$ e um conjunto solução $\mathcal{I'}$.
        Tem-se duas restrições.

        A primeira restrição garante que um item só é alocado no recipiente se couber nele.

        Já a segunda impede sobreposição entre as peças.
    }
\end{frame}

\subsection{Empacotamento da mochila}\label{subsec:empacotamento-da-mochila}
\begin{frame}
    \frametitle{Problema}
    \framesubtitle{Empacotamento da mochila}
    Dado:
    \begin{itemize}
        \item Recipiente $\mathcal{B} = (W, H)$.
        \item Conjunto de itens $\mathcal{I}$.
        \item Cada item $i \in \mathcal{I}$ possui um valor $p_i$.
    \end{itemize}
    Encontrar:
    \begin{itemize}
        \item Conjunto solução $\mathcal{I'} \subseteq \mathcal{I}$.
    \end{itemize}
    De forma que:
    \[
        \max \sum_{i \in \mathcal{I'}} p_i
    \]
    \note{
        Como já mencionado, existem muitas versões do problema e a de interesse é o empacotamento
        da mochila.

        Nessa versão, dado um recipiente $\mathcal{B}$ e um conjunto de itens, onde cada item $i$
        possui um valor $p_i$, deve-se encontrar um conjunto solução que maximize o somatório dos
        valores dos itens alocados.

        No trabalho, o valor $p_i$ é a área do item $i$.
        Ou seja, o objetivo é maximizar a área ocupada do recipiente $\mathcal{B}$.
    }
\end{frame}
