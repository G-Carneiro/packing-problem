\section{Métodos de solução}\label{sec:bottom-left}
\begin{frame}
    \frametitle{Métodos de solução}
    \framesubtitle{\textit{Bottom-left}}
    \begin{figure}[!htb]
        \centering
        \caption{Representação de alocação.}
        \includegraphics[scale=0.8]{utils/images/bottom-left}
        \caption*{Fonte: \citeauthoryear{aprendizado-reforco}.}
        \label{fig:bottom-left}
    \end{figure}
    \note{
        Como o problema é NP-difícil, usar métodos exatos exigiria muitos recursos computacionais
        e/ou tempo de execução.
        Então, a heurística \textit{bottom-left} foi escolhida para solucionar o problema.

        Ela é bem simples, dado uma lista como entrada, os itens são retirados um a um
        e posicionados no ponto mais a baixo e mais a esquerda quanto for possível.

        Caso a peça não caiba em nenhuma posição ela não entra na solução e passa-se para
        a próxima da fila.

        Aqui fica claro que a sequência de alocação tem impacto direto na qualidade da solução.
        Mas como definir essa ordenação? Existe algum critério que seja melhor que os demais?
        Para responder essas perguntas, escolhi alguns critérios para ordenar a lista e comparar
        suas soluções.
    }
\end{frame}

\subsection{Critérios de ordenação}\label{subsec:criterios-de-ordenacao}
\begin{frame}
    \frametitle{Métodos de solução}
    \framesubtitle{Critérios de ordenação}
    \begin{itemize}
        \item Área.
        \item Perímetro.
        \item Largura.
        \item Altura.
        \item Id.
    \end{itemize}
    \note{
        A partir desse ponto começa de fato o desenvolvimento do trabalho.

        Onde 5 critérios de ordenação foram escolhidos: área, perímetro, largura, altura e id.
        Com cada critério podendo ser usado de forma crescente ou decrescente.

        A ordenação por id considera a ordem em que os itens foram colocados na lista (ou criados),
        ou seja, seria a forma padrão de resolver, sem ordenações.

        Agora que os critérios foram definidos, podemos passar para os próximos desafios do
        problema, que são a sobreposição e o domínio contínuo.
    }
\end{frame}

\subsection{Sobreposição e domínio contínuo}\label{subsec:sobreposicao-e-dominio-continuo}
\begin{frame}
    \frametitle{Métodos de solução}
    \framesubtitle{Sobreposição e domínio infinito}
    \only<1>{\begin{figure}[H]
    \centering
    \caption{Itens 0 e 1 posicionados.}
    \includegraphics[scale=0.5]{utils/images/continuous_example}
    \label{fig:sobreposicao-dominio}
    \fonte{feito pelo autor.}
\end{figure}
}
    \only<2>{\begin{figure}[H]
    \centering
    \includegraphics[scale=0.5]{utils/images/continuous_example2}
    \caption{Resolvendo sobreposição e domínio infinito.}
    \label{fig:sobreposicao-dominio2}
\end{figure}
}
    \only<3>{\begin{figure}[!htb]
    \centering
    \includegraphics[scale=0.5]{utils/images/continuous_example3}
    \caption{Resolvendo sobreposição e domínio infinito.}
    \label{fig:sobreposicao-dominio3}
\end{figure}
}
    \only<4>{\begin{figure}[H]
    \centering
    \includegraphics[scale=0.5]{utils/images/discrete_example}
    \caption{Itens 0 e 1 posicionados, com domínio discreto.}
    \label{fig:sobreposicao-dominio4}
\end{figure}
}
    \only<5>{\begin{figure}[H]
    \centering
    \caption{Itens 0, 1 e 2 posicionados, com domínio discreto.}
    \includegraphics[scale=0.5]{utils/images/discrete_example2}
    \label{fig:sobreposicao-dominio5}
    \fonte{feito pelo autor.}
\end{figure}
}
    \note<1>{Bom, supondo que estejamos em um estado do algoritmo como mostra a figura, onde o
    item 0 foi o primeiro alocado e o item 1 foi alocado a sua direita na posição (2, 0), porque
    não cabia logo acima na posição (0, 2) devido a \Cref{eq:1}.}
    \note<2>{Agora queremos alocar um terceiro item, de largura 3 e altura 1. Ao posicionar a peça
    na posição (0, 2) percebe-se que a \Cref{eq:1} é satisfeita, porém a \Cref{eq:2} não.}
    \note<3>{Nesse caso, com poucas peças alocadas e um auxílio visual, é fácil dizer
    que a posição (0, 4) é a primeira posição válida de acordo com a \textit{bottom-left}.
    Mas como chegar até ela? Existem infinitos pontos entre as coordenadas (0, 2) e (0, 4).}
    \note<4>{Como todas as instâncias de teste usadas tratam somente de peças e recipientes com
    valores inteiros uma abordagem possível seria discretizar o domínio.}
    \note<5>{Dessa forma somente coordenadas de valores inteiros precisariam ser checadas,
        resolvendo parcialmente o problema com o domínio, já que ainda temos muitos pontos para
        checar, principalmente em recipientes grandes. Mas isso não resolve a parte de sobreposição.
        Para cada ponto ainda é necessário verificar se existe sobreposição com cada uma
        das peças já alocadas, algo extremamente custoso. Além disso, a discretização não
        funcionaria tão bem em casos com valores não inteiros, prejudicando a aplicação
        em vários problemas do mundo real.}
\end{frame}

\subsection{Regiões}\label{subsec:regioes}
\begin{frame}
    \frametitle{Métodos de solução}
    \framesubtitle{Regiões}
    \only<1,6->{
        \begin{table}
            \centering
            \caption{Resumo das regiões.}
            \label{tab:modos-regioes}
            \small\ttfamily\begin{tabular}{clll}
    Modo & Divisão             & \Cref{eq:2} & Tipo      \\
    \hline
    1    & Vertical            & Trivial     & Simples   \\
    2    & Horizontal          & Trivial     & Simples   \\
    3    & Maior área          & Trivial     & Simples   \\
    4    & Regiões sobrepostas & Não trivial & Complexas \\
    \hline
\end{tabular}
        \end{table}
    }
    \only<2>{\begin{figure}[H]
    \centering
    \caption{Regiões criadas traçando uma linha vertical.}
    \includegraphics[scale=0.5]{output/figures/bkw/bkw01/vertically/height/false/01}
    \label{fig:regiao-vertical}
    \fonte{feito pelo autor.}
\end{figure}
}
    \only<3>{\begin{figure}[!htb]
    \centering
    \includegraphics[scale=0.5]{output/figures/bkw/bkw01/horizontally/height/false/01}
    \caption{Regiões criadas traçando uma linha horizontal.}
    \label{fig:regiao-horizontal}
\end{figure}
}
    \only<4>{\begin{figure}[H]
    \centering
    \includegraphics[scale=0.5]{output/figures/bkw/bkw01/max_area/height/false/01}
    \caption{Regiões criadas maximizando uma das regiões.}
    \label{fig:regiao-max}
\end{figure}
}
    \only<5>{\begin{figure}[!htb]
    \centering
    \includegraphics[scale=0.5]{output/figures/bkw/bkw01/none/height/false/01}
    \caption{Regiões criadas possibilitando sobreposições.}
    \label{fig:regiao-none}
\end{figure}
}

    \note<1>{
        Ambos desafios, de sobreposição e de domínio contínuo, podem ser
        resolvidos utilizando a estratégia de criação de regiões.
        Utilizando essa técnica a \Cref{eq:2} é trivialmente satisfeita.
        Nela, ao posicionar uma peça, duas regiões são criadas e o item seguinte somente será
        posicionado se couber em uma dessas regiões.

        O domínio passa a ser somente o canto inferior esquerdo de cada uma das regiões
        e sobreposições não são mais possíveis.
        Além disso, a regra para definir se uma peça cabe em dada região é igual a
        \Cref{eq:1}, tornando o algoritmo de solução bem simples.

        Eu escolhi criar as regiões de 4 formas diferentes, para identificar se isso teria algum
        impacto na solução.
        Teoricamente, resolver trivialmente a \Cref{eq:2}, pode causar prejuízos na qualidade de
        solução.
        Então, um dos modos não resolve trivialmente a \Cref{eq:2} e foi criado verificar seu
        custo-benefício.
    }
    \note<2>{A primeira forma de criar regiões é traçando uma linha vertical a partir do canto
    superior direito de cada peça alocada. Nas figuras, retângulos indicados com um R são regiões.}
    \note<3>{A segunda é semelhante a primeira, porém usando uma linha horizontal.}
    \note<4>{Já na terceira, é traçada uma linha (vertical ou horizontal) que maximize a
    área de uma das regiões geradas, basicamente identifica qual dos dois primeiros métodos
    gera a região de maior área e traça a reta correspondente. Isso é interessente pois dá uma
    garantia maior de que o item seguinte será alocado, em contrapartida pode gerar muitas regiões
    pequenas que podem não ser utilizadas, diminuindo a qualidade da solução.}
    \note<5>{No último modo nenhuma linha é traçada, todas as regiões vão até o final do recipiente,
        ficando sobrepostas.
        Nesse caso sobreposições de peças podem ocorrer, então verificações são necessárias para
        cumprir a \Cref{eq:2}.}
    \note<6>{Resumindo os modos das regiões, 3 geram regiões simples e resolvem trivialmente a
    \Cref{eq:2} e o último deveria apresentar melhores resultados na qualidade de solução, já que
    a \Cref{eq:2} é não-trivial.}
\end{frame}
