\section{\textit{Bottom-left}}\label{sec:bottom-left}
\begin{frame}
    \frametitle{\textit{Bottom-left}}
    \begin{figure}[!htb]
        \centering
        \includegraphics[scale=0.8]{utils/images/bottom-left}
        \caption{Representação de alocação.}
        \caption*{Fonte:\cite{aprendizado-reforco}}
        \label{fig:bottom-left}
    \end{figure}
    \note{
        Como o problema é NP-difícil uma heurística será usada e a \textit{bottom-left}
        foi a escolhida.

        Ela é bem simples, dado uma lista como entrada, os itens são retirados um a um
        e posicionados no ponto mais a baixo a mais a esquerda quanto for possível.

        Caso a peça não caiba em nenhuma posição ela não entra na solução e passa-se para
        a próxima da fila.

        Aqui fica claro que a sequência de alocação tem impacto direto na qualidade da solução
        e é um ponto a ser resolvido.
        Como definir essa ordenação? Existe algum critério que se sobressai dos demais?
    }
\end{frame}

\subsection{Critérios de ordenação}\label{subsec:criterios-de-ordenacao}
\begin{frame}
    \frametitle{Critérios de ordenação}
    \begin{itemize}
        \item Área.
        \item Perímetro.
        \item Largura.
        \item Altura.
        \item Id.
    \end{itemize}
    \note{
        5 critérios de ordenação foram escolhidos: área, perímetro, largura, altura e id.

        A ordenação por id considera a ordem em que os itens foram colocados na lista (ou criados),
        ou seja, seria a forma padrão de resolver.

        Cada critério pode ser usado de forma crescente ou decrescente.
        Com os critérios definidos, podemos passar para os próximos pontos do problema,
        que são a sobreposição e o domínio infinito.
    }
\end{frame}

\subsection{Sobreposição e domínio infinito}\label{subsec:sobreposicao-e-dominio-infinito}
\begin{frame}
    \frametitle{Sobreposição e domínio infinito}
    \framesubtitle{}
    \begin{figure}[!htb]
        \centering
        \includegraphics<1>[scale=0.5]{utils/images/continuous_example}
        \includegraphics<2>[scale=0.5]{utils/images/continuous_example2}
        \includegraphics<3>[scale=0.5]{utils/images/continuous_example3}
        \includegraphics<4>[scale=0.5]{utils/images/discrete_example}
        \includegraphics<5>[scale=0.5]{utils/images/discrete_example2}
        \caption{Resolvendo sobreposição e domínio infinito.}
        \label{fig:sobreposicao-dominio}
    \end{figure}
    \note<1>{Supondo que estejamos em um estado do modelo como mostra a figura, onde o item 0
    foi o primeiro alocado e o item 1 foi alocado a sua direita na posição (2, 0), porque
    não cabia logo acima na posição (0, 2) devido a restrição \autoref{eq:eq1}.}
    \note<2>{Agora queremos alocar um terceiro item de largura 3 e altura 1. Ao posicionar a peça
    na posição (0, 2) percebe-se que a restrição \autoref{eq:eq1} é satisfeita, porém a restrição
    \autoref{eq:eq2} não.}
    \note<3>{Nesse caso, com poucas peças, com caixa pequena e um auxílio visual é fácil dizer
    que a posição (0, 4) é válida, mas como chegar até ela? Existem infinitos pontos entre
    as coordenadas (0, 2) e (0, 4).}
    \note<4>{Como todas as instâncias tratam somente de peças e recipientes com valores inteiros
    uma abordagem possível seria discretizar o domínio.}
    \note<5>{Dessa forma somente coordenadas de valores inteiros precisariam ser checadas,
        resolvendo parcialmente o problema com o domínio, já que ainda temos muitos pontos para
        checar, principalmente em instâncias grandes. Mas isso não resolve a parte de sobreposição.
        Para cada ponto ainda é necessário verificar se existe sobreposição com cada uma
        das peças já alocadas, algo extremamente custoso. Além disso, a discretização só
        funcionaria em casos como os das intâncias, com valores inteiros, não sendo aplicável em
        vários problemas do mundo real.}
\end{frame}

\subsection{Regiões}\label{subsec:regioes}
\begin{frame}
    \frametitle{Regiões}
    \begin{itemize}
        \item Vertical.
        \item Horizontal.
        \item $\max$(área).
        \item Nenhuma.
    \end{itemize}
    \note{
        Ambos os problemas, de sobreposição e de domínio infinito, podem ser
        resolvidos utilizando a estratégia de regiões.

        As regiões são criadas de 4 formas diferentes, traçando uma linha vertical,
        uma horizontal, traçando uma linha (vertical ou horizontal) que maximize a área de uma das
        regiões geradas e com nenhuma linha.

        Nesse último modo sobreposições de peças podem ocorrer, então verificações
        são necessárias para cumprir a restrição.
    }
\end{frame}

\subsection{Testes}\label{subsec:testes}
\begin{frame}
    \frametitle{Testes}
    \begin{itemize}
        \item 45 instâncias.
        \begin{itemize}
            \item BKW\@.
            \item GCUT\@.
            \item NGCUT\@.
            \item OF\@.
            \item OKP\@.
        \end{itemize}
        \item 5 testes por configuração.
        \item $45 \cdot 5 \cdot 2 \cdot 4 \cdot 5 = 9000$ execuções.
        \item ±5 horas.
    \end{itemize}
    \note{
        Para testar os métodos de solução criados foram usados 5 conjuntos de instâncias:
        BKW, GCUT, NGCUT, OF e OKP, totalizando 45 instâncias de teste.

        Cada método foi executado 5 vezes em cada uma das instâncias para se obter um média,
        também foi calculado a mediana e desvio padrão.

        Como temos 45 instâncias, 5 critérios de ordenação, cada critério pode ser crescente ou
        decrescente, 4 formas de criar regiões e cada uma dessas combinações foi executada 5
        vezes, temos o total de 9000 execuções.

        O tempo somado de todas as execuções foi de aproximadamente 5 horas (valor que ainda
        será alterado, pois falta rodar a maior instância com o método de solução mais demorado).
    }
\end{frame}