\section{\textit{Bottom-left}}\label{sec:bottom-left}
\begin{frame}
    \frametitle{\textit{Bottom-left}}
    \begin{figure}[!htb]
        \centering
        \includegraphics[scale=0.8]{utils/images/bottom-left}
        \caption{Representação de alocação.}
        \caption*{Fonte:\cite{aprendizado-reforco}}
        \label{fig:bottom-left}
    \end{figure}
    \note{
        Como o problema é NP-difícil uma heurística será usada e a \textit{bottom-left}
        foi a escolhida.

        Ela é bem simples, dado uma lista como entrada, os itens são retirados um a um
        e posicionados no ponto mais a baixo a mais a esquerda quanto for possível.

        Caso a peça não caiba em nenhuma posição ela não entra na solução e passa-se para
        a próxima da fila.

        Aqui fica claro que a sequência de alocação tem impacto direto na qualidade da solução.
        Então como definir essa ordenação? Existe algum critério que se sobressai dos demais?
    }
\end{frame}

\subsection{Critérios de ordenação}\label{subsec:criterios-de-ordenacao}
\begin{frame}
    \frametitle{Critérios de ordenação}
    \begin{itemize}
        \item Área.
        \item Perímetro.
        \item Largura.
        \item Altura.
        \item Id.
    \end{itemize}
    \note{
        5 critérios de ordenação foram escolhidos: área, perímetro, largura, altura e id.

        A ordenação por id considera a ordem em que os itens foram colocados na lista (ou criados),
        ou seja, seria a forma padrão de resolver.

        Cada critério pode ser usado de forma crescente ou decrescente.
    }
\end{frame}

\subsection{Regiões}\label{subsec:regioes}
\begin{frame}
    \frametitle{Regiões}
    \begin{itemize}
        \item Vertical.
        \item Horizontal.
        \item $\max$(área).
        \item Nenhuma.
    \end{itemize}
    \note{
        As regiões são criadas de 4 formas diferentes, traçando uma linha vertical,
        uma horizontal, traçando uma linha (vertical ou horizontal) que maximize a área de uma das
        regiões geradas e com nenhuma linha.

        Nesse último modo sobreposições de peças podem ocorrer, então verificações
        são necessárias para cumprir a restrição.
    }
\end{frame}

\subsection{Testes}\label{subsec:testes}
\begin{frame}
    \frametitle{Testes}
    \begin{itemize}
        \item 45 instâncias.
        \begin{itemize}
            \item BKW\@.
            \item GCUT\@.
            \item NGCUT\@.
            \item OF\@.
            \item OKP\@.
        \end{itemize}
        \item 5 testes por configuração.
        \item $45 \cdot 5 \cdot 2 \cdot 4 \cdot 5 = 9000$ execuções.
        \item ±5 horas.
    \end{itemize}
    \note{
        Para testar os métodos de solução criados foram usados 5 conjuntos de instâncias:
        BKW, GCUT, NGCUT, OF e OKP, totalizando 45 instâncias de teste.

        Cada método foi executado 5 vezes em cada uma das instâncias para se obter um média,
        também foi calculado a mediana e desvio padrão.

        Como temos 45 instâncias, 5 critérios de ordenação, cada critério pode ser crescente ou
        decrescente, 4 formas de criar regiões e cada uma dessas combinações foi executada 5
        vezes, temos o total de 9000 execuções.

        O tempo somado de todas as execuções foi de aproximadamente 5 horas (valor que ainda
        será alterado, pois falta rodar a maior instância com o método de solução mais demorado).
    }
\end{frame}