\section{Resultados}\label{sec:resultados}

\subsection{Comparativo - Ordenação}\label{subsec:configuracoes-ruins}
\begin{frame}
    \frametitle{Comparativo}
    \framesubtitle{Ordenação}

    \only<1>{
        \begin{table}
            \centering
            \caption{Comparativo entre ordenação crescente e decrescente.}
            \label{tab:descending}
            \begin{tabular}{lrrrr}
    \hline
    Decrescente & Vitórias & Empates & Qualidade \% & Tempo (s)  \\
    \hline
    Sim         & 736      & 8       & 78.9136      & 1.7798e+00 \\
    Não         & 167      & 8       & 57.3060      & 2.3715e+00 \\
    \hline
\end{tabular}
        \end{table}
    }
    \only<2->{
        \begin{figure}
            \centering
            \includegraphics<2>[scale=0.5]{output/figures/bkw/bkw01/horizontally/height/false/00}
            \includegraphics<3>[scale=0.5]{output/figures/bkw/bkw01/horizontally/height/false/01}
            \includegraphics<4>[scale=0.5]{output/figures/bkw/bkw01/horizontally/height/false/02}
            \includegraphics<5>[scale=0.5]{output/figures/bkw/bkw01/horizontally/height/false/06}
            \caption{Regiões criadas na ordenação crescente.}
            \label{fig:false}
        \end{figure}
    }
    \note<1>{A primeira coisa que fica evidente com os resultados é discrepância
    na qualidade de solução entre a ordenação crescente e a decrescente, algo já esperado.}
    \note<2>{Isso se deve a como as regiões são criadas, as figuras mostram o caso para
    ordenação crescente com a altura como critério e linha horizontal para criar a região.}
    \note<3>{Ao posicionar uma peça uma das regiões ficará com a mesma altura do item
    recém-posicionado, como a ordenação é crescente a próxima peça terá no mínimo
    a mesma altura, mas o provável é que seja mais alta, impossibilitando que seja
    alocada nessa região.}
    \note<4>{Fazendo com que muitas regiões fiquem sem poder receber peças.}
    \note<5>{Essa figura mostra o estado final do modelo e grande parte do espaço
    ainda está livre. Algo semelhante ocorre com outros critérios de ordenação e criação regiões.}
\end{frame}

\begin{frame}
    \frametitle{Comparativo}
    \framesubtitle{Regiões}
    \only<1>{
        \begin{table}
            \centering
            \caption{Comparativo entre criação de regiões.}
            \label{tab:regioes}
            \begin{tabular}{lrrrrr}
    \hline
    SplitMode & Wons & Draws & Quality \% & Items \% & Time (s)   \\
    \hline
    V         & 155  & 102   & 65.0025    & 45.6288  & 0.00243109 \\
    H         & 122  & 101   & 63.0163    & 44.1025  & 0.00723233 \\
    M         & 189  & 158   & 69.5409    & 48.7214  & 0.0133133  \\
    N         & 334  & 195   & 73.3659    & 48.5347  & 8.23366    \\
    \hline
\end{tabular}
        \end{table}
    }
    \only<2>{
        \begin{itemize}
            \item Sem sobreposição: $R = O\left(\dfrac{n^2 + n}{2}\right)$.
            \item Com sobreposição: $\displaystyle R = O\left(\dfrac{n^2 + n}{2}\right),
            S = O\left(\sum_{i=1}^{n} i(i - 1)\right)$.
            $n = 3152 \to S = \numprint{10438481552}.$
        \end{itemize}
    }

    \note<1>{Indo para o comparativo entre regiões percebemos que a que permite sobreposições se saiu
    melhor, tanto em quantidade como em qualidade, porém com um custo autíssimo de tempo.
    Regiões criadas com linhas verticais e horizontais foram mais rápidas, mas com soluções de pior
    qualidade. Enquanto maximizando as regiões criadas levou um pouco a mais de tempo, mas também
    com acréscimo na qualidade. Ter sobreposição demora em torno de 1000 vezes mais.
    Mas por que tanta diferença entre com e sem sobreposição?}
    \note<2>{Como dito antes, sem sobreposições temos somente que checar se
    um item cabe em uma região, no pior caso teremos que fazer isso para $(n^2 + n) / 2$ regiões.
    Enquanto com sobreposição, além de ter esse número de regiões, para cada uma delas
    também é necessário checar possíveis sobreposições com as peças já alocadas, sendo o número
    de verificações igual o somatório de $i(i - 1)$, isso no pior caso, algo extremamente custoso.
    Por exemplo, para uma instância com 3152 itens podem ser necessárias mais de 10 bilhões
    de verificações de sobreposição.
    Então, aquela diferença de 1000 vezes fica ainda maior de acordo com a quantidade de itens
    a serem alocados.}
\end{frame}