\section{Resultados}\label{sec:resultados}

\subsection{Comparativo - Ordenação}\label{subsec:configuracoes-ruins}
\begin{frame}
    \frametitle{Comparativo}
    \framesubtitle{Ordenação}

    \only<1>{
        \begin{table}
            \centering
            \caption{Comparativo entre ordenação crescente e decrescente.}
            \label{tab:descending}
            \begin{tabular}{lrrrr}
    \hline
    Decrescente & Vitórias & Empates & Qualidade \% & Tempo (s)  \\
    \hline
    Sim         & 736      & 8       & 78.9136      & 1.7798e+00 \\
    Não         & 167      & 8       & 57.3060      & 2.3715e+00 \\
    \hline
\end{tabular}
        \end{table}
    }
    \only<2->{
        \begin{figure}
            \centering
            \includegraphics<2>[scale=0.5]{output/figures/bkw/bkw01/horizontally/height/false/00}
            \includegraphics<3>[scale=0.5]{output/figures/bkw/bkw01/horizontally/height/false/01}
            \includegraphics<4>[scale=0.5]{output/figures/bkw/bkw01/horizontally/height/false/02}
            \includegraphics<5>[scale=0.5]{output/figures/bkw/bkw01/horizontally/height/false/06}
            \caption{Regiões criadas na ordenação crescente.}
            \label{fig:false}
        \end{figure}
    }
    \note<1>{A primeira coisa que fica evidente com os resultados é discrepância
    na qualidade de solução entre a ordenação crescente e a decrescente, algo já esperado.}
    \note<2>{Isso se deve a como as regiões são criadas, as figuras mostram o caso para
    ordenação crescente com a altura como critério e linha horizontal para criar a região.}
    \note<3>{Ao posicionar uma peça uma das regiões ficará com a mesma altura do item
    recém-posicionado, como a ordenação é crescente a próxima peça terá no mínimo
    a mesma altura, mas o provável é que seja mais alta, impossibilitando que seja
    alocada nessa região.}
    \note<4>{Fazendo com que muitas regiões fiquem sem poder receber peças.}
    \note<5>{Algo semelhante ocorre com outros critérios de ordenação e criação regiões.}
\end{frame}