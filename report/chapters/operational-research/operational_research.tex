\chapter{Pesquisa Operacional}\label{ch:pesquisa-operacional}


Pesquisa Operacional pode ser entendida como o estudo e a aplicação de métodos científicos para tomada de decisões em problemas complexos.
Ela permite modelar, analisar e solucionar tais problemas de modo, geralmente, satisfatório.


\section{Conceitos Básicos}\label{sec:conceitos-basicos}

\subsection{Modelos de Otimização}\label{subsec:modelos-de-otimizacao}

Modelos são aproximações de realidade, representam o problema de maneira simples e objetiva, usando restrições.
Eles são o que baseiam a Pesquisa Operacional.
De forma geral, um modelo de otimização quer minimizar ou maximizar uma função $f(x)$ com $x$ obedecendo algumas restrições.
Pode-se então representar o modelo do seguinte modo:

\[
    \min f(x), x \in \mathcal{X}.
\]

Onde

\begin{itemize}
    \item $x$: variável de decisão, $x = x_1, x_2, \dots, x_n$.
    \item $\mathcal{X}$: conjunto factível ou domínio, possui todas as soluções possíveis para o problema.
    \item $f(x)$: função objetivo, a qual determinará o critério de escolha da solução.
\end{itemize}

\subsection{Solução Factível}\label{subsec:solucao-factivel}

Uma solução $x'$ é factível somente se satisfaz todas as restrições dados ao problema, ou seja, $x' \in \mathcal{X}$.

\subsection{Problema Infactível}\label{subsec:solucao-infactivel}

Existem casos onde o problema não tem solução, possivelmente por muitas restrições terem sido aplicadas.
Isso é chamado problema infactível e $\mathcal{X} = \emptyset$.

\subsection{Problema Ilimitado}\label{subsec:problema-ilimitado}

Se para toda solução for possível encontrar outra melhor o problema é ilimitado.

\subsection{Solução Ótima}\label{subsec:solucao-otima}

Uma solução $x'$ é ótima somente se for factível e possuir resultado melhor que as demais soluções, isto é, $f(x') \le f(x), \forall x \in \mathcal{X}$ (caso seja um problema de maximização é necessário substituir “$\le$” por “$\ge$”).
Importante observar, que existe somente solução ótima se o problema não for infactível nem ilimitado.

\subsection{Tipos de Modelo}\label{subsec:tipos-de-modelo}
% TODO: adicionar figuras

É importante saber diferenciar os modelos devido ao método de resolução que varia para cada um deles.

\subsubsection{Modelo Linear x Não-linear}\label{subsubsec:modelo-linear}


Modelos lineares possuem como função objetivo uma função linear e todas as restrições também são lineares.
Exemplos:

\begin{itemize}
    \item $f(x) = ax + b$.
    \item $f(x_1, x_2) = x_1 + x_2 - 5$.
\end{itemize}

Já os não-lineares não obedecem essa regra, podendo ter suas variáveis se multiplicando ou funções trigonométricas e logarítmicas.
Exemplos:

\begin{itemize}
    \item $f(x_1, x_2) = x_1^2 + x_2^2$.
    \item $f(x_1, x_2) = \tan(x_1 + x_2)$.
\end{itemize}

\subsubsection{Modelo Contínuo x Discreto}\label{subsubsec:modelo-continuo-x-discreto}

Um modelo é contínuo quando sua região factível é contínua, ou seja, dado um ponto dessa região todos os seus vizinhos também serão uma solução.
Modelos discretos não possuem seu domínio contínuo.

\subsubsection{Modelo Determinístico x Estocástico}\label{subsubsec:modelo-deterministico-x-estocastico}

Em modelos determinísticos seus dados são conhecidos, enquanto os estocásticos possuem uma incerteza quanto aos dados.

\subsubsection{Tipos de Programação}\label{subsubsec:tipos-de-programacao}

Com base nas categorias de modelo é possível também dividir métodos de programação (planejamento) para sua solução.

\begin{itemize}
    \item Linear: modelo linear contínuo determinístico.
    \item Inteira: modelo linear discreto determinístico.
    \item Estocástica: modelo linear contínuo estocástico.
    \item Não-linear: modelo não-linear contínuo determinístico.
\end{itemize}

\subsection{Métodos Exatos x Heurísticas}\label{subsec:metodos-exatos-x-heuristicas}

Métodos exatos sempre vão garantir a solução ótima para o problema, porém encontrar tal solução pode requerer grande tempo e/ou muitos recursos computacionais.
Já heurísticas buscam por soluções factíveis e são geralmente usadas em problemas de grande porte.
