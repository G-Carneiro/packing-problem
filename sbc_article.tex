\documentclass[12pt]{article}

\usepackage{setup/sbc-template}
\RequirePackage[T1]{fontenc}                        % Pacote de fontes
\usepackage{graphicx}
\usepackage{url}
\usepackage{float}
\usepackage[utf8]{inputenc}
\usepackage[brazil]{babel}
\usepackage{hyperref}
\usepackage{amsmath}
\usepackage{amsfonts}
\usepackage[brazilian, nameinlink, capitalize]{cleveref}

\Crefname{equation}{Restrição}{Restrições}
\creflabelformat{equation}{#2#1#3}
%\labelformat{equation}{Restrição~#1}
\renewcommand{\equationautorefname}{Restrição}
\newcommand{\citeauthoryear}[1]{~\cite{#1}}
\ifdefined\fonte
\else
\newcommand{\fonte}[1]{}
\fi

\ifdefined\IBGEtab
\else
\newcommand{\IBGEtab}[1]{#1}
\fi


\sloppy

\title{Problema de empacotamento de retângulos: \\
avaliação de métodos de solução baseados em \textit{bottom-left}}

\author{Gabriel Medeiros Lopes Carneiro\inst{1}, Pedro Belin Castellucci\inst{1},
    Rafael de Santiago\inst{1} }


\address{
    Departamento de Informática e Estatística\\
    Universidade Federal de Santa Catarina (UFSC) -- Florianólis, SC -- Brasil
    \email{gabriel.mlc@grad.ufsc.br, pedro.castellucci@ufsc.br, r.santiago@ufsc.br}
}

\begin{document}

    \maketitle

    \begin{resumo}
    Problemas de empacotamento consistem em alocar um conjunto de itens $\mathcal{I}$ em uma
    caixa $\mathcal{B}$.
    No problema de empacotamento da mochila, foco deste trabalho, cada item é associado a um valor
    e busca-se uma solução que maximize a soma dos valores dos itens alocados.
    Este trabalho compara 40 métodos de solução criados com base na heurística
    \textit{bottom-left} para o problema de empacotamento de retângulos.
    Os métodos criados são uma combinação de diferentes formas de ordenação dos itens e criação
    de regiões, as quais evitam as sobreposições e o domínio contínuo presentes no problema.
    O principal resultado foi a alta competitividade de diferentes modos de ordenação,
    não sendo a área a única relevante, com o perímetro obtendo os melhores resultados.
\end{resumo}

% --------------------------------------------
% Caso seja necessário um resumo em inglês,
% descomentar linhas abaixo.
% --------------------------------------------
\begin{abstract}
    Packing problems consist of allocating a set of items $\mathcal{I}$ into a box $\mathcal{B}$.
    In the knapsack packing problem, the focus of this work, each item is associated with a value
    and a solution is sought that maximizes the sum of the values of the allocated items.
    This work compare 40 created solution methods based on \textit{bottom-left}
    heuristic for the rectangle packing problem.
    The methods created are a combination of different ways of ordering items and creating regions,
    which avoid superposition and continuous domain present in the problem.
    The main result was the high competitiveness of different ordering modes, the area not being
    the only relevant one, with the perimeter obtaining the best results.
\end{abstract}

    \section{Introdução}\label{sec:introducao}

Serviços de loja \textit{online} com entrega como Amazon e Mercado Livre estão tornando-se cada vez
mais presentes no dia a dia.
Para tornar as entregas mais rápidas é necessário fazer uma série de estudos de logística e
planejamento sobre como organizar os produtos nos estoques e nos veículos de entrega~\cite{
    silva2022integer,morabito1992abordagem}, muitas vezes sendo necessário considerar a ordem
em que eles precisarão ser retirados.
Além de tornar o processo mais rápido, a organização também pode permitir o melhor uso de espaços,
aumentando a quantidade máxima de itens ou evitando o desperdício dos espaços.

Ainda sobre evitar desperdícios, esse quesito é muito importante para as indústrias de papel,
móveis, têxtil e metal-mecânica~\cite{queiroz2022estudo,cavali2004problemas,belluzzo2005otimizacao}.
Todas essas áreas querem gerar o máximo de produtos com o mínimo de recursos materiais utilizados,
para evitar o descarte desnecessário do material e prejuízos financeiros.

Os problemas citados são considerados problemas de corte e empacotamento.
Problemas de corte envolvem cortar um objeto, como blocos de gesso, chapas de aço e barras de ferro,
em itens menores.
Enquanto problemas de empacotamento tratam sobre alocar um conjunto de itens $\mathcal{I}$ em um
recipiente $\mathcal{B}$.
Ambos são equivalentes entre si e é possível separá-los de acordo com sua dimensão.

O caso 2D, dimensão de estudo deste trabalho, possui uma vasta literatura de métodos de solução.
As abordagens de solução se dividem entre exatas, que buscam a solução ótima do problema, e
heurísticas, as quais podem não encontrar uma solução ótima, mas conseguem uma solução aceitável em
tempo hábil.
Dentre os métodos de solução exatos, um que se destaca é o procedimento de busca em árvore~\cite{
    beasley1985exact}, mas existem muitos outros na literatura~\cite{exact-solution-techniques,
    fekete1997new,delorme2016bin,kenmochi2009exact}.
Na parte de heurísticas, tem-se a \textit{bottom-left}~\cite{baker1980orthogonal,chehrazad2022fast}
e \textit{skyline}~\cite{wei2011skyline}, as heurísticas também possuem grande presença na
literatura~\cite{burke2004new,rakotonirainy2020improved,hopper2001empirical,chen2019efficient,
    huang2007efficient,hopper2001review}.

Este trabalho visa criar métodos de solução para o problema de empacotamento no espaço de duas
dimensões, onde as peças
são retangulares e com um recipiente também retangular, mais especificamente na versão do
empacotamento 2D da mochila, considerado NP-difícil~\cite{2DPackLib}.
Nessa versão do problema, dado um conjunto de itens $\mathcal{I}$, com cada item $i$ possuindo um
valor $p_i$, e uma caixa $\mathcal{B}$, o objetivo é maximizar a soma dos valores dos itens alocados
dentro do recipiente.
A abordagem escolhida para resolver o problema foi utilizar a heurística \textit{bottom-left},
devido a sua simplicidade e aos limites computacionais e de tempo ao escolher algum método exato.
Mesmo com a heurística sendo proposta em 1980, ela ainda está presente
na literatura recente~\cite{chehrazad2022fast,hopper2001empirical,wei2011skyline}.

O principal objetivo deste trabalho é criar métodos de solução para o problema de
empacotamento da mochila de peças retangulares, todos baseados na heurística \textit{bottom-left}.
Outros objetivos mais específicos são: implementar a \textit{bottom-left} e os métodos derivados em
Python, executá-los com instâncias de teste da literatura, comparar seus resultados e identificar
vantagens e desvantagens de cada um.

    \section{Definição}\label{sec:definicao}

De acordo com~\cite{2DPackLib}, dado uma caixa retangular $\mathcal{B} = (W, H)$ de comprimento $W \in \mathbb{Z}_+$ e altura $H \in \mathbb{Z}_+$ e um conjunto $\mathcal{I}$ de itens também retangulares, onde cada item $i \in \mathcal{I}$ com comprimento $w_i \in \mathbb{Z}_+, w_i \le W$ e altura $h_i \in \mathbb{Z}_+, h_i \le H$.
Um empacotamento $\mathcal{I}' \subseteq \mathcal{I}$ em $\mathcal{B}$ pode ser descrito como uma função $\mathcal{F}: \mathcal{I}' \to \mathbb{Z}_+^2$ que mapeie cada item $i \in \mathcal{I}'$ para um par de coordenadas $\mathcal{F}(i) = (x_i, y_i)$, de forma

\begin{restrictions}
    x_i \in \{0, \dots, W - w_i\}, y_i \in \{0, \dots, H - h_i\} \left(i \in \mathcal{I}'\right) \label[restriction]{eq:1} \\
    [x_i, x_i + w_i) \cap [x_j, x_j + w_j) = \emptyset \text{ ou } [y_i, y_i + h_i) \cap [y_j, y_j + h_j) = \emptyset \left(i, j \in \mathcal{I}', i \neq j\right) \label[restriction]{eq:2}
\end{restrictions}


Nessa forma de representação a caixa está posicionada no plano cartesiano, com seu canto inferior esquerdo na origem.
Já as coordenadas $\mathcal{F}(i) = (x_i, y_i)$ representam a posição em que o canto inferior esquerdo da peça será alocado.
A \Cref{eq:1} garante que cada item deve estar inteiramente na caixa, enquanto a \Cref{eq:2} impede sobreposição entre peças.
Ambas restrições indicam uma orientação fixa, ou seja, peças não podem ser rotacionadas.

    \section{Métodos de solução}\label{sec:metodos-de-solucao}

Como descrito na \autoref{sec:introducao}, a maioria das classes do problema são NP-difíceis.
Isso torna métodos de soluções exatos, os quais buscam pela solução ótima, extremamente custosos
em tempo e recursos computacionais em instâncias de porte moderado, muitas vezes sendo inviáveis
por falta de algum desses dois motivos.
Consequentemente a literatura é dominada por abordagens que usam heurísticas e meta-heurísticas,
sendo a \textit{bottom-left} uma das principais estratégias de solução e será usada no estudo
deste trabalho.

A \textit{bottom-left} é uma heurística construtiva proposta por\citeauthoryear{baker1980orthogonal}.
Embora tenha sido proposta a décadas, ainda é bastante usada na literatura atual, além de poder
ser usada como componente de algoritmos mais sofisticados e para diferentes classes e variantes
do problema.
Ela foi utilizada nos trabalhos de \citeauthoryear{hopper2001empirical} na comparação de vários
métodos de solução, \citeauthoryear{wei2011skyline} trazendo uma revisão do método e seus derivados
e, mais recentemente, \citeauthoryear{chehrazad2022fast} através de uma adaptação para o
empacotamento de itens irregulares de forma gulosa.

Sua premissa é simples, dado uma fila de itens como entrada, enquanto ela não estiver vazia,
basta retirar o primeiro item dela e alocar no canto mais a baixo e à esquerda quanto for
possível~\cite{aprendizado-reforco}, sem sobreposições entre peças.
Caso não exista uma posição válida, a peça é desconsiderada e passa-se para próxima da fila.
A \autoref{fig:bottom-left} mostra um exemplo de alocação para um dado conjunto de peças regulares.

\begin{figure}[H]
    \centering
    \includegraphics{utils/images/bottom-left}
    \caption{Representação de alocação usando \textit{bottom-left}.}
    \label{fig:bottom-left}
    \fonte{\cite{aprendizado-reforco}}
\end{figure}


Vale destacar que a própria ordem da fila pode gerar resultados diferentes, alterando a qualidade
da solução.
Um dos resultados esperados deste trabalho é identificar se há alguma forma de ordenação que
se destaque na qualidade de solução, através da comparação entre os diferentes modos.
Para isso, serão usados conjuntos de instâncias frequentemente utilizados na literatura.


\subsection{Critérios de ordenação}\label{subsec:criterios-de-ordenacao}

Para determinar o impacto da ordenação da fila, cinco critérios de ordenação
foram escolhidos, sendo eles: área, perímetro, largura, altura e \textit{id}.
A ordenação por \textit{id} considera a ordem em que os itens foram colocados na lista, ou seja,
seria a forma padrão de resolver e ele será a base para definir se os demais critérios possuem
algum benefício.
Além disso, cada critério pode ser usado para ordenar a fila em ordem crescente ou decrescente,
algo que também será analisado.
Na literatura o mais comum é utilizar a ordenação decrescente pela área~\cite{chen2019efficient}.

\section{Criação de regiões}\label{sec:criacao-de-regioes}
% TODO: talvez dar nomes para soluções com e sem região

Os dois problemas expostos na \cref{sec:sobreposicao-e-dominio-infinito} podem ser facilmente
resolvidos utilizando a estratégia de criação de regiões.
Com essa técnica é possível ignorar a \Cref{eq:2}.
Nela, ao posicionar uma peça, duas regiões são criadas (\Cref{fig:regiao-vertical}) e o item
seguinte será somente posicionado se couber em uma das regiões disponíveis.

\begin{figure}[H]
    \centering
    \caption{Regiões criadas traçando uma linha vertical.}
    \includegraphics[scale=0.5]{output/figures/bkw/bkw01/vertically/height/false/01}
    \label{fig:regiao-vertical}
    \fonte{feito pelo autor.}
\end{figure}


Agora o domínio passa a ser somente o canto inferior esquerdo de cada uma das regiões e
sobreposições deixam de ser possíveis.
Além disso, a regra para definir se uma peça cabe em dada região é igual a \Cref{eq:1}, tornando
o algoritmo de solução bem simples.
A fim de identificar o impacto das regiões na solução do modelo, quatro formas de criação
delas foram usadas.

A primeira delas é \textbf{traçando uma linha vertical} a partir do canto superior direito de cada
peça alocada (\Cref{fig:regiao-vertical}).
A segunda é igual a primeira, porém \textbf{traçando uma linha horizontal} (\Cref{fig:regiao-horizontal}).

\begin{figure}[!htb]
    \centering
    \includegraphics[scale=0.5]{output/figures/bkw/bkw01/horizontally/height/false/01}
    \caption{Regiões criadas traçando uma linha horizontal.}
    \label{fig:regiao-horizontal}
\end{figure}


Já na terceira, a linha traçada (vertical ou horizontal) depende da área das regiões criadas
com cada linha.
Nesse modo o objetivo é maximizar a área de uma das regiões geradas, ele identifica qual linha
irá gerar a região de maior área e a traça.
Por exemplo, a \Cref{fig:regiao-vertical} gerou uma região com 252 de área e outra com 1320,
enquanto a \Cref{fig:regiao-horizontal} obteve regiões com 1440 e 132, então, nesse caso, a linha
traçada será a horizontal (\Cref{fig:regiao-max}).

\begin{figure}[H]
    \centering
    \includegraphics[scale=0.5]{output/figures/bkw/bkw01/max_area/height/false/01}
    \caption{Regiões criadas maximizando uma das regiões.}
    \label{fig:regiao-max}
\end{figure}


Maximizar uma região pode ser interessante, pois aumenta as chances do próximo item conseguir ser
alocado, visto que uma das regiões será mais espaçosa.
Em contrapartida, esse método também pode acabar gerando muitas regiões pequenas que não sejam
utilizadas, diminuindo a qualidade da solução.

No quarto e último modo de criar regiões nenhuma linha é traçada, todas as regiões vão até o final
do recipiente (\Cref{fig:regiao-none}).
Nesse caso, sobreposições de peças podem ocorrer, então verificações são necessária para cumprir
a \Cref{eq:2}.
Ao fazer isso, possibilita que mais peças sejam alocadas, visto que todas as regiões possuem área
máxima.
Esse modo foi criado para identificar se é de fato melhor que os demais e qual seu custo.

\begin{figure}[!htb]
    \centering
    \includegraphics[scale=0.5]{output/figures/bkw/bkw01/none/height/false/01}
    \caption{Regiões criadas possibilitando sobreposições.}
    \label{fig:regiao-none}
\end{figure}


Com os critérios para criação de regiões explicados, é possível diferenciá-los em dois tipos.
O primeiro é dos que permitem sobreposição entre peças e, por isso, precisam de verificações para
respeitar a \Cref{eq:2}, nesse tipo se encaixa somente o quarto modo.
O segundo tipo contém os três primeiros critérios, onde somente a \Cref{eq:1} precisa ser checada.

    \section{Resultados}\label{sec:resultados}

Para testar os métodos de solução criados foram usadas 45 instâncias de teste da literatura,
separadas em cinco conjuntos de instância de características diferentes:
BKW, GCUT, NGCUT, OF e OKP\@.
Todas as elas foram obtidas através da biblioteca pública
\href{https://site.unibo.it/operations-research/en/research/2dpacklib}{2DPackLib}\footnote{
    Disponível em: https://site.unibo.it/operations-research/en/research/2dpacklib.
    Acessado em: \today.}~\cite{2DPackLib}.

O foco do trabalho é no empacotamento 2D da mochila, com o critério de maximização sendo a
área ocupada do espaço.
Mas nem todos conjuntos foram feitos para ser resolvidos dessa forma, nesses casos foram
feitas leves adaptações para usá-los.
O motivo de usar instâncias feitas com outro objetivo é para não viciar o modelo em instâncias
específicas.

Como são cinco critérios de ordenação~(\cref{subsec:criterios-de-ordenacao}), com cada critério
podendo ser crescente ou decrescente, quatro formas de criar regiões~(\cref{subsec:criacao-de-regioes})
e 45 instâncias, tem-se o total de 1800 casos de teste.
Além disso, para conseguir resultados mais fiéis, a média, mediana e desvio padrão do tempo de
execução foram calculados.
Por isso, cada caso foi executado cinco vezes, totalizando 9000 execuções.
Outros dados como a qualidade de solução (objetivo do trabalho), porcentagem de itens alocados
e tempo, também foram computados.
Todas as execuções foram feitas em um mesmo computador, com configurações conforme a
\Cref{tab:config}.

\begin{table}
    \centering
    \caption{Configuração do computador de testes.}
    \label{tab:config}
    \begin{tabular}{ll}
        \hline
        CPU    & AMD Ryzen™ 5 3600X       \\
        RAM    & 16 GiB                   \\
        Python & 3.11.0                   \\
        SO     & Linux Mint 21.1 Cinnamon \\
        Kernel & 5.15                     \\
        \hline
    \end{tabular}
\end{table}


Ao analisar a média, a mediana e o desvio padrão do tempo de execução, observou-se que a média
e mediana possuem valores quase idênticos, enquanto o desvio padrão é pequeno ao ponto de poder
ser ignorado, indicando que cinco execuções por caso de teste são suficientes.
Portanto, a mediana e desvio padrão serão omitidos no restante do trabalho, podendo ser encontrados
na versão completa dos dados gerados no \href{https://github.com/G-Carneiro/packing-problem/}{
    Github}\footnote{Disponível em: https://github.com/G-Carneiro/packing-problem/. Acessado em: \today.}.

Nas tabelas apresentadas nas seções seguintes, as colunas “Vitórias” e “Empates” trazem os
resultados de forma quantitativa, enquanto a coluna “Qualidade \%” de modo qualitativo.
Nos testes, o valor $p_i$ dos itens sempre é sua área, desconsiderando os valores dados pelas
instâncias, caso hajam.

A qualidade de solução é determinada pela área ocupada do recipiente, no caso de todos os itens
serem alocados, a qualidade é $100\%$ independente da área do recipiente.
A coluna “Vitórias” indica quantas vezes tal método de solução obteve o melhor resultado em
comparação com os demais métodos em outras linhas.
Enquanto a coluna “Empates” mostra a quantidade de vezes que o método conseguiu a melhor qualidade,
mas outros também conseguiram.
Essas colunas foram feitas da seguinte forma: entre cada combinação de critério de ordenação,
modo de criar regiões e instâncias, é feita a comparação se a qualidade de solução foi melhor
para ordenação crescente ou decrescente.
No caso de ambas conseguirem o melhor resultado, é acrescido 1 tanto na coluna “Vitórias”, quanto
na “Empates” de ambas.
Por fim, a coluna “Tempo (s)” mostra o tempo médio de execução do método em segundos.

\subsection{Ordenação crescente × decrescente}\label{subsec:ordenacao-crescente-decrescente}

Uma primeira observação é a discrepância na qualidade de solução entre a ordenação crescente
e a decrescente, algo já esperado.
Na \Cref{tab:ordenacao} é possível notar que ordenando de forma decrescente é possível ocupar
cerca de $20\%$ a mais do espaço (coluna “Qualidade \%”), quando comparado a ordenação crescente,
em média.

\begin{table}[!htb]
    \centering
    \caption{Resultado da comparação entre Descending.}
    \label{tab:descending}
    \IBGEtab{}{
        \begin{tabular}{lrrrrr}
            \hline
            Descending & Wons & Draws & Quality \% & Items \% & Time (s) \\
            \hline
            F          & 167  & 8     & 57.306     & 47.6518  & 2.37153  \\
            T          & 736  & 8     & 78.9136    & 46.3642  & 1.77985  \\
            \hline
        \end{tabular}
    }{}
    \fonte{autor}
\end{table}

Com isso, fica claro que ordenar a fila de entrada da \textit{bottom-left} de modo decrescente é
vantajoso em termos de qualidade, quantidade e tempo de execução.

\subsection{Comparativo entre critérios de ordenação}\label{subsec:comparativo-entre-criterios-de-ordenacao}

A \Cref{tab:ordenacoes-true} mostra o comparativo entre os critérios de criação de regiões,
somente considerando a ordenação decrescente, já que não existem motivos para usar a crescente.
Representando os dados dessa forma fica fácil identificar que utilizar algum critério de ordenação
para a fila de entrada é vantajoso, pois ao usar o \textit{id} os resultados foram os piores.
Além disso, percebe-se que as ordenações por área e perímetro obtiveram os melhores resultados,
ainda que os demais também sejam competitivos.

\begin{table}[!htb]
    \centering
    \caption{Resultado da comparação entre critérios de ordenação decrescente.}
    \label{tab:ordenacoes-true}
    \IBGEtab{}{
        \ttfamily\input{utils/tabular/compare/ordenacao_true}
    }{}
    \fonte{feito pelo autor.}
\end{table}

A alta competitividade entre os critérios de ordenação é interessante, pois a maioria dos trabalhos
na literatura, como o de \citeauthoryear{chen2019efficient}, usam somente ordenação pela área, e
isso pode ser um forte indicativo que os demais critérios devem ser mais explorados em certas
circunstâncias.

\subsection{Comparativo entre criação de regiões}\label{subsec:comparativo-entre-criacao-de-regioes}

Conforme mostra a \autoref{tab:regioes-true}, os modos criados traçando uma linha vertical ou
horizontal apresentaram qualidades semelhantes e
os menores tempos de execução, mas o método o qual traça uma linha vertical obteve mais vitórias.
Regiões criadas para maximizar uma das mesmas conseguiram o segundo melhor resultado qualitativo e
quantitativo, ao custo de um pequeno acréscimo no tempo de execução em relação aos dois primeiros.
O último modo de fato conseguiu os melhores resultados, porém a um custo altíssimo, levando cerca
de 1000 vezes mais tempo que métodos mais rápidos.

\begin{table}[!htb]
    \centering
    \caption{Resultado da comparação entre criação de regiões.}
    \label{tab:regioes-true}
    \IBGEtab{}{
        \ttfamily\input{utils/tabular/compare/regioes_true}
    }{}
    \fonte{autor}
\end{table}

A \Cref{tab:superposition} traz um comparativo entre os dois tipos de criação de regiões,
os que é preciso checar sobreposição e os que não (\cref{subsec:criacao-de-regioes}).
Na primeira linha da tabela a coluna “Qualidade \%” representa a média do melhor resultado obtido
em cada instância e a coluna “Tempo Total (s)” mostra a soma dos tempos que cada método de solução
levou para cada instância.
As duas colunas consideram somente métodos de solução que usam regiões onde não são necessárias
verificações de sobreposição, ou seja, são considerados 30 modos de solução.
A segunda linha considera somente método de solução o qual utiliza ordenação decrescente pela área
e criação de regiões onde é necessário verificar sobreposições, esse método foi escolhido para o
comparativo por apresentar os melhores resultados quantitativos e ser o segundo melhor
qualitativamente.

\begin{table}[!htb]
    \centering
    \caption{Resultado da comparação entre tipos de regiões.}
    \label{tab:superposition}
    \IBGEtab{}{
        \ttfamily\input{utils/tabular/compare/superposition}
    }{}
    \fonte{autor}
\end{table}

Com a \Cref{tab:superposition} fica claro que, por mais que usar regiões onde é preciso checar
sobreposições apresente o melhor resultado, é melhor executar todos demais métodos os quais não
precisem e escolher somente o de melhor solução, pois assim é possível obter, na média, soluções
de maior qualidade e ainda levando 10 vezes menos tempo.

    \section{Conclusão}\label{sec:conclusao}
\addcontentsline{toc}{chapter}{\texorpdfstring{CONCLUSÃO E TRABALHOS FUTUROS}{Conclusão e trabalhos futuros}}

No trabalho foram avaliados experimentalmente 40 métodos de solução baseados na heurística
\textit{bottom-left} para o problema de empacotamento de retângulos, considerando a versão da
mochila e com o valor dos itens sua própria área.
O resultado mais óbvio obtido foi a supremacia da ordenação decrescente sobre a crescente
(\cref{subsec:ordenacao-crescente-decrescente}), não compensando utilizar a segunda para solucionar o
problema.

O alto uso da ordenação decrescente pela área na literatura~\cite{chen2019efficient} foi justificado
pelos resultados, já que essa composição conseguiu alguns dos melhores números
(\cref{subsec:comparativo-entre-criterios-de-ordenacao}).
Os resultados também indicam que qualquer critério de ordenação obtêm melhores resultados do que
deixar a fila desordenada (ordenação por \textit{id}).
Mas a alta competitividade de outros critérios de ordenação, esses não tão comuns na literatura,
conseguindo melhores resultados em algumas instâncias, mostra que ainda há espaço para outras
formas de ordenação.

Em relação às diferentes regiões criadas, por mais que regiões complexas tenham obtido melhores
resultados (\cref{subsec:comparativo-entre-criacao-de-regioes}), não compensa utilizá-las.
Ao executar todos os métodos de regiões simples é possível conseguir melhores resultados e em menor
tempo, quando comparado ao método de ordenação decrescente pela área e usando regiões complexas
(ver a \Cref{tab:superposition} na \cref{subsec:comparativo-entre-criacao-de-regioes}).


    \bibliographystyle{setup/sbc}
    \bibliography{aftertext/references}

\end{document}
