% --------------------------------------------
% Definição do tipo do documento e suas
% características.
% --------------------------------------------
\documentclass[
    12pt,                   % Tamanho da fonte.
    a4paper,                % Tipo de papel.
    sumario=tradicional,    % Tipo de sumário.
    brazil,                 % Linguagem principal do documento.
    oneside                 % Imprimir documento em um lado da folha.
]{abntex2}                  % Estilo do documento.

% --------------------------------------------
% Importação de configurações pessoais.
% --------------------------------------------
\usepackage{setup/packages}


\input{setup/author}
%! Author = gabriel
%! Date = 5/17/21

% --------------------------------------------
% Insira o título do trabalho.
% --------------------------------------------
\title{Problemas de Empacotamento}
% --------------------------------------------
% Caso o trabalho tenha subtítulo, descomente
% a linha abaixo.
% OBS.: NÃO APAGAR ":~" irá desconfigurar o arquivo.
% --------------------------------------------
\subtitle{um comparativo entre métodos de solução}

% --------------------------------------------
% Insira o tipo de trabalho do documento.
% --------------------------------------------
\worktype{Trabalho de conclusão de curso}

% --------------------------------------------
% Insira o local de apresentação do documento.
% --------------------------------------------
\local{Florianópolis, SC}

% --------------------------------------------
% Define a data do documento. Por padrão mostra
% apenas o ano, caso queira a data completa,
% substitua por \today.
% --------------------------------------------
\date{2023}

% --------------------------------------------
% Caso o trabalho possua um orientador,
% comente a linha abaixo.
% --------------------------------------------
%\orientador[Professor:]{Nome completo do professor}

% --------------------------------------------
% Caso o documento possua um orientador,
% descomente a linha abaixo.
% --------------------------------------------
\orientador{Prof. Dr. Pedro Belin Castellucci}

% --------------------------------------------
% Caso o documento possua um coorientador,
% descomente a linha abaixo.
% --------------------------------------------
\coorientador{Prof. Dr. Rafael de Santiago}

% --------------------------------------------
% Substituir '[mestre/doutor] em título obtido'
% pelo grau adequado.
% --------------------------------------------
\formation{Bacharel em Ciências da Computação}

% --------------------------------------------
% Substituir nome do curso pelo nome do curso.
% --------------------------------------------
\program{curso de Ciências da Computação}

% --------------------------------------------
% Caso precise do preâmbulo do documento,
% descomente as linhas abaixo. Ele deve conter,
% o tipo do documento, o objetivo, o nome da
% instituição e a área de concentração.
% --------------------------------------------
\preambulo{
    \printworktype\ submetido ao
    \printprogram\ da \printuniversity\
    para a obtenção do título de \printformation.
}

% --------------------------------------------
% Definição de cores de hyperlinks
% e formatações do pdf.
% --------------------------------------------
\hypersetup{
    colorlinks=true,
    linkcolor=black,
    filecolor=magenta,
    urlcolor=blue,
    citecolor=black,
    pdfauthor=\theauthor,
    pdftitle=\thetitle,
    pdfsubject=\imprimirpreambulo,
    pdfkeywords={problema de empacotamento, \textit{bottom-left}, heurística, pesquisa operacional.},
    bookmarksopen=true,
    breaklinks=true,
}

% --------------------------------------------
% Início do documento.
% --------------------------------------------
% Aqui devem ser inseridas todas informações.
% --------------------------------------------
\begin{document}
    % --------------------------------------------
    % Inserção dos elementos pré-textuais.
    % --------------------------------------------
    % Capa, folha de rosto, ficha catalográfica,
    % errata, folha de aprovação, dedicatória,
    % agradecimentos, epígrafe, resumos,
    % lista de ilustrações, lista de tabelas,
    % lista de abreviaturas e siglas,
    % lista de símbolos e sumário.
    % --------------------------------------------
    %! Author = gabriel
%! Date = 5/17/21

% Logo da UFSC
%\begin{figure}
%    \includegraphics[scale=0.2]{logo-ufsc}
%    \centering
%    \label{fig:logo-ufsc}
%\end{figure}

% Geração de Título

% --------------------------------------------
% Para usar título padrão latex, descomente
% a linha abaixo.
% --------------------------------------------
%\maketitle

% TODO: melhorar capa


% --------------------------------------------
% Geração da capa. (obrigatório)
% --------------------------------------------
\printcoverufsc
%\printcover

% --------------------------------------------
% Geração da folha de rosto. (obrigatório)
% --------------------------------------------
\printtitlepage

% --------------------------------------------
% Geração da ficha catalográfica.(obrigatório)
% --------------------------------------------
% TODO:
%\begin{fichacatalografica}
%\end{fichacatalografica}

% --------------------------------------------
% Errata. (opcional)
% --------------------------------------------
%\begin{errata}
%\end{errata}

% --------------------------------------------
% Folha de aprovação. (obrigatório)
% --------------------------------------------
% TODO
%\includepdf{}

% --------------------------------------------
% Dedicatória. (opcional)
% --------------------------------------------
%\begin{dedicatoria}
%\end{dedicatoria}

% --------------------------------------------
% Agradecimentos. (opcional)
% --------------------------------------------
%\begin{agradecimentos}
%\end{agradecimentos}

% --------------------------------------------
% Epígrafe. (opcional)
% --------------------------------------------
%\begin{epigrafe}
%\end{epigrafe}

% --------------------------------------------
% Resumo/Abstract. (obrigatório)
% --------------------------------------------
% --------------------------------------------
% Resumo do documento de acordo com padrões
% Latex. Caso queira de acordo com a ABNT,
% comente as linhas abaixo.
% --------------------------------------------
%\begin{abstract}
%    Write here.
%\end{abstract}

% --------------------------------------------
% Comentar linhas abaixo para não ter um
% resumo de acordo com a ABNT.
% --------------------------------------------
\begin{resumo}
    Problemas de empacotamento consistem em alocar um conjunto de itens $\mathcal{I}$ em uma
    caixa $\mathcal{B}$.
    No problema de empacotamento da mochila, foco deste trabalho, cada item é associado a um valor
    e busca-se uma solução que maximize a soma dos valores dos itens alocados.
    Este trabalho compara 40 métodos de solução criados com base na heurística construtiva
    \textit{bottom-left} para o problema de empacotamento de retângulos.
    A escolha dessa heurística se deve a sua simplicidade e a dificuldade de usar métodos exatos
    para resolução do problema em tempo hábil.
    Os métodos criados são uma combinação de diferentes formas de ordenação dos itens e criação
    de regiões, as quais evitam as sobreposições e o domínio contínuo presentes no problema.
    Algoritmos foram implementados em Python e testados com instâncias da literatura, dados como
    qualidade de solução, porcentagem de itens alocados e tempo de execução foram coletados.
    O principal resultado foi a alta competitividade de diferentes modos de ordenação,
    não sendo a área a única relevante, com o perímetro obtendo os melhores resultados.

    \textbf{Palavras-chave: problema de empacotamento, \textit{bottom-left}, heurística,
        pesquisa operacional.}
\end{resumo}
\newpage

% --------------------------------------------
% Caso seja necessário um resumo em inglês,
% descomentar linhas abaixo.
% --------------------------------------------
\begin{resumo}[Abstract]
    Packing problems consist of allocating a set of items $\mathcal{I}$ into a box $\mathcal{B}$.
    In the knapsack packing problem, the focus of this work, each item is associated with a value
    and a solution is sought that maximizes the sum of the values of the allocated items.
    This work compare 40 created solution methods based on \textit{bottom-left} constructive
    heuristic for the rectangle packing problem.
    The choice of this heuristic is due to its simplicity and the difficulty of using exact methods
    to solve the problem in a timely manner.
    The methods created are a combination of different ways of ordering items and creating regions,
    which avoid superposition and continuous domain present in the problem.
    Algorithms were implemented in Python and tested with instances from the literature,
    data such as solution quality, percentage of allocated items and execution time were collected.
    The main result was the high competitiveness of different ordering modes, the area not being
    the only relevant one, with the perimeter obtaining the best results.

    \textbf{Keywords: packing problem, \textit{bottom-left}, heuristic, operational research.}
\end{resumo}
\newpage


% --------------------------------------------
% Lista de Figuras. (opcional)
% --------------------------------------------
\pdfbookmark[0]{\listfigurename}{lof}
\listoffigures*
\cleardoublepage

% --------------------------------------------
% Lista de Tabelas. (opcional)
% --------------------------------------------
\pdfbookmark[0]{\listtablename}{lot}
\listoftables*
\cleardoublepage

% --------------------------------------------
% Lista de Siglas. (opcional)
% --------------------------------------------
%\begin{siglas}
%\end{siglas}

% --------------------------------------------
% Lista de Símbolos. (opcional)
% --------------------------------------------
%\begin{simbolos}
%\end{simbolos}

% --------------------------------------------
% Geração do Sumário. (obrigatório)
% --------------------------------------------
\pdfbookmark[0]{\contentsname}{toc}
\tableofcontents*
\cleardoublepage

    % --------------------------------------------
    % Inserção dos elementos textuais.
    % --------------------------------------------
    % Aqui devem ser inseridos todos os capítulos
    % e/ou seções do trabalho.
    % --------------------------------------------
    \chapter{Introdução}\label{ch:introducao}
    % --------------------------------------------
% Aqui você deve organizar as seções.
% --------------------------------------------
\chapter*[Introdução]{Introdução}\label{ch:introducao}
\addcontentsline{toc}{chapter}{Introdução}

Este trabalho visa estudar o Problema de Empacotamento de peças retangulares em uma caixa também retangular no espaço de duas dimensões, sendo sua solução considerada NP-difícil \cite{2DPackLib}.
Antes de abordar o problema (\autoref{ch:problema-de-empacotamento}) e buscar soluções alguns conceitos básicos são mostrados na \autoref{ch:conceitos-basicos}.

O \autoref{ch:conceitos-basicos} foca em definições sobre otimização (\autoref{sec:definicoes}) e em modelos de otimização (\autoref{sec:modelos-de-otimizacao} e \autoref{sec:tipos-de-modelo}).
No \autoref{ch:problema-de-empacotamento} é dada a definição do problema (\autoref{sec:definicao}), para então mostrar algumas classificações (\autoref{sec:classificacao}) e variantes (\autoref{sec:variantes}), por fim é explicada a heurística \textit{bottom-left} (\autoref{sec:bottom-left}), a qual será utilizada na resolução das instâncias de teste.

O problema tem várias a aplicações nas indústrias de móveis, têxtil e metal-mecânica \cite{queiroz2022estudo, cavali2004problemas, belluzzo2005otimizacao}, além ser extramente útil em carregamento de paletes e contêineres \cite{morabito1992abordagem}.
É possível dividir o problema de acordo com sua dimensão.

Problemas unidimensionais podem ser associados ao corte de barras ou canos, para atender uma demanda por peças de diferentes tamanhos.
As indústrias de tecido ou couro usam o caso 2D para minimizar o desperdício ao se cortar suas peças.
O caso 3D é fácilmente associável ao carregamento de contêineres, onde objetos são geralmente caixas a serem alocadas em algum veículo.
A \autoref{fig:packing-example} mostra um exemplo para cada dimensão do problema.

\begin{figure}[!htb]
    \centering
    \caption{Representação para o problema de empacotamento 1D, 2D e 3D.}
    \includegraphics[scale=1]{utils/images/packing-example}
    \label{fig:packing-example}
    \fonte{\citeauthoryear{castellucci2019consolidation}.}
\end{figure}


Basicamente, pode-se aplicá-lo em qualquer área que precise de organização ou logística, bem como situações que envolvam o corte de algum material.
Ao utilizar soluções para resolver problemas de empacotamento, é possível reduzir o desperdício de materiais e impacto ambiental, diminuir tempo de entregas e otimizar espaços de estoque.

% --------------------------------------------
% Aqui você deve organizar as subsubseções.
% --------------------------------------------
\section*{Objetivos}\label{sec:Objetivos}

O principal objetivo deste trabalho é estudar e compreender o problema de empacotamento bem como suas aplicações no mundo real.
Outros objetivos mais específicos são: revisar a bibliografia, implementar heurísticas baseadas em \textit{bottom-left}, definir instâncias de teste e analisar os dados obtidos e compará-los com os de outros autores.



    % --------------------------------------------
    % Definição do estilo da bibliografia.
    % Deve estar comentado para padrão ABNT.
    % --------------------------------------------
    %\bibliographystyle{unsrt}

    % --------------------------------------------
    % Definição do arquivo da bibliografia.
    % --------------------------------------------
    \bibliography{aftertext/references}

    % --------------------------------------------
    % Inserção dos elementos pós-textuais.
    % --------------------------------------------
    % Glossário, apêndices, anexos e índice.
    % --------------------------------------------
    % --------------------------------------------
% Organização dos elementos pós-textuais.
% --------------------------------------------
\postextual
% --------------------------------------------
% Definição do estilo da bibliografia.
% Deve estar comentado para padrão ABNT.
% --------------------------------------------
%\bibliographystyle{unsrt}

% --------------------------------------------
% Definição do arquivo da bibliografia.
% --------------------------------------------
\addcontentsline{toc}{chapter}{REFERÊNCIAS}
\printbibliography

% --------------------------------------------
% Glossário. (opcional)
% --------------------------------------------
% TODO: use glossaries package

% --------------------------------------------
% Apêndice. (opcional)
% --------------------------------------------
\apendices


\chapter{Resultados das instâncias}\label{ch:resultados-das-instancias}
\chapter{Resultados das instâncias}\label{ch:resultados-das-instancias}


\section{BKW}\label{sec:bkw}
%
%As \Crefrange{tab:bkw01}{tab:bkw13} mostram os resultados dos métodos de solução propostos para
%as instâncias BKW\@.

\begin{table}[H]
    \centering
    \caption{Resultados da instância BKW01.}
    \label{tab:bkw01}
    \IBGEtab{}{
        \ttfamily\input{utils/tabular/bkw/bkw01}
    }{}
    \fonte{feito pelo autor.}
\end{table}
\begin{table}[!htb]
    \centering
    \caption{Resultados da instância BKW02.}
    \label{tab:bkw02}
    \IBGEtab{}{
        \begin{tabular}{llllrrr}
            \hline
            Instance & Split & Order & Descending & Quality \% & Time (s)    & Items \% \\
            \hline
            BKW02    & V     & A     & T          & 61         & 0.000150204 & 70       \\
            BKW02    & V     & A     & F          & 29.1333    & 0.000172043 & 60       \\
            BKW02    & V     & P     & T          & 63.4       & 0.000176191 & 75       \\
            BKW02    & V     & P     & F          & 29.1333    & 0.000126314 & 60       \\
            BKW02    & V     & H     & T          & 57.2667    & 0.000154543 & 65       \\
            BKW02    & V     & H     & F          & 34.2       & 0.00014596  & 65       \\
            BKW02    & V     & W     & T          & 81.0667    & 0.000174713 & 90       \\
            BKW02    & V     & W     & F          & 26.7333    & 0.000114441 & 55       \\
            BKW02    & V     & I     & T          & 40.7333    & 0.000146866 & 65       \\
            BKW02    & V     & I     & F          & 57.4667    & 0.000182724 & 60       \\
            BKW02    & H     & A     & T          & 82.6667    & 0.000167894 & 65       \\
            BKW02    & H     & A     & F          & 40.1333    & 0.000172758 & 70       \\
            BKW02    & H     & P     & T          & 82.6667    & 0.000169897 & 65       \\
            BKW02    & H     & P     & F          & 49.3333    & 0.000175238 & 75       \\
            BKW02    & H     & H     & T          & 82.5333    & 0.00019927  & 80       \\
            BKW02    & H     & H     & F          & 47.4667    & 0.000169945 & 70       \\
            BKW02    & H     & W     & T          & 73.4       & 0.000184059 & 85       \\
            BKW02    & H     & W     & F          & 60.2       & 0.000254059 & 85       \\
            BKW02    & H     & I     & T          & 50.4       & 0.000168943 & 75       \\
            BKW02    & H     & I     & F          & 78.5333    & 0.000185442 & 80       \\
            BKW02    & M     & A     & T          & 71.9333    & 0.000417852 & 85       \\
            BKW02    & M     & A     & F          & 43.8667    & 0.000322151 & 75       \\
            BKW02    & M     & P     & T          & 71.9333    & 0.00037384  & 85       \\
            BKW02    & M     & P     & F          & 60.2       & 0.000353718 & 85       \\
            BKW02    & M     & H     & T          & 68.2       & 0.000350428 & 80       \\
            BKW02    & M     & H     & F          & 49.3333    & 0.000337648 & 75       \\
            BKW02    & M     & W     & T          & 79.4667    & 0.000356483 & 90       \\
            BKW02    & M     & W     & F          & 55.2667    & 0.000311279 & 75       \\
            BKW02    & M     & I     & T          & 54.1333    & 0.000328588 & 80       \\
            BKW02    & M     & I     & F          & 76.8       & 0.000387478 & 85       \\
            BKW02    & N     & A     & T          & 85.8667    & 0.00555429  & 70       \\
            BKW02    & N     & A     & F          & 43.5333    & 0.0119086   & 75       \\
            BKW02    & N     & P     & T          & 89.8667    & 0.00630503  & 80       \\
            BKW02    & N     & P     & F          & 49.9333    & 0.0107538   & 80       \\
            BKW02    & N     & H     & T          & 82         & 0.00732751  & 85       \\
            BKW02    & N     & H     & F          & 59.1333    & 0.0150655   & 85       \\
            BKW02    & N     & W     & T          & 96.5333    & 0.0157199   & 90       \\
            BKW02    & N     & W     & F          & 65.5333    & 0.0182476   & 80       \\
            BKW02    & N     & I     & T          & 64.3333    & 0.0130246   & 85       \\
            BKW02    & N     & I     & F          & 76.8       & 0.00792232  & 85       \\
            \hline
        \end{tabular}
    }{}
    \fonte{autor}
\end{table}
\begin{table}[!htb]
    \centering
    \caption{Resultados da instância BKW03.}
    \label{tab:bkw03}
    \IBGEtab{}{
        \ttfamily\input{utils/tabular/bkw/bkw03}
    }{}
    \fonte{feito pelo autor.}
\end{table}
\begin{table}[!htb]
    \centering
    \caption{Resultados da instância BKW04.}
    \label{tab:bkw04}
    \IBGEtab{}{
        \begin{tabular}{llllrrr}
            \hline
            Instance & Split & Order & Descending & Quality \% & Time (s)    & Items \% \\
            \hline
            BKW04    & V     & A     & T          & 91.0781    & 0.000413847 & 85       \\
            BKW04    & V     & A     & F          & 46.2812    & 0.000360441 & 92.5     \\
            BKW04    & V     & P     & T          & 93.4062    & 0.00041132  & 90       \\
            BKW04    & V     & P     & F          & 44.4688    & 0.000353956 & 90       \\
            BKW04    & V     & H     & T          & 88.4375    & 0.000381422 & 80       \\
            BKW04    & V     & H     & F          & 48.2812    & 0.000416327 & 95       \\
            BKW04    & V     & W     & T          & 94.5       & 0.000481224 & 87.5     \\
            BKW04    & V     & W     & F          & 42.2812    & 0.000336885 & 87.5     \\
            BKW04    & V     & I     & T          & 46.2812    & 0.000394678 & 92.5     \\
            BKW04    & V     & I     & F          & 81.5938    & 0.000416422 & 85       \\
            BKW04    & H     & A     & T          & 87.3125    & 0.000555563 & 85       \\
            BKW04    & H     & A     & F          & 18.6875    & 0.000481176 & 70       \\
            BKW04    & H     & P     & T          & 85.7812    & 0.000363636 & 77.5     \\
            BKW04    & H     & P     & F          & 17.5625    & 0.000398636 & 65       \\
            BKW04    & H     & H     & T          & 74.6094    & 0.000433826 & 90       \\
            BKW04    & H     & H     & F          & 17.5625    & 0.000416327 & 65       \\
            BKW04    & H     & W     & T          & 75.6406    & 0.00042057  & 85       \\
            BKW04    & H     & W     & F          & 18.6875    & 0.000400496 & 70       \\
            BKW04    & H     & I     & T          & 33.8594    & 0.000436449 & 85       \\
            BKW04    & H     & I     & F          & 62.7656    & 0.000449085 & 85       \\
            BKW04    & M     & A     & T          & 86.5       & 0.000859547 & 90       \\
            BKW04    & M     & A     & F          & 38.5938    & 0.00074091  & 90       \\
            BKW04    & M     & P     & T          & 88.125     & 0.000876093 & 92.5     \\
            BKW04    & M     & P     & F          & 38.5938    & 0.00076685  & 90       \\
            BKW04    & M     & H     & T          & 92.7969    & 0.000802946 & 90       \\
            BKW04    & M     & H     & F          & 49.3125    & 0.000779629 & 90       \\
            BKW04    & M     & W     & T          & 90.3125    & 0.000989199 & 95       \\
            BKW04    & M     & W     & F          & 48.2812    & 0.000853062 & 95       \\
            BKW04    & M     & I     & T          & 61.6719    & 0.000852537 & 95       \\
            BKW04    & M     & I     & F          & 85.5312    & 0.00112362  & 92.5     \\
            BKW04    & N     & A     & T          & 96.1875    & 0.0452466   & 82.5     \\
            BKW04    & N     & A     & F          & 44.6719    & 0.114416    & 87.5     \\
            BKW04    & N     & P     & T          & 95.0625    & 0.0556697   & 75       \\
            BKW04    & N     & P     & F          & 48.2812    & 0.13407     & 95       \\
            BKW04    & N     & H     & T          & 94         & 0.0527213   & 92.5     \\
            BKW04    & N     & H     & F          & 58.1562    & 0.103932    & 95       \\
            BKW04    & N     & W     & T          & 93.875     & 0.118128    & 95       \\
            BKW04    & N     & W     & F          & 63.7812    & 0.154315    & 97.5     \\
            BKW04    & N     & I     & T          & 48.2812    & 0.0522459   & 95       \\
            BKW04    & N     & I     & F          & 88.7656    & 0.0849498   & 92.5     \\
            \hline
        \end{tabular}
    }{}
    \fonte{autor}
\end{table}
\begin{table}[H]
    \centering
    \caption{Resultados da instância BKW05.}
    \label{tab:bkw05}
    \IBGEtab{}{
        \ttfamily\input{utils/tabular/bkw/bkw05}
    }{}
    \fonte{feito pelo autor.}
\end{table}
\begin{table}[!htb]
    \centering
    \caption{Resultados da instância BKW06.}
    \label{tab:bkw06}
    \IBGEtab{}{
        \input{utils/tabular/bkw/bkw06}
    }{}
    \fonte{autor}
\end{table}
\begin{table}[!htb]
    \centering
    \caption{Resultados da instância BKW07.}
    \label{tab:bkw07}
    \IBGEtab{}{
        \begin{tabular}{llllrrr}
            \hline
            Instance & Split & Order & Descending & Quality \% & Time (s)    & Items \% \\
            \hline
            BKW07    & V     & A     & T          & 78.8       & 0.000760746 & 84.2857  \\
            BKW07    & V     & A     & F          & 41.5375    & 0.000629044 & 85.7143  \\
            BKW07    & V     & P     & T          & 75.775     & 0.000708008 & 84.2857  \\
            BKW07    & V     & P     & F          & 40.6375    & 0.000621176 & 84.2857  \\
            BKW07    & V     & H     & T          & 67.8       & 0.000747824 & 81.4286  \\
            BKW07    & V     & H     & F          & 42.0875    & 0.000708628 & 87.1429  \\
            BKW07    & V     & W     & T          & 91.6625    & 0.000865507 & 95.7143  \\
            BKW07    & V     & W     & F          & 38.725     & 0.000543356 & 80       \\
            BKW07    & V     & I     & T          & 66.6       & 0.000843382 & 88.5714  \\
            BKW07    & V     & I     & F          & 46.725     & 0.000789738 & 84.2857  \\
            BKW07    & H     & A     & T          & 74.0125    & 0.00088582  & 85.7143  \\
            BKW07    & H     & A     & F          & 8.725      & 0.000820494 & 62.8571  \\
            BKW07    & H     & P     & T          & 83.775     & 0.000965977 & 88.5714  \\
            BKW07    & H     & P     & F          & 11.525     & 0.000853729 & 67.1429  \\
            BKW07    & H     & H     & T          & 86.1       & 0.000823164 & 95.7143  \\
            BKW07    & H     & H     & F          & 6.75       & 0.00078702  & 58.5714  \\
            BKW07    & H     & W     & T          & 46.675     & 0.000730562 & 78.5714  \\
            BKW07    & H     & W     & F          & 26.7375    & 0.000958061 & 80       \\
            BKW07    & H     & I     & T          & 46.825     & 0.000719404 & 78.5714  \\
            BKW07    & H     & I     & F          & 59.8       & 0.00112739  & 87.1429  \\
            BKW07    & M     & A     & T          & 82.725     & 0.00155454  & 90       \\
            BKW07    & M     & A     & F          & 40.1375    & 0.00144701  & 88.5714  \\
            BKW07    & M     & P     & T          & 90.5875    & 0.00149145  & 91.4286  \\
            BKW07    & M     & P     & F          & 42.0875    & 0.00145383  & 90       \\
            BKW07    & M     & H     & T          & 82.35      & 0.00155249  & 95.7143  \\
            BKW07    & M     & H     & F          & 38.9       & 0.00141234  & 81.4286  \\
            BKW07    & M     & W     & T          & 49.075     & 0.00137482  & 82.8571  \\
            BKW07    & M     & W     & F          & 61.2125    & 0.00142508  & 94.2857  \\
            BKW07    & M     & I     & T          & 79.4375    & 0.00166879  & 94.2857  \\
            BKW07    & M     & I     & F          & 64.3375    & 0.00148234  & 91.4286  \\
            BKW07    & N     & A     & T          & 90.2375    & 0.314977    & 95.7143  \\
            BKW07    & N     & A     & F          & 50.2875    & 0.657058    & 92.8571  \\
            BKW07    & N     & P     & T          & 91.5875    & 0.234417    & 92.8571  \\
            BKW07    & N     & P     & F          & 60.7125    & 0.639531    & 92.8571  \\
            BKW07    & N     & H     & T          & 82.35      & 0.256859    & 95.7143  \\
            BKW07    & N     & H     & F          & 56.6875    & 0.54043     & 88.5714  \\
            BKW07    & N     & W     & T          & 94.5       & 0.19643     & 95.7143  \\
            BKW07    & N     & W     & F          & 64.3       & 0.728899    & 94.2857  \\
            BKW07    & N     & I     & T          & 79.4375    & 0.337527    & 94.2857  \\
            BKW07    & N     & I     & F          & 75.8125    & 0.334543    & 97.1429  \\
            \hline
        \end{tabular}
    }{}
    \fonte{autor}
\end{table}
\begin{table}[H]
    \centering
    \caption{Resultados da instância BKW08.}
    \label{tab:bkw08}
    \IBGEtab{}{
        \ttfamily\input{utils/tabular/bkw/bkw08}
    }{}
    \fonte{feito pelo autor.}
\end{table}
\begin{table}[!htb]
    \centering
    \caption{Resultados da instância BKW09.}
    \label{tab:bkw09}
    \IBGEtab{}{
        \input{utils/tabular/bkw/bkw09}
    }{}
    \fonte{autor}
\end{table}
\begin{table}[H]
    \centering
    \caption{Resultados da instância BKW10.}
    \label{tab:bkw10}
    \IBGEtab{}{
        \ttfamily\input{utils/tabular/bkw/bkw10}
    }{}
    \fonte{feito pelo autor.}
\end{table}
\begin{table}[!htb]
    \centering
    \caption{Resultados da instância BKW11.}
    \label{tab:bkw11}
    \IBGEtab{}{
        \input{utils/tabular/bkw/bkw11}
    }{}
    \fonte{autor}
\end{table}
\begin{table}[!htb]
    \centering
    \caption{Resultados da instância BKW12.}
    \label{tab:bkw12}
    \IBGEtab{}{
        \input{utils/tabular/bkw/bkw12}
    }{}
    \fonte{autor}
\end{table}

A instância BKW13, mostrada na \Cref{tab:bkw13}, não foi executada com regiões complexas, pois seu
tempo de execução ultrapassaria uma hora e conseguiria resultados semelhantes aos demais modos
(como mostrado na \Cref{sec:criacao-de-regioes}).
\begin{table}[htb!]
    \IBGEtab{
        \caption{Resultados da instância BKW13.}
        \label{tab:bkw13}
    }{
        \begin{tabular}{llllrrr}
            \hline
            Instance & Split & Order & Descending & Quality \% & Time (s)  & Items \% \\
            \hline
            BKW13    & V     & A     & T          & 94.9339    & 0.116755  & 84.0102  \\
            BKW13    & V     & A     & F          & 72.6849    & 0.0956778 & 96.4467  \\
            BKW13    & V     & P     & T          & 91.0596    & 0.0820553 & 74.2386  \\
            BKW13    & V     & P     & F          & 73.5465    & 0.0809931 & 96.1294  \\
            BKW13    & V     & H     & T          & 86.7168    & 0.0906435 & 86.9607  \\
            BKW13    & V     & H     & F          & 78.2632    & 0.0731152 & 96.986   \\
            BKW13    & V     & W     & T          & 97.9108    & 0.0926637 & 90.736   \\
            BKW13    & V     & W     & F          & 73.9661    & 0.0609208 & 88.8325  \\
            BKW13    & V     & I     & T          & 61.0007    & 0.120608  & 83.9467  \\
            BKW13    & V     & I     & F          & 60.3755    & 0.0897346 & 82.2018  \\
            BKW13    & H     & A     & T          & 51.3125    & 0.230229  & 57.868   \\
            BKW13    & H     & A     & F          & 4.21354    & 0.372174  & 32.2335  \\
            BKW13    & H     & P     & T          & 63.4264    & 0.183033  & 63.2614  \\
            BKW13    & H     & P     & F          & 4.26562    & 0.355907  & 29.9492  \\
            BKW13    & H     & H     & T          & 85.3234    & 0.236992  & 86.0406  \\
            BKW13    & H     & H     & F          & 1.2526     & 0.322697  & 18.7817  \\
            BKW13    & H     & W     & T          & 32.3698    & 0.310274  & 54.3147  \\
            BKW13    & H     & W     & F          & 11.562     & 0.418725  & 55.7424  \\
            BKW13    & H     & I     & T          & 38.9907    & 0.250041  & 77.1574  \\
            BKW13    & H     & I     & F          & 38.8179    & 0.240461  & 76.7449  \\
            BKW13    & M     & A     & T          & 92.3503    & 0.54441   & 94.1307  \\
            BKW13    & M     & A     & F          & 32.1917    & 0.717847  & 81.8528  \\
            BKW13    & M     & P     & T          & 91.9723    & 0.534671  & 93.8135  \\
            BKW13    & M     & P     & F          & 33.3984    & 0.743062  & 82.0431  \\
            BKW13    & M     & H     & T          & 97.1108    & 0.526851  & 95.8122  \\
            BKW13    & M     & H     & F          & 9.58626    & 0.435822  & 44.4162  \\
            BKW13    & M     & W     & T          & 55.9587    & 0.610397  & 85.0254  \\
            BKW13    & M     & W     & F          & 82.3021    & 0.14048   & 95.3363  \\
            BKW13    & M     & I     & T          & 94.4312    & 0.589158  & 98.7627  \\
            BKW13    & M     & I     & F          & 94.6325    & 0.596748  & 98.7944  \\
            \hline
        \end{tabular}
    }{
        \fonte{autor}
    }
\end{table} 



\section{GCUT}\label{sec:gcut}
\begin{table}[H]
    \centering
    \caption{Resultados da instância GCUT01.}
    \label{tab:gcut01}
    \IBGEtab{}{
        \ttfamily\input{utils/tabular/gcut/gcut01}
    }{}
    \fonte{feito pelo autor.}
\end{table}
\begin{table}[!htb]
    \centering
    \caption{Resultados da instância GCUT02.}
    \label{tab:gcut02}
    \IBGEtab{}{
        \input{utils/tabular/gcut/gcut02}
    }{}
    \fonte{autor}
\end{table}
\begin{table}[H]
    \centering
    \caption{Resultados da instância GCUT03.}
    \label{tab:gcut03}
    \IBGEtab{}{
        \ttfamily\input{utils/tabular/gcut/gcut03}
    }{}
    \fonte{feito pelo autor.}
\end{table}
\begin{table}[htb!]
    \IBGEtab{
        \caption{Resultados da instância GCUT04.}
        \label{tab:gcut04}
    }{
        \begin{tabular}{llllrrr}
            \hline
            Instance & Split & Order & Descending & Quality \% & Time (s)    & Items \% \\
            \hline
            GCUT04   & V     & A     & T          & 92.816     & 6.84738e-05 & 8        \\
            GCUT04   & V     & A     & F          & 58.848     & 0.00010767  & 12       \\
            GCUT04   & V     & P     & T          & 92.816     & 7.41959e-05 & 8        \\
            GCUT04   & V     & P     & F          & 58.848     & 0.000107574 & 12       \\
            GCUT04   & V     & H     & T          & 90.8624    & 7.66754e-05 & 10       \\
            GCUT04   & V     & H     & F          & 61.568     & 0.000107479 & 12       \\
            GCUT04   & V     & W     & T          & 86.7136    & 6.89507e-05 & 8        \\
            GCUT04   & V     & W     & F          & 44.5504    & 6.60896e-05 & 8        \\
            GCUT04   & V     & I     & T          & 72.9616    & 7.76768e-05 & 8        \\
            GCUT04   & V     & I     & F          & 63.7264    & 8.564e-05   & 10       \\
            GCUT04   & H     & A     & T          & 83.6768    & 6.53744e-05 & 6        \\
            GCUT04   & H     & A     & F          & 71.9872    & 0.000146818 & 14       \\
            GCUT04   & H     & P     & T          & 83.6768    & 6.11782e-05 & 6        \\
            GCUT04   & H     & P     & F          & 54.7504    & 0.000130653 & 12       \\
            GCUT04   & H     & H     & T          & 82.728     & 7.10487e-05 & 8        \\
            GCUT04   & H     & H     & F          & 34.8656    & 7.44343e-05 & 6        \\
            GCUT04   & H     & W     & T          & 96.1424    & 8.43525e-05 & 10       \\
            GCUT04   & H     & W     & F          & 64.032     & 0.000129986 & 12       \\
            GCUT04   & H     & I     & T          & 79.336     & 8.54015e-05 & 8        \\
            GCUT04   & H     & I     & F          & 67.7552    & 8.90255e-05 & 8        \\
            GCUT04   & M     & A     & T          & 94.3856    & 0.00011673  & 8        \\
            GCUT04   & M     & A     & F          & 64.5616    & 0.000192499 & 14       \\
            GCUT04   & M     & P     & T          & 94.3856    & 0.000116968 & 8        \\
            GCUT04   & M     & P     & F          & 64.5616    & 0.000191879 & 14       \\
            GCUT04   & M     & H     & T          & 93.4368    & 0.000138712 & 10       \\
            GCUT04   & M     & H     & F          & 53.4192    & 0.000158453 & 10       \\
            GCUT04   & M     & W     & T          & 96.1424    & 0.0001441   & 10       \\
            GCUT04   & M     & W     & F          & 71.2128    & 0.000163174 & 12       \\
            GCUT04   & M     & I     & T          & 72.9616    & 0.00012188  & 8        \\
            GCUT04   & M     & I     & F          & 67.7552    & 0.000131512 & 8        \\
            GCUT04   & N     & A     & T          & 94.3856    & 0.000424194 & 8        \\
            GCUT04   & N     & A     & F          & 75.2704    & 0.0063355   & 16       \\
            GCUT04   & N     & P     & T          & 94.3856    & 0.000369072 & 8        \\
            GCUT04   & N     & P     & F          & 75.2704    & 0.00636544  & 16       \\
            GCUT04   & N     & H     & T          & 93.4368    & 0.000379848 & 10       \\
            GCUT04   & N     & H     & F          & 62.848     & 0.00367494  & 12       \\
            GCUT04   & N     & W     & T          & 96.1424    & 0.000389242 & 10       \\
            GCUT04   & N     & W     & F          & 71.2128    & 0.00524592  & 12       \\
            GCUT04   & N     & I     & T          & 79.336     & 0.000852728 & 8        \\
            GCUT04   & N     & I     & F          & 67.7552    & 0.000873375 & 8        \\
            \hline
        \end{tabular}
    }{
        \fonte{autor}
    }
\end{table} 

\begin{table}[htb!]
    \IBGEtab{
        \caption{Resultados da instância GCUT05.}
        \label{tab:gcut05}
    }{
        \begin{tabular}{llllrrr}
            \hline
            Instance & Split & Order & Descending & Quality \% & Time (s)    & Items \% \\
            \hline
            GCUT05   & V     & A     & T          & 65.744     & 3.80516e-05 & 30       \\
            GCUT05   & V     & A     & F          & 36.6852    & 3.66211e-05 & 30       \\
            GCUT05   & V     & P     & T          & 65.744     & 3.65734e-05 & 30       \\
            GCUT05   & V     & P     & F          & 36.6852    & 3.80039e-05 & 30       \\
            GCUT05   & V     & H     & T          & 65.744     & 3.80039e-05 & 30       \\
            GCUT05   & V     & H     & F          & 56.6948    & 4.72546e-05 & 40       \\
            GCUT05   & V     & W     & T          & 63.9648    & 3.94821e-05 & 30       \\
            GCUT05   & V     & W     & F          & 36.6852    & 3.93867e-05 & 30       \\
            GCUT05   & V     & I     & T          & 46.244     & 3.88145e-05 & 30       \\
            GCUT05   & V     & I     & F          & 52.5148    & 4.87328e-05 & 40       \\
            GCUT05   & H     & A     & T          & 56.5568    & 2.79903e-05 & 20       \\
            GCUT05   & H     & A     & F          & 56.6948    & 5.04494e-05 & 40       \\
            GCUT05   & H     & P     & T          & 56.5568    & 2.72751e-05 & 20       \\
            GCUT05   & H     & P     & F          & 56.9356    & 4.96387e-05 & 40       \\
            GCUT05   & H     & H     & T          & 56.5568    & 4.03404e-05 & 20       \\
            GCUT05   & H     & H     & F          & 57.0664    & 3.7241e-05  & 30       \\
            GCUT05   & H     & W     & T          & 63.9648    & 3.69072e-05 & 30       \\
            GCUT05   & H     & W     & F          & 52.9212    & 5.15938e-05 & 40       \\
            GCUT05   & H     & I     & T          & 57.342     & 5.03063e-05 & 40       \\
            GCUT05   & H     & I     & F          & 52.5148    & 5.2166e-05  & 40       \\
            GCUT05   & M     & A     & T          & 72.7928    & 7.59125e-05 & 30       \\
            GCUT05   & M     & A     & F          & 56.6948    & 9.5892e-05  & 40       \\
            GCUT05   & M     & P     & T          & 72.7928    & 7.1907e-05  & 30       \\
            GCUT05   & M     & P     & F          & 52.9212    & 9.57966e-05 & 40       \\
            GCUT05   & M     & H     & T          & 72.7928    & 7.11441e-05 & 30       \\
            GCUT05   & M     & H     & F          & 66.2536    & 9.4986e-05  & 40       \\
            GCUT05   & M     & W     & T          & 63.9648    & 7.1907e-05  & 30       \\
            GCUT05   & M     & W     & F          & 52.9212    & 9.17435e-05 & 40       \\
            GCUT05   & M     & I     & T          & 57.342     & 9.08375e-05 & 40       \\
            GCUT05   & M     & I     & F          & 68.9148    & 0.000117874 & 50       \\
            GCUT05   & N     & A     & T          & 72.7928    & 0.000122213 & 30       \\
            GCUT05   & N     & A     & F          & 56.6948    & 0.000473309 & 40       \\
            GCUT05   & N     & P     & T          & 72.7928    & 0.000122213 & 30       \\
            GCUT05   & N     & P     & F          & 52.5148    & 0.000568724 & 40       \\
            GCUT05   & N     & H     & T          & 72.7928    & 0.000115204 & 30       \\
            GCUT05   & N     & H     & F          & 66.2536    & 0.000424099 & 40       \\
            GCUT05   & N     & W     & T          & 63.9648    & 0.000118256 & 30       \\
            GCUT05   & N     & W     & F          & 52.9212    & 0.000395823 & 40       \\
            GCUT05   & N     & I     & T          & 73.3356    & 0.000308418 & 50       \\
            GCUT05   & N     & I     & F          & 52.5148    & 0.000315094 & 40       \\
            \hline
        \end{tabular}
    }{
        \fonte{autor}
    }
\end{table} 

\begin{table}[!htb]
    \centering
    \caption{Resultados da instância GCUT06.}
    \label{tab:gcut06}
    \IBGEtab{}{
        \ttfamily\input{utils/tabular/gcut/gcut06}
    }{}
    \fonte{feito pelo autor.}
\end{table}
\begin{table}[!htb]
    \centering
    \caption{Resultados da instância GCUT07.}
    \label{tab:gcut07}
    \IBGEtab{}{
        \input{utils/tabular/gcut/gcut07}
    }{}
    \fonte{autor}
\end{table}
\begin{table}[htb!]
    \IBGEtab{
        \caption{Resultados da instância GCUT08.}
        \label{tab:gcut08}
    }{
        \begin{tabular}{llllrrr}
            \hline
            Instance & Split & Order & Descending & Quality \% & Time (s)    & Items \% \\
            \hline
            GCUT08   & V     & A     & T          & 79.6752    & 5.63145e-05 & 6        \\
            GCUT08   & V     & A     & F          & 63.7732    & 0.000107336 & 12       \\
            GCUT08   & V     & P     & T          & 79.6752    & 5.80311e-05 & 6        \\
            GCUT08   & V     & P     & F          & 52.9528    & 9.30786e-05 & 10       \\
            GCUT08   & V     & H     & T          & 79.6752    & 6.20842e-05 & 6        \\
            GCUT08   & V     & H     & F          & 62.9792    & 0.000108194 & 12       \\
            GCUT08   & V     & W     & T          & 81.3776    & 7.27654e-05 & 8        \\
            GCUT08   & V     & W     & F          & 35.9152    & 6.1512e-05  & 6        \\
            GCUT08   & V     & I     & T          & 72.664     & 7.7343e-05  & 8        \\
            GCUT08   & V     & I     & F          & 74.4596    & 9.20773e-05 & 10       \\
            GCUT08   & H     & A     & T          & 72.1296    & 6.44684e-05 & 4        \\
            GCUT08   & H     & A     & F          & 61.8724    & 0.00012908  & 12       \\
            GCUT08   & H     & P     & T          & 72.1296    & 4.93526e-05 & 4        \\
            GCUT08   & H     & P     & F          & 61.8724    & 0.000138092 & 12       \\
            GCUT08   & H     & H     & T          & 72.1296    & 6.45161e-05 & 4        \\
            GCUT08   & H     & H     & F          & 39.6812    & 7.40528e-05 & 6        \\
            GCUT08   & H     & W     & T          & 63.4032    & 5.84126e-05 & 6        \\
            GCUT08   & H     & W     & F          & 66.0228    & 0.000132227 & 12       \\
            GCUT08   & H     & I     & T          & 72.664     & 8.59261e-05 & 8        \\
            GCUT08   & H     & I     & F          & 66.914     & 9.01699e-05 & 8        \\
            GCUT08   & M     & A     & T          & 79.6752    & 9.20296e-05 & 6        \\
            GCUT08   & M     & A     & F          & 48.0484    & 0.000157833 & 10       \\
            GCUT08   & M     & P     & T          & 79.6752    & 9.16958e-05 & 6        \\
            GCUT08   & M     & P     & F          & 48.0484    & 0.000147963 & 10       \\
            GCUT08   & M     & H     & T          & 79.6752    & 9.43184e-05 & 6        \\
            GCUT08   & M     & H     & F          & 39.6812    & 0.000107527 & 6        \\
            GCUT08   & M     & W     & T          & 63.4032    & 9.35555e-05 & 6        \\
            GCUT08   & M     & W     & F          & 55.2024    & 0.000148726 & 10       \\
            GCUT08   & M     & I     & T          & 84.072     & 0.000149632 & 10       \\
            GCUT08   & M     & I     & F          & 66.914     & 0.000127172 & 8        \\
            GCUT08   & N     & A     & T          & 79.6752    & 0.000170326 & 6        \\
            GCUT08   & N     & A     & F          & 61.8724    & 0.00434747  & 12       \\
            GCUT08   & N     & P     & T          & 79.6752    & 0.000167513 & 6        \\
            GCUT08   & N     & P     & F          & 61.8724    & 0.00390096  & 12       \\
            GCUT08   & N     & H     & T          & 79.6752    & 0.000202131 & 6        \\
            GCUT08   & N     & H     & F          & 47.5604    & 0.00232964  & 8        \\
            GCUT08   & N     & W     & T          & 81.3776    & 0.000264549 & 8        \\
            GCUT08   & N     & W     & F          & 72.9048    & 0.00430069  & 12       \\
            GCUT08   & N     & I     & T          & 84.072     & 0.000531626 & 10       \\
            GCUT08   & N     & I     & F          & 66.914     & 0.00116959  & 8        \\
            \hline
        \end{tabular}
    }{
        \fonte{autor}
    }
\end{table} 

\begin{table}[H]
    \centering
    \caption{Resultados da instância GCUT09.}
    \label{tab:gcut09}
    \IBGEtab{}{
        \ttfamily\input{utils/tabular/gcut/gcut09}
    }{}
    \fonte{feito pelo autor.}
\end{table}
\begin{table}[!htb]
    \centering
    \caption{Resultados da instância GCUT10.}
    \label{tab:gcut10}
    \IBGEtab{}{
        \input{utils/tabular/gcut/gcut10}
    }{}
    \fonte{autor}
\end{table}
\begin{table}[H]
    \centering
    \caption{Resultados da instância GCUT11.}
    \label{tab:gcut11}
    \IBGEtab{}{
        \ttfamily\input{utils/tabular/gcut/gcut11}
    }{}
    \fonte{feito pelo autor.}
\end{table}
\begin{table}[!htb]
    \centering
    \caption{Resultados da instância GCUT12.}
    \label{tab:gcut12}
    \IBGEtab{}{
        \input{utils/tabular/gcut/gcut12}
    }{}
    \fonte{autor}
\end{table}
\begin{table}[H]
    \centering
    \caption{Resultados da instância GCUT13.}
    \label{tab:gcut13}
    \IBGEtab{}{
        \ttfamily\input{utils/tabular/gcut/gcut13}
    }{}
    \fonte{feito pelo autor.}
\end{table}


\section{NGCUT}\label{sec:ngcut}
\begin{table}[!htb]
    \centering
    \caption{Resultados da instância NGCUT01.}
    \label{tab:ngcut01}
    \IBGEtab{}{
        \input{utils/tabular/ngcut/ngcut01}
    }{}
    \fonte{autor}
\end{table}
\begin{table}[!htb]
    \centering
    \caption{Resultados da instância NGCUT02.}
    \label{tab:ngcut02}
    \IBGEtab{}{
        \input{utils/tabular/ngcut/ngcut02}
    }{}
    \fonte{autor}
\end{table}
\begin{table}[!htb]
    \centering
    \caption{Resultados da instância NGCUT03.}
    \label{tab:ngcut03}
    \IBGEtab{}{
        \begin{tabular}{llllrrr}
            \hline
            Instance & Split & Order & Descending & Quality \% & Time (s)    & Items \% \\
            \hline
            NGCUT03  & V     & A     & T          & 90         & 5.84602e-05 & 28.5714  \\
            NGCUT03  & V     & A     & F          & 66         & 8.54015e-05 & 38.0952  \\
            NGCUT03  & V     & P     & T          & 84         & 5.04494e-05 & 23.8095  \\
            NGCUT03  & V     & P     & F          & 57         & 7.47681e-05 & 33.3333  \\
            NGCUT03  & V     & H     & T          & 85         & 5.80311e-05 & 28.5714  \\
            NGCUT03  & V     & H     & F          & 66         & 8.50201e-05 & 38.0952  \\
            NGCUT03  & V     & W     & T          & 88         & 5.87463e-05 & 28.5714  \\
            NGCUT03  & V     & W     & F          & 64         & 5.24044e-05 & 23.8095  \\
            NGCUT03  & V     & I     & T          & 93         & 7.20024e-05 & 38.0952  \\
            NGCUT03  & V     & I     & F          & 70         & 7.57217e-05 & 38.0952  \\
            NGCUT03  & H     & A     & T          & 81         & 5.05924e-05 & 19.0476  \\
            NGCUT03  & H     & A     & F          & 72         & 9.57489e-05 & 42.8571  \\
            NGCUT03  & H     & P     & T          & 90         & 6.59466e-05 & 33.3333  \\
            NGCUT03  & H     & P     & F          & 72         & 9.64165e-05 & 42.8571  \\
            NGCUT03  & H     & H     & T          & 93         & 6.63757e-05 & 38.0952  \\
            NGCUT03  & H     & H     & F          & 80         & 9.48906e-05 & 42.8571  \\
            NGCUT03  & H     & W     & T          & 90         & 6.85215e-05 & 33.3333  \\
            NGCUT03  & H     & W     & F          & 78         & 8.61168e-05 & 38.0952  \\
            NGCUT03  & H     & I     & T          & 90         & 0.000159931 & 52.381   \\
            NGCUT03  & H     & I     & F          & 70         & 9.40323e-05 & 38.0952  \\
            NGCUT03  & M     & A     & T          & 80         & 9.90868e-05 & 23.8095  \\
            NGCUT03  & M     & A     & F          & 60         & 0.000185966 & 38.0952  \\
            NGCUT03  & M     & P     & T          & 90         & 0.000130463 & 33.3333  \\
            NGCUT03  & M     & P     & F          & 72         & 0.000187302 & 42.8571  \\
            NGCUT03  & M     & H     & T          & 85         & 0.000122499 & 28.5714  \\
            NGCUT03  & M     & H     & F          & 80         & 0.000188351 & 42.8571  \\
            NGCUT03  & M     & W     & T          & 90         & 0.000130367 & 33.3333  \\
            NGCUT03  & M     & W     & F          & 78         & 0.000167227 & 38.0952  \\
            NGCUT03  & M     & I     & T          & 90         & 0.00020771  & 52.381   \\
            NGCUT03  & M     & I     & F          & 70         & 0.000174761 & 38.0952  \\
            NGCUT03  & N     & A     & T          & 90         & 0.000606346 & 28.5714  \\
            NGCUT03  & N     & A     & F          & 86         & 0.00535707  & 42.8571  \\
            NGCUT03  & N     & P     & T          & 84         & 0.000549841 & 23.8095  \\
            NGCUT03  & N     & P     & F          & 72         & 0.00190721  & 42.8571  \\
            NGCUT03  & N     & H     & T          & 93         & 0.000912476 & 38.0952  \\
            NGCUT03  & N     & H     & F          & 88         & 0.00291681  & 47.619   \\
            NGCUT03  & N     & W     & T          & 90         & 0.000712872 & 33.3333  \\
            NGCUT03  & N     & W     & F          & 72         & 0.00305543  & 33.3333  \\
            NGCUT03  & N     & I     & T          & 93         & 0.00315442  & 38.0952  \\
            NGCUT03  & N     & I     & F          & 80         & 0.00251474  & 42.8571  \\
            \hline
        \end{tabular}
    }{}
    \fonte{autor}
\end{table}
\begin{table}[!htb]
    \centering
    \caption{Resultados da instância NGCUT04.}
    \label{tab:ngcut04}
    \IBGEtab{}{
        \input{utils/tabular/ngcut/ngcut04}
    }{}
    \fonte{autor}
\end{table}
\begin{table}[H]
    \centering
    \caption{Resultados da instância NGCUT05.}
    \label{tab:ngcut05}
    \IBGEtab{}{
        \ttfamily\input{utils/tabular/ngcut/ngcut05}
    }{}
    \fonte{feito pelo autor.}
\end{table}
\begin{table}[!htb]
    \centering
    \caption{Resultados da instância NGCUT06.}
    \label{tab:ngcut06}
    \IBGEtab{}{
        \input{utils/tabular/ngcut/ngcut06}
    }{}
    \fonte{autor}
\end{table}
\begin{table}[H]
    \centering
    \caption{Resultados da instância NGCUT07.}
    \label{tab:ngcut07}
    \IBGEtab{}{
        \ttfamily\input{utils/tabular/ngcut/ngcut07}
    }{}
    \fonte{feito pelo autor.}
\end{table}
\begin{table}[!htb]
    \centering
    \caption{Resultados da instância NGCUT08.}
    \label{tab:ngcut08}
    \IBGEtab{}{
        \input{utils/tabular/ngcut/ngcut08}
    }{}
    \fonte{autor}
\end{table}
\begin{table}[!htb]
    \centering
    \caption{Resultados da instância NGCUT09.}
    \label{tab:ngcut09}
    \IBGEtab{}{
        \input{utils/tabular/ngcut/ngcut09}
    }{}
    \fonte{autor}
\end{table}
\begin{table}[!htb]
    \centering
    \caption{Resultados da instância NGCUT10.}
    \label{tab:ngcut10}
    \IBGEtab{}{
        \begin{tabular}{llllrrr}
            \hline
            Instance & Split & Order & Descending & Quality \% & Time (s)    & Items \% \\
            \hline
            NGCUT10  & V     & A     & T          & 91         & 7.08103e-05 & 46.1538  \\
            NGCUT10  & V     & A     & F          & 29.4444    & 3.46184e-05 & 30.7692  \\
            NGCUT10  & V     & P     & T          & 91         & 5.14507e-05 & 46.1538  \\
            NGCUT10  & V     & P     & F          & 29.4444    & 3.29971e-05 & 30.7692  \\
            NGCUT10  & V     & H     & T          & 91         & 4.15325e-05 & 46.1538  \\
            NGCUT10  & V     & H     & F          & 39.4444    & 3.51906e-05 & 30.7692  \\
            NGCUT10  & V     & W     & T          & 74         & 4.17233e-05 & 38.4615  \\
            NGCUT10  & V     & W     & F          & 29.4444    & 3.37601e-05 & 30.7692  \\
            NGCUT10  & V     & I     & T          & 90.1111    & 6.86646e-05 & 53.8462  \\
            NGCUT10  & V     & I     & F          & 29.4444    & 3.39031e-05 & 30.7692  \\
            NGCUT10  & H     & A     & T          & 87.6667    & 5.0211e-05  & 30.7692  \\
            NGCUT10  & H     & A     & F          & 90.1111    & 6.19411e-05 & 53.8462  \\
            NGCUT10  & H     & P     & T          & 87.6667    & 5.04971e-05 & 30.7692  \\
            NGCUT10  & H     & P     & F          & 90.1111    & 6.34193e-05 & 53.8462  \\
            NGCUT10  & H     & H     & T          & 91         & 4.03881e-05 & 46.1538  \\
            NGCUT10  & H     & H     & F          & 79.8889    & 5.3978e-05  & 46.1538  \\
            NGCUT10  & H     & W     & T          & 74         & 4.67777e-05 & 38.4615  \\
            NGCUT10  & H     & W     & F          & 90.1111    & 6.5136e-05  & 53.8462  \\
            NGCUT10  & H     & I     & T          & 80.6667    & 6.39439e-05 & 46.1538  \\
            NGCUT10  & H     & I     & F          & 83.4444    & 4.9305e-05  & 46.1538  \\
            NGCUT10  & M     & A     & T          & 87.6667    & 9.34601e-05 & 30.7692  \\
            NGCUT10  & M     & A     & F          & 90.1111    & 0.000129843 & 53.8462  \\
            NGCUT10  & M     & P     & T          & 87.6667    & 9.12189e-05 & 30.7692  \\
            NGCUT10  & M     & P     & F          & 90.1111    & 0.000138569 & 53.8462  \\
            NGCUT10  & M     & H     & T          & 91         & 9.6941e-05  & 46.1538  \\
            NGCUT10  & M     & H     & F          & 79.8889    & 0.000114059 & 46.1538  \\
            NGCUT10  & M     & W     & T          & 74         & 9.68456e-05 & 38.4615  \\
            NGCUT10  & M     & W     & F          & 90.1111    & 0.000138569 & 53.8462  \\
            NGCUT10  & M     & I     & T          & 80.6667    & 0.000123978 & 46.1538  \\
            NGCUT10  & M     & I     & F          & 83.4444    & 0.000103998 & 46.1538  \\
            NGCUT10  & N     & A     & T          & 87.6667    & 0.000439978 & 30.7692  \\
            NGCUT10  & N     & A     & F          & 90.1111    & 0.000567341 & 53.8462  \\
            NGCUT10  & N     & P     & T          & 87.6667    & 0.000444937 & 30.7692  \\
            NGCUT10  & N     & P     & F          & 90.1111    & 0.000564098 & 53.8462  \\
            NGCUT10  & N     & H     & T          & 91         & 0.000398493 & 46.1538  \\
            NGCUT10  & N     & H     & F          & 79.8889    & 0.000414515 & 46.1538  \\
            NGCUT10  & N     & W     & T          & 74         & 0.000288916 & 38.4615  \\
            NGCUT10  & N     & W     & F          & 90.1111    & 0.000548363 & 53.8462  \\
            NGCUT10  & N     & I     & T          & 80.6667    & 0.000660706 & 46.1538  \\
            NGCUT10  & N     & I     & F          & 83.4444    & 0.000408363 & 46.1538  \\
            \hline
        \end{tabular}
    }{}
    \fonte{autor}
\end{table}
\begin{table}[!htb]
    \centering
    \caption{Resultados da instância NGCUT11.}
    \label{tab:ngcut11}
    \IBGEtab{}{
        \input{utils/tabular/ngcut/ngcut11}
    }{}
    \fonte{autor}
\end{table}
\begin{table}[!htb]
    \centering
    \caption{Resultados da instância NGCUT12.}
    \label{tab:ngcut12}
    \IBGEtab{}{
        \begin{tabular}{llllrrr}
            \hline
            Instance & Split & Order & Descending & Quality \% & Time (s)    & Items \% \\
            \hline
            NGCUT12  & V     & A     & T          & 74.8889    & 8.01086e-05 & 31.8182  \\
            NGCUT12  & V     & A     & F          & 48.6667    & 0.000100374 & 40.9091  \\
            NGCUT12  & V     & P     & T          & 95         & 9.67503e-05 & 40.9091  \\
            NGCUT12  & V     & P     & F          & 52         & 8.28743e-05 & 31.8182  \\
            NGCUT12  & V     & H     & T          & 84.3333    & 8.41141e-05 & 40.9091  \\
            NGCUT12  & V     & H     & F          & 48.6667    & 9.60827e-05 & 40.9091  \\
            NGCUT12  & V     & W     & T          & 82.5556    & 9.75609e-05 & 45.4545  \\
            NGCUT12  & V     & W     & F          & 65.6667    & 7.26223e-05 & 36.3636  \\
            NGCUT12  & V     & I     & T          & 71.1111    & 9.0456e-05  & 36.3636  \\
            NGCUT12  & V     & I     & F          & 97.2222    & 9.84669e-05 & 40.9091  \\
            NGCUT12  & H     & A     & T          & 68.4444    & 8.7738e-05  & 36.3636  \\
            NGCUT12  & H     & A     & F          & 53.3333    & 0.000135231 & 45.4545  \\
            NGCUT12  & H     & P     & T          & 97.6667    & 9.408e-05   & 45.4545  \\
            NGCUT12  & H     & P     & F          & 53.3333    & 0.000137234 & 45.4545  \\
            NGCUT12  & H     & H     & T          & 84.3333    & 7.47204e-05 & 40.9091  \\
            NGCUT12  & H     & H     & F          & 53.3333    & 0.000135565 & 45.4545  \\
            NGCUT12  & H     & W     & T          & 54.4444    & 0.000120687 & 40.9091  \\
            NGCUT12  & H     & W     & F          & 83.4444    & 9.75132e-05 & 40.9091  \\
            NGCUT12  & H     & I     & T          & 65.7778    & 0.000121212 & 45.4545  \\
            NGCUT12  & H     & I     & F          & 53.3333    & 0.000130987 & 45.4545  \\
            NGCUT12  & M     & A     & T          & 57.7778    & 0.000163317 & 31.8182  \\
            NGCUT12  & M     & A     & F          & 53.3333    & 0.000246143 & 45.4545  \\
            NGCUT12  & M     & P     & T          & 95         & 0.000192642 & 40.9091  \\
            NGCUT12  & M     & P     & F          & 53.3333    & 0.000226068 & 45.4545  \\
            NGCUT12  & M     & H     & T          & 84.3333    & 0.000176334 & 40.9091  \\
            NGCUT12  & M     & H     & F          & 53.3333    & 0.000241232 & 45.4545  \\
            NGCUT12  & M     & W     & T          & 54.4444    & 0.000209951 & 40.9091  \\
            NGCUT12  & M     & W     & F          & 83.4444    & 0.000195646 & 40.9091  \\
            NGCUT12  & M     & I     & T          & 57.7778    & 0.00017643  & 31.8182  \\
            NGCUT12  & M     & I     & F          & 53.3333    & 0.000235033 & 45.4545  \\
            NGCUT12  & N     & A     & T          & 85.5556    & 0.000835943 & 36.3636  \\
            NGCUT12  & N     & A     & F          & 53.3333    & 0.00555758  & 45.4545  \\
            NGCUT12  & N     & P     & T          & 95         & 0.00142913  & 40.9091  \\
            NGCUT12  & N     & P     & F          & 53.3333    & 0.00465908  & 45.4545  \\
            NGCUT12  & N     & H     & T          & 84.3333    & 0.000932407 & 40.9091  \\
            NGCUT12  & N     & H     & F          & 53.3333    & 0.008707    & 45.4545  \\
            NGCUT12  & N     & W     & T          & 65.4444    & 0.00359254  & 45.4545  \\
            NGCUT12  & N     & W     & F          & 65.6667    & 0.00548348  & 36.3636  \\
            NGCUT12  & N     & I     & T          & 71.1111    & 0.00125933  & 36.3636  \\
            NGCUT12  & N     & I     & F          & 53.3333    & 0.00314965  & 45.4545  \\
            \hline
        \end{tabular}
    }{}
    \fonte{autor}
\end{table}


\section{OF}\label{sec:of}
\begin{table}[!htb]
    \centering
    \caption{Resultados da instância OF1.}
    \label{tab:of1}
    \IBGEtab{}{
        \begin{tabular}{llllrrr}
            \hline
            Instance & Split & Order & Descending & Quality \% & Time (s)    & Items \% \\
            \hline
            OF1      & V     & A     & T          & 71         & 8.69751e-05 & 26.087   \\
            OF1      & V     & A     & F          & 56.6071    & 8.91685e-05 & 34.7826  \\
            OF1      & V     & P     & T          & 84.8571    & 9.25064e-05 & 30.4348  \\
            OF1      & V     & P     & F          & 56.6071    & 9.00745e-05 & 34.7826  \\
            OF1      & V     & H     & T          & 95         & 0.000108004 & 34.7826  \\
            OF1      & V     & H     & F          & 74.75      & 8.81195e-05 & 39.1304  \\
            OF1      & V     & W     & T          & 75.5714    & 0.00010376  & 34.7826  \\
            OF1      & V     & W     & F          & 85.8571    & 0.000110388 & 34.7826  \\
            OF1      & V     & I     & T          & 74.5714    & 0.000113153 & 39.1304  \\
            OF1      & V     & I     & F          & 78.75      & 0.000104761 & 34.7826  \\
            OF1      & H     & A     & T          & 80.8929    & 7.71046e-05 & 30.4348  \\
            OF1      & H     & A     & F          & 56.2143    & 9.57012e-05 & 30.4348  \\
            OF1      & H     & P     & T          & 77.7857    & 6.67572e-05 & 21.7391  \\
            OF1      & H     & P     & F          & 70.7143    & 0.000101662 & 34.7826  \\
            OF1      & H     & H     & T          & 90.4286    & 8.51154e-05 & 34.7826  \\
            OF1      & H     & H     & F          & 58.1429    & 8.58784e-05 & 26.087   \\
            OF1      & H     & W     & T          & 60.25      & 8.81672e-05 & 26.087   \\
            OF1      & H     & W     & F          & 85.8571    & 9.25541e-05 & 34.7826  \\
            OF1      & H     & I     & T          & 69.1071    & 9.54628e-05 & 34.7826  \\
            OF1      & H     & I     & F          & 60.25      & 8.06332e-05 & 26.087   \\
            OF1      & M     & A     & T          & 75.7143    & 0.000147247 & 26.087   \\
            OF1      & M     & A     & F          & 56.6071    & 0.000204086 & 34.7826  \\
            OF1      & M     & P     & T          & 83.1786    & 0.000163364 & 30.4348  \\
            OF1      & M     & P     & F          & 56.6071    & 0.000185156 & 34.7826  \\
            OF1      & M     & H     & T          & 95         & 0.000195026 & 34.7826  \\
            OF1      & M     & H     & F          & 64.4286    & 0.00017972  & 30.4348  \\
            OF1      & M     & W     & T          & 74.75      & 0.000170326 & 30.4348  \\
            OF1      & M     & W     & F          & 85.8571    & 0.000205135 & 34.7826  \\
            OF1      & M     & I     & T          & 51.4286    & 0.000172853 & 30.4348  \\
            OF1      & M     & I     & F          & 69.1786    & 0.000281477 & 34.7826  \\
            OF1      & N     & A     & T          & 80.8929    & 0.00123076  & 30.4348  \\
            OF1      & N     & A     & F          & 56.6071    & 0.00387659  & 34.7826  \\
            OF1      & N     & P     & T          & 84.8571    & 0.00121098  & 30.4348  \\
            OF1      & N     & P     & F          & 56.6071    & 0.00546713  & 34.7826  \\
            OF1      & N     & H     & T          & 95         & 0.00144744  & 34.7826  \\
            OF1      & N     & H     & F          & 64.4286    & 0.00351505  & 30.4348  \\
            OF1      & N     & W     & T          & 81.0357    & 0.00174747  & 34.7826  \\
            OF1      & N     & W     & F          & 89         & 0.00117393  & 39.1304  \\
            OF1      & N     & I     & T          & 74.5714    & 0.00252275  & 39.1304  \\
            OF1      & N     & I     & F          & 78.75      & 0.00174804  & 34.7826  \\
            \hline
        \end{tabular}
    }{}
    \fonte{autor}
\end{table}
\begin{table}[!htb]
    \centering
    \caption{Resultados da instância OF2.}
    \label{tab:of2}
    \IBGEtab{}{
        \input{utils/tabular/of/of2}
    }{}
    \fonte{autor}
\end{table}


\section{OKP}\label{sec:okp}
\begin{table}[!htb]
    \centering
    \caption{Resultados da instância OKP1.}
    \label{tab:okp1}
    \IBGEtab{}{
        \ttfamily\input{utils/tabular/okp/okp1}
    }{}
    \fonte{feito pelo autor.}
\end{table}
\begin{table}[!htb]
    \centering
    \caption{Resultados da instância OKP2.}
    \label{tab:okp2}
    \IBGEtab{}{
        \ttfamily\input{utils/tabular/okp/okp2}
    }{}
    \fonte{feito pelo autor.}
\end{table}
\begin{table}[!htb]
    \centering
    \caption{Resultados da instância OKP3.}
    \label{tab:okp3}
    \IBGEtab{}{
        \ttfamily\input{utils/tabular/okp/okp3}
    }{}
    \fonte{feito pelo autor.}
\end{table}
\begin{table}[!htb]
    \centering
    \caption{Resultados da instância OKP4.}
    \label{tab:okp4}
    \IBGEtab{}{
        \ttfamily\input{utils/tabular/okp/okp4}
    }{}
    \fonte{feito pelo autor.}
\end{table}
\begin{table}[!htb]
    \centering
    \caption{Resultados da instância OKP5.}
    \label{tab:okp5}
    \IBGEtab{}{
        \begin{tabular}{llllrrr}
            \hline
            Instance & Split & Order & Descending & Quality \% & Time (s)    & Items \% \\
            \hline
            OKP5     & V     & A     & T          & 84.86      & 0.000170279 & 10.3093  \\
            OKP5     & V     & A     & F          & 58.74      & 0.000478125 & 22.6804  \\
            OKP5     & V     & P     & T          & 96.27      & 0.000140762 & 12.3711  \\
            OKP5     & V     & P     & F          & 64.23      & 0.000298357 & 18.5567  \\
            OKP5     & V     & H     & T          & 92.46      & 0.000273609 & 12.3711  \\
            OKP5     & V     & H     & F          & 83.79      & 0.000238085 & 19.5876  \\
            OKP5     & V     & W     & T          & 98.17      & 0.000152016 & 17.5258  \\
            OKP5     & V     & W     & F          & 74.86      & 0.000524998 & 24.7423  \\
            OKP5     & V     & I     & T          & 82.92      & 0.000376892 & 19.5876  \\
            OKP5     & V     & I     & F          & 76.46      & 0.000355959 & 18.5567  \\
            OKP5     & H     & A     & T          & 87.79      & 0.000333214 & 13.4021  \\
            OKP5     & H     & A     & F          & 62.2       & 0.000425053 & 21.6495  \\
            OKP5     & H     & P     & T          & 96.27      & 0.000154543 & 12.3711  \\
            OKP5     & H     & P     & F          & 75.96      & 0.000391293 & 21.6495  \\
            OKP5     & H     & H     & T          & 92.46      & 0.000174999 & 12.3711  \\
            OKP5     & H     & H     & F          & 58.19      & 0.000386477 & 15.4639  \\
            OKP5     & H     & W     & T          & 98.17      & 0.000229216 & 17.5258  \\
            OKP5     & H     & W     & F          & 70.21      & 0.00038805  & 22.6804  \\
            OKP5     & H     & I     & T          & 71.31      & 0.000295877 & 16.4948  \\
            OKP5     & H     & I     & F          & 80.3       & 0.00025239  & 17.5258  \\
            OKP5     & M     & A     & T          & 86.82      & 0.000329733 & 11.3402  \\
            OKP5     & M     & A     & F          & 58.74      & 0.000707722 & 22.6804  \\
            OKP5     & M     & P     & T          & 96.27      & 0.000255251 & 12.3711  \\
            OKP5     & M     & P     & F          & 69.06      & 0.000460052 & 17.5258  \\
            OKP5     & M     & H     & T          & 92.46      & 0.000409222 & 12.3711  \\
            OKP5     & M     & H     & F          & 79.11      & 0.000539827 & 18.5567  \\
            OKP5     & M     & W     & T          & 98.17      & 0.000410271 & 17.5258  \\
            OKP5     & M     & W     & F          & 74.86      & 0.0008008   & 24.7423  \\
            OKP5     & M     & I     & T          & 77.06      & 0.00061326  & 20.6186  \\
            OKP5     & M     & I     & F          & 66.86      & 0.000523663 & 16.4948  \\
            OKP5     & N     & A     & T          & 84.86      & 0.00472121  & 10.3093  \\
            OKP5     & N     & A     & F          & 83.74      & 0.101231    & 27.8351  \\
            OKP5     & N     & P     & T          & 96.27      & 0.00186729  & 12.3711  \\
            OKP5     & N     & P     & F          & 72.24      & 0.0847682   & 19.5876  \\
            OKP5     & N     & H     & T          & 92.46      & 0.00197048  & 12.3711  \\
            OKP5     & N     & H     & F          & 74.15      & 0.0703918   & 17.5258  \\
            OKP5     & N     & W     & T          & 98.17      & 0.00453982  & 17.5258  \\
            OKP5     & N     & W     & F          & 74.86      & 0.0895785   & 24.7423  \\
            OKP5     & N     & I     & T          & 82.92      & 0.0148974   & 19.5876  \\
            OKP5     & N     & I     & F          & 89.9       & 0.0277001   & 19.5876  \\
            \hline
        \end{tabular}
    }{}
    \fonte{autor}
\end{table}



\chapter{Código fonte}\label{ch:codigo-fonte}
Código fonte disponível em \url{https://github.com/G-Carneiro/packing-problem/}.


\section{LICENSE}\label{sec:license}
\inputminted{text}{LICENSE}


\section{main.py}\label{sec:main.py}
\inputminted{py}{main.py}


\section{src/}\label{sec:src/}

\subsection{model/}\label{subsec:model/}

\subsubsection{coordinate.py}\label{subsubsec:coordinate.py}
\inputminted{py}{src/model/coordinate.py}

\subsubsection{item.py}\label{subsubsec:item.py}
\inputminted{py}{src/model/item.py}

\subsubsection{model.py}\label{subsubsec:model.py}
\inputminted{py}{src/model/model.py}

\subsubsection{order\_mode.py}\label{subsubsec:order_mode.py}
\inputminted{py}{src/model/order_mode.py}

\subsubsection{ordered\_queue.py}\label{subsubsec:ordered_queue.py}
\inputminted{py}{src/model/ordered_queue.py}

\subsubsection{rect.py}\label{subsubsec:rect.py}
\inputminted{py}{src/model/rect.py}

\subsubsection{region.py}\label{subsubsec:region.py}
\inputminted{py}{src/model/region.py}

\subsection{utils/}\label{subsec:utils/}

\subsubsection{descending.py}\label{subsubsec:descending.py}
\inputminted{py}{src/utils/descending.py}

\subsubsection{folders.py}\label{subsubsec:folders.py}
\inputminted{py}{src/utils/folders.py}

\subsubsection{functions.py}\label{subsubsec:functions.py}
\inputminted{py}{src/utils/functions.py}

\subsubsection{instances.py}\label{subsubsec:instances.py}
\inputminted{py}{src/utils/instances.py}

\subsubsection{order\_key.py}\label{subsubsec:order_key.py}
\inputminted{py}{src/utils/order_key.py}

\subsubsection{split\_mode.py}\label{subsubsec:split_mode.py}
\inputminted{py}{src/utils/split_mode.py}


% --------------------------------------------
% Anexos. (opcional)
% --------------------------------------------
\anexos

% --------------------------------------------
% Índice. (opcional)
% --------------------------------------------
% TODO: needs MakeIndex compiler.
%\printindex

\end{document}
% --------------------------------------------
% Final do documento. Não adicionar nada após.
% --------------------------------------------