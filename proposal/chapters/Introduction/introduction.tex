% --------------------------------------------
% Aqui você deve organizar as seções.
% --------------------------------------------
Problemas de empacotamento estão na categoria NP-difícil e possuem cada vez mais relevância no cenário global.
Eles possuem uma grande variedade de aplicações, como a organização de estoque em uma pequena mercearia, alocação milhares de produtos em containers (o que pode tornar o problema recursivo, dado que geralmente os produtos são embalados em caixas, que também podem ter seu espaço otimizado), disposição de móveis em um cômodo, etc.
De forma geral, eles variam em dimensão, 2D ou 3D, categorias de peças, regulares ou irregulares, e método de solução, heurísticos ou exatos.

Quando se trata de um problema tridimensional, encontrar uma solução se torna extremante complicado, nesses casos geralmente usamos um método heurístico para encontrar uma solução boa, mas não necessariamente ótima.

Portanto, dada a importância do problema, este trabalho visa encontrar e comparar os melhores métodos de solução atuais, além de verificar em quais situações se obtêm um melhor ganho com cada algoritmo.

%\section{Motivação}\label{sec:Motivacao}
%% --------------------------------------------
% Aqui você deve escrever o texto.
% --------------------------------------------


\section{Motivação}\label{sec:Motivacao}

%
%\section{Justificativas}\label{sec:Justificativas}
%% --------------------------------------------
% Aqui você deve escrever o texto.
% --------------------------------------------
Escreva aqui.\cite{einstein}

\section{Objetivos}\label{sec:Objetivos}
% --------------------------------------------
% Aqui você deve organizar as subsubseções.
% --------------------------------------------
\section*{Objetivos}\label{sec:Objetivos}

O principal objetivo deste trabalho é estudar e compreender o problema de empacotamento bem como suas aplicações no mundo real.
Outros objetivos mais específicos são: revisar a bibliografia, implementar heurísticas baseadas em \textit{bottom-left}, definir instâncias de teste e analisar os dados obtidos e compará-los com os de outros autores.
