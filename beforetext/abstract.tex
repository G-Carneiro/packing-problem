% --------------------------------------------
% Resumo do documento de acordo com padrões
% Latex. Caso queira de acordo com a ABNT,
% comente as linhas abaixo.
% --------------------------------------------
%\begin{abstract}
%    Write here.
%\end{abstract}

% --------------------------------------------
% Comentar linhas abaixo para não ter um
% resumo de acordo com a ABNT.
% --------------------------------------------
\begin{resumo}
    Problemas de empacotamento consistem em alocar um conjunto de itens $\mathcal{I}$ em uma
    caixa $\mathcal{B}$.
    No problema de empacotamento da mochila, foco deste trabalho, cada item é associado a um valor
    e busca-se uma solução que maximize a soma dos valores dos itens alocados.
    Este trabalho compara 40 métodos de solução criados com base na heurística construtiva
    \textit{bottom-left} para o problema de empacotamento de retângulos.
    A escolha dessa heurística se deve a sua simplicidade e a dificuldade de usar métodos exatos
    para resolução do problema em tempo hábil.
    Os métodos criados são uma combinação de diferentes formas de ordenação dos itens e criação
    de regiões, as quais evitam as sobreposições e o domínio contínuo presentes no problema.
    Algoritmos foram implementados em Python e testados com instâncias da literatura, dados como
    qualidade de solução, porcentagem de itens alocados e tempo de execução foram coletados.
    O principal resultado foi a alta competitividade de diferentes modos de ordenação,
    não sendo a área a única relevante, com o perímetro obtendo os melhores resultados.

    \textbf{Palavras-chave: problema de empacotamento, \textit{bottom-left}, heurística,
        pesquisa operacional.}
\end{resumo}
\newpage

% --------------------------------------------
% Caso seja necessário um resumo em inglês,
% descomentar linhas abaixo.
% --------------------------------------------
\begin{resumo}[Abstract]
    Packing problems consist of allocating a set of items $\mathcal{I}$ into a box $\mathcal{B}$.
    In the knapsack packing problem, the focus of this work, each item is associated with a value
    and a solution is sought that maximizes the sum of the values of the allocated items.
    This work compare 40 created solution methods based on \textit{bottom-left} constructive
    heuristic for the rectangle packing problem.
    The choice of this heuristic is due to its simplicity and the difficulty of using exact methods
    to solve the problem in a timely manner.
    The methods created are a combination of different ways of ordering items and creating regions,
    which avoid superposition and continuous domain present in the problem.
    Algorithms were implemented in Python and tested with instances from the literature,
    data such as solution quality, percentage of allocated items and execution time were collected.
    The main result was the high competitiveness of different ordering modes, the area not being
    the only relevant one, with the perimeter obtaining the best results.

    \textbf{Keywords: packing problem, \textit{bottom-left}, heuristic, operational research.}
\end{resumo}
\newpage
